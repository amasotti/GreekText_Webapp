% PREAMBOLO
\documentclass[twocolumn,12pt]{article}

%----------- IMMANCABILI ------------
\usepackage{graphicx}
\usepackage{float}
\usepackage{tipa}
\usepackage[dvipsnames]{xcolor}
\usepackage{hyperref}
%------------ MATEMATICA--------------
\usepackage{amsmath}
\usepackage{amsfonts}
\usepackage{amssymb}

% ------------ LINGUE ---------------
\usepackage{polyglossia}
\setdefaultlanguage[babelshorthands=true]{german}
\setotherlanguage[variant=ancient]{greek}

\usepackage{fontspec}
\usepackage{libertine}

%----------------FONTS-----------------------
\newfontfamily{\greekfont}{Palatino Linotype} 
\newfontfamily{\greekfontrm}{Palatino Linotype}
\newfontfamily{\greekfontsf}{Candara}
\newfontfamily{\greeksansfont}{Times New Roman}

%-------------- LINGUISTICS --------------
\usepackage{linguex}

%------------- LAYOUT --------------------------
\newcommand{\mydate}{\today}
\newcommand{\mytext}{\textsc{Gr. Wortformen \hspace{13cm} \textsc{A. Masotti}}}
\renewcommand{\baselinestretch}{1} 

\usepackage{tikz}
\usepackage{placeins}
%Geometry Einstellungen und Page Layout
\usepackage{eso-pic}
\usepackage[
left=2cm,
right=2cm,
top=2cm,
bottom=1.5cm,
headsep=2cm]{geometry}
%\usepackage{fancyhdr}
\AddToShipoutPicture{%
    \AtTextUpperLeft{%
        \makebox(8,30)[l]{%
            \textbf{\mytext}%
            \textbf{\hfill}%
            \vspace{1cm}%
        }\makebox(30,20)[tl]{\begin{tikzpicture}
            \draw (-1,-1) -- (17,-1);
            \end{tikzpicture}}}}
%\pagestyle{fancy}


%------------- COMANDI PERSONALI -----------------------
\newcommand{\wortform}[1]{\textgreek{\textbf{{\large #1 }}}}

\newcommand{\glossa}[1]{\textgreek{,#1‘ }}
\newcommand{\tradu}[1]{\textcolor{teal}{\small \sl »#1« }}
\newcommand{\morphobestimmung}[1]{{\footnotesize #1 }}

\newcommand{\eintrag}[4]{\wortform{#1} (siehe \glossa{#2}, \tradu{ #3 }) : \morphobestimmung{#4} }

%-------------------------- DOC INFO --------------------------------------
\title{Griechische Wortformen}
\author{A. Masotti \\ {\footnotesize Created with Flask \& SQLAlchemy}}


\begin{document}
\maketitle
{\Large Query:  }

\begin{itemize}
  \item \eintrag{·}{ }{keine Übersetzung gefunden}{Nichts gefunden}
 \item \eintrag{Αἰγύπτου}{ Αἴγυπτος }{the river Nile}{noun sg masc gen OR noun sg fem gen}
 \item \eintrag{Αἰγύπτῳ}{ Αἴγυπτος }{the river Nile}{noun sg masc dat OR noun sg fem dat}
 \item \eintrag{Αἰτωλοί}{ Αἰτωλός }{Aetolian}{adj pl masc voc OR adj pl masc nom}
 \item \eintrag{Αἰτωλοὶ}{ Αἰτωλός }{Aetolian}{adj pl masc voc OR adj pl masc nom}
 \item \eintrag{Αἰτωλοὺς}{ Αἰτωλός }{Aetolian}{adj pl masc acc}
 \item \eintrag{Αἰτωλοῖς}{ Αἰτωλός }{Aetolian}{adj pl neut dat OR adj pl masc dat}
 \item \eintrag{Αἰτωλῶν}{ Αἰτωλός }{Aetolian}{adj pl neut gen OR adj pl masc gen OR adj pl fem gen}
 \item \eintrag{Αἴγυπτον}{ Αἴγυπτος }{the river Nile}{noun sg masc acc OR noun sg fem acc}
 \item \eintrag{Αὐτὸς}{ αὐτός }{self}{adj sg masc nom}
 \item \eintrag{Βαστέρνας}{ }{keine Übersetzung gefunden}{Nichts gefunden}
 \item \eintrag{Βοιωτοῖς}{ Βοιωτός }{a Boeotian}{noun pl masc dat}
 \item \eintrag{Βρεντεσίου}{ }{keine Übersetzung gefunden}{Nichts gefunden}
 \item \eintrag{Βρεντεσίῳ}{ }{keine Übersetzung gefunden}{Nichts gefunden}
 \item \eintrag{Βυζαντίοις}{ }{keine Übersetzung gefunden}{Nichts gefunden}
 \item \eintrag{Γένθιος}{ }{keine Übersetzung gefunden}{Nichts gefunden}
 \item \eintrag{Γέταις}{ }{keine Übersetzung gefunden}{Nichts gefunden}
 \item \eintrag{Γέτας}{ }{keine Übersetzung gefunden}{Nichts gefunden}
 \item \eintrag{Γετῶν}{ }{keine Übersetzung gefunden}{Nichts gefunden}
 \item \eintrag{Δελφοὺς}{ Δελφοί }{Delphi,}{noun pl masc acc}
 \item \eintrag{Δελφοῖς}{ Δελφοί }{Delphi,}{noun pl masc dat}
 \item \eintrag{Δεξαμένου}{ δέχομαι δείκνυμι }{take, accept, receive,}{part sg aor mid neut gen OR part sg aor mid masc gen OR part sg aor mid neut gen ionic OR part sg aor mid masc gen ionic}
 \item \eintrag{Δημήτριον}{ Δημήτριος }{of}{adj sg neut voc OR noun sg masc acc OR adj sg neut nom OR adj sg neut acc OR adj sg masc acc OR adj sg fem acc}
 \item \eintrag{Δημήτριος}{ Δημήτριος }{of}{noun sg masc nom OR adj sg masc nom OR adj sg fem nom}
 \item \eintrag{Δημητριάδι}{ Δημητριάς }{named in honour of Demetrius Poliorcetes,}{noun sg fem dat}
 \item \eintrag{Δόλοπες}{ Δόλοψ }{Dolopians}{noun pl masc nom OR noun pl masc voc}
 \item \eintrag{Δόλοψι}{ Δόλοψ }{Dolopians}{noun pl masc dat epic}
 \item \eintrag{ΕΚ}{ ἐκ }{from out of,}{prep proclitic indeclform OR prep proclitic indeclform}
 \item \eintrag{Εὐβοεῦσι}{ Εὐβοεύς }{an Euboean}{noun pl masc dat}
 \item \eintrag{Εὐμένη}{ Εὐμενέω Εὐμενής εὐμενέω εὐμενής }{to be gracious}{verb 3rd sg imperf ind act doric aeolic contr OR verb 2nd sg pres imperat act doric aeolic contr OR adj sg masc acc attic epic doric contr OR adj sg fem acc attic epic doric contr OR adj pl neut voc attic epic doric contr OR adj pl neut nom attic epic doric contr OR adj pl neut acc attic epic doric contr OR adj dual neut voc doric aeolic contr OR adj dual neut nom doric aeolic contr OR adj dual neut acc doric aeolic contr OR adj dual masc voc doric aeolic contr OR adj dual masc nom doric aeolic contr OR adj dual masc acc doric aeolic contr OR adj dual fem voc doric aeolic contr OR adj dual fem nom doric aeolic contr OR adj dual fem acc doric aeolic contr OR verb 3rd sg imperf ind act doric aeolic poetic contr unaugmented OR verb 2nd sg pres imperat act doric aeolic contr OR adj sg masc acc attic epic doric contr OR adj sg fem acc attic epic doric contr OR adj pl neut voc attic epic doric contr OR adj pl neut nom attic epic doric contr OR adj pl neut acc attic epic doric contr OR adj dual neut voc doric aeolic contr OR adj dual neut nom doric aeolic contr OR adj dual neut acc doric aeolic contr OR adj dual masc voc doric aeolic contr OR adj dual masc nom doric aeolic contr OR adj dual masc acc doric aeolic contr OR adj dual fem voc doric aeolic contr OR adj dual fem nom doric aeolic contr OR adj dual fem acc doric aeolic contr}
 \item \eintrag{Εὐμένης}{ Εὐμενέω Εὐμενής εὐμενέω εὐμενής }{to be gracious}{verb 2nd sg pres ind act doric contr OR verb 2nd sg imperf ind act doric aeolic contr OR adj sg masc nom OR adj sg fem nom OR adj pl masc voc doric aeolic contr OR adj pl masc nom doric aeolic contr OR adj pl masc acc attic epic doric contr OR adj pl fem nom doric aeolic contr OR adj pl fem voc doric aeolic contr OR adj pl fem acc attic epic doric contr OR verb 2nd sg pres ind act doric contr OR verb 2nd sg imperf ind act doric aeolic poetic contr unaugmented OR adj sg masc nom OR adj pl masc voc doric aeolic contr OR adj sg fem nom OR adj pl masc nom doric aeolic contr OR adj pl masc acc attic epic doric contr OR adj pl fem voc doric aeolic contr OR adj pl fem nom doric aeolic contr OR adj pl fem acc attic epic doric contr}
 \item \eintrag{Εὐμένους}{ Εὐμενής εὐμενής }{well-disposed, kindly}{adj sg neut gen attic epic doric contr OR adj sg masc gen attic epic doric contr OR adj sg fem gen attic epic doric contr OR adj sg neut gen attic epic doric contr OR adj sg masc gen attic epic doric contr OR adj sg fem gen attic epic doric contr}
 \item \eintrag{Θεσσαλίαν}{ Θεσσαλία }{Thessaly}{noun sg fem acc attic doric aeolic OR noun pl fem gen doric aeolic}
 \item \eintrag{Θεσσαλοῖς}{ Θεσσαλός }{shoe}{noun pl masc dat OR adj pl neut dat OR adj pl masc dat}
 \item \eintrag{Θετταλοὺς}{ Θεσσαλός }{shoe}{noun pl masc acc attic}
 \item \eintrag{Θρᾴκην}{ }{keine Übersetzung gefunden}{Nichts gefunden}
 \item \eintrag{Θρᾴκης}{ }{keine Übersetzung gefunden}{Nichts gefunden}
 \item \eintrag{Θρᾷκας}{ Θρᾷξ }{a Thracian;}{noun pl masc acc}
 \item \eintrag{Θρᾷκες}{ Θρᾷξ }{a Thracian;}{noun pl masc voc OR noun pl masc nom}
 \item \eintrag{Κέρκυραν}{ Κέρκυρα }{BMus.Cat.Coins Thessaly}{noun sg fem acc OR noun pl fem gen doric aeolic}
 \item \eintrag{Κίρρας}{ Κίρρα κιρράς κιρρός }{keine Übersetzung gefunden}{noun sg fem gen attic doric aeolic OR noun pl fem acc OR noun sg fem nom OR adj sg fem gen attic doric aeolic OR adj pl fem acc}
 \item \eintrag{Καπιτώλιον}{ Καπετώλιον }{victor in the Ludi Capitolini,}{noun sg neut voc OR noun sg neut acc OR noun sg neut nom}
 \item \eintrag{Καρχηδονίων}{ Καρχηδονίζω Καρχηδών }{side with the Carthaginians}{part sg fut act masc nom attic epic doric contr OR adj pl neut gen OR adj pl masc gen OR adj pl fem gen}
 \item \eintrag{Καρχηδόνα}{ Καρχηδών }{Carthage}{noun sg fem acc}
 \item \eintrag{Καὶ}{ καί καί2 καί3 καί4 }{and}{conj indeclform OR conj indeclform OR conj indeclform OR conj indeclform}
 \item \eintrag{Κερκύρας}{ Κέρκυρα }{BMus.Cat.Coins Thessaly}{noun sg fem gen attic doric aeolic OR noun pl fem acc}
 \item \eintrag{Κερκύρᾳ}{ Κέρκυρα }{BMus.Cat.Coins Thessaly}{noun sg fem dat attic doric aeolic OR noun pl fem nom OR noun pl fem voc}
 \item \eintrag{Κλοίλιον}{ }{keine Übersetzung gefunden}{Nichts gefunden}
 \item \eintrag{Κλοίλιος}{ }{keine Übersetzung gefunden}{Nichts gefunden}
 \item \eintrag{Κλοιλίῳ}{ }{keine Übersetzung gefunden}{Nichts gefunden}
 \item \eintrag{Κοΐντιος}{ }{keine Übersetzung gefunden}{Nichts gefunden}
 \item \eintrag{Κορίνθῳ}{ Κόρινθος }{at}{noun sg fem dat}
 \item \eintrag{Κράσσῳ}{ }{keine Übersetzung gefunden}{Nichts gefunden}
 \item \eintrag{Κυκλάδας}{ κυκλάς }{encircling}{noun pl fem acc}
 \item \eintrag{Κυρήνην}{ Κυρήνη }{Cyrene}{noun sg fem acc attic epic ionic}
 \item \eintrag{Κόρινθον}{ Κόρινθος }{at}{noun sg fem acc}
 \item \eintrag{Κύπρον}{ Κύπρος κύπρος }{from Cyprus,}{noun sg fem acc OR noun sg fem acc}
 \item \eintrag{Λακεδαιμονίων}{ Λακεδαιμόνιος Λακεδαιμονιάζω }{Spartan}{adj pl neut gen OR adj pl masc gen OR adj pl fem gen OR part sg fut act neut voc contr OR part sg fut act neut acc contr OR part sg fut act neut nom contr OR part sg fut act masc voc contr OR part sg fut act masc nom attic epic ionic contr}
 \item \eintrag{Λευκίου}{ }{keine Übersetzung gefunden}{Nichts gefunden}
 \item \eintrag{Λευκίῳ}{ }{keine Übersetzung gefunden}{Nichts gefunden}
 \item \eintrag{Λεύκιος}{ }{keine Übersetzung gefunden}{Nichts gefunden}
 \item \eintrag{Λιβύην}{ Λιβύη }{the west bank of the Nile,}{noun sg fem acc attic epic ionic}
 \item \eintrag{Λοκροῖς}{ Λοκρός Λοκροί }{Locrian}{adj pl neut dat OR adj pl masc dat OR noun pl masc dat}
 \item \eintrag{ΜΑΚΕΔΟΝΙΚΗΣ}{ }{keine Übersetzung gefunden}{Nichts gefunden}
 \item \eintrag{Μάγνησιν}{ Μάγνης }{the magnet}{noun pl masc dat nu movable}
 \item \eintrag{Μάξιμόν}{ }{keine Übersetzung gefunden}{Nichts gefunden}
 \item \eintrag{Μάρκιον}{ }{keine Übersetzung gefunden}{Nichts gefunden}
 \item \eintrag{Μάρκιος}{ }{keine Übersetzung gefunden}{Nichts gefunden}
 \item \eintrag{Μακεδονίαν}{ Μακεδονία }{keine Übersetzung gefunden}{noun pl fem gen doric aeolic OR noun sg fem acc attic doric aeolic}
 \item \eintrag{Μακεδονίας}{ Μακεδονία }{keine Übersetzung gefunden}{noun sg fem gen attic doric aeolic OR noun pl fem acc}
 \item \eintrag{Μακεδονίᾳ}{ Μακεδονία }{keine Übersetzung gefunden}{noun pl fem voc OR noun sg fem dat attic doric aeolic OR noun pl fem nom}
 \item \eintrag{Μακεδόνας}{ Μακεδών }{a Macedonian}{noun pl masc acc OR noun pl fem acc}
 \item \eintrag{Μακεδόνος}{ Μακεδών }{a Macedonian}{noun sg masc gen OR noun sg fem gen}
 \item \eintrag{Μακεδόνων}{ Μακεδών }{a Macedonian}{noun pl masc gen OR noun pl fem gen}
 \item \eintrag{Μακεδόσι}{ Μακεδών }{a Macedonian}{noun pl fem dat OR noun pl masc dat}
 \item \eintrag{Μακεδών}{ Μακεδών }{a Macedonian}{noun sg masc voc OR noun sg masc nom OR noun sg fem voc OR noun sg fem nom}
 \item \eintrag{Μακηδόνες}{ }{keine Übersetzung gefunden}{Nichts gefunden}
 \item \eintrag{Μασσανάσσην}{ }{keine Übersetzung gefunden}{Nichts gefunden}
 \item \eintrag{Μηλιέα}{ Μηλιεύς }{inhabitant of Malis}{noun sg masc acc}
 \item \eintrag{Μιτυληναίων}{ }{keine Übersetzung gefunden}{Nichts gefunden}
 \item \eintrag{Νάβιδος}{ }{keine Übersetzung gefunden}{Nichts gefunden}
 \item \eintrag{Νικίαν}{ }{keine Übersetzung gefunden}{Nichts gefunden}
 \item \eintrag{Ξενοφάνης}{ }{keine Übersetzung gefunden}{Nichts gefunden}
 \item \eintrag{Οἱ}{ ἕ ὅς ὅς ὁ }{sui.}{pron sg masc dat epic ionic enclitic indeclform OR pron sg fem dat epic ionic enclitic indeclform OR pron pl masc nom indeclform OR pron pl masc nom indeclform OR article pl masc voc proclitic indeclform OR article pl masc nom proclitic indeclform}
 \item \eintrag{Οὐ}{ οὐ οὐ10 οὐ11 οὐ12 οὐ13 οὐ14 οὐ15 οὐ16 οὐ17 οὐ18 οὐ2 οὐ3 οὐ4 οὐ5 οὐ6 οὐ7 οὐ8 οὐ9 }{fact}{adv proclitic indeclform OR adv proclitic indeclform OR adv proclitic indeclform OR adv proclitic indeclform OR adv proclitic indeclform OR adv proclitic indeclform OR adv proclitic indeclform OR adv proclitic indeclform OR adv proclitic indeclform OR adv proclitic indeclform OR adv proclitic indeclform OR adv proclitic indeclform OR adv proclitic indeclform OR adv proclitic indeclform OR adv proclitic indeclform OR adv proclitic indeclform OR adv proclitic indeclform OR adv proclitic indeclform}
 \item \eintrag{Πέργαμον}{ Πέργαμον Πέργαμος γαμέω }{citadel, acropolis}{noun sg neut voc OR noun sg neut nom OR noun sg neut acc OR noun sg fem acc OR verb 2nd sg aor imperat act doric poetic elide preverb}
 \item \eintrag{Παύλῳ}{ }{keine Übersetzung gefunden}{Nichts gefunden}
 \item \eintrag{Παῦλος}{ }{keine Übersetzung gefunden}{Nichts gefunden}
 \item \eintrag{Πελοπόννησον}{ Πελοπόννησος }{the Peloponnesus,}{noun sg fem acc}
 \item \eintrag{Περγάμου}{ Πέργαμον Πέργαμος }{citadel, acropolis}{noun sg neut gen OR noun sg fem gen}
 \item \eintrag{Περπένναν}{ }{keine Übersetzung gefunden}{Nichts gefunden}
 \item \eintrag{Περραιβοὺς}{ ῥαιβόω }{make crooked, bend}{verb 2nd sg pres ind act doric poetic contr elide preverb OR verb 2nd sg imperf ind act homeric ionic poetic contr unaugmented elide preverb}
 \item \eintrag{Περσέα}{ Πέρσης Περσεύς πέρσις περσέα }{a throw on the dice}{noun sg masc acc epic ionic OR noun sg masc acc OR noun sg fem acc poetic OR noun sg fem voc attic doric aeolic OR noun sg fem nom attic doric aeolic OR noun dual fem voc OR noun dual fem acc OR noun dual fem nom}
 \item \eintrag{Περσέως}{ Περσεύς πέρσις }{a fish}{noun sg masc gen epic doric ionic aeolic OR noun sg masc nom epic ionic OR noun sg fem gen attic}
 \item \eintrag{Περσεύς}{ Περσεύς }{a fish}{noun sg masc nom OR noun sg masc gen epic contr}
 \item \eintrag{Περσεὺς}{ Περσεύς }{a fish}{noun sg masc nom OR noun sg masc gen epic contr}
 \item \eintrag{Περσεῖ}{ Περσεύς πέρθω πέρσις }{a fish}{noun sg masc dat epic poetic OR verb 3rd sg fut ind act doric contr OR verb 2nd sg fut ind mid OR verb 3rd sg aor subj act epic short subj OR noun sg fem dat epic OR noun dual fem voc attic epic contr OR noun dual fem nom attic epic contr OR noun dual fem acc attic epic contr}
 \item \eintrag{Πετίλιον}{ }{keine Übersetzung gefunden}{Nichts gefunden}
 \item \eintrag{Ποιμὴν}{ ποιμήν }{herdsman,}{noun sg masc voc OR noun sg masc nom}
 \item \eintrag{Ποπλίου}{ }{keine Übersetzung gefunden}{Nichts gefunden}
 \item \eintrag{Πτολεμαίου}{ Πτολεμαῖος }{keine Übersetzung gefunden}{noun sg masc gen}
 \item \eintrag{Πτολεμαῖον}{ Πτολεμαῖος }{keine Übersetzung gefunden}{noun sg masc acc}
 \item \eintrag{Πτολεμαῖος}{ Πτολεμαῖος }{keine Übersetzung gefunden}{noun sg masc nom}
 \item \eintrag{Σάμον}{ Σάμος }{a height.}{noun sg fem acc}
 \item \eintrag{Σαμοθρᾴκῃ}{ Σαμοθρᾴκη }{Samothrace}{noun sg fem dat attic epic ionic}
 \item \eintrag{Σιβύλλεια}{ Σιβύλλειος }{Sibylline}{adj sg fem voc attic doric aeolic OR adj sg fem nom attic doric aeolic OR adj pl neut nom OR adj pl neut voc OR adj pl neut acc OR adj dual fem voc OR adj dual fem acc OR adj dual fem nom}
 \item \eintrag{Σικελίαν}{ Σικελία }{Sicily}{noun sg fem acc attic doric aeolic OR noun pl fem gen doric aeolic}
 \item \eintrag{Σκιπίωνα}{ }{keine Übersetzung gefunden}{Nichts gefunden}
 \item \eintrag{Σουλπίκιος}{ }{keine Übersetzung gefunden}{Nichts gefunden}
 \item \eintrag{Σουλπικίου}{ }{keine Übersetzung gefunden}{Nichts gefunden}
 \item \eintrag{Σύρων}{ Σύρα Σύρος Σύρος Σῦρος σύρα σύρον σύρος σύρω }{a Syrian woman}{noun pl fem gen OR noun pl masc gen OR noun pl masc gen OR noun pl masc gen OR noun pl fem gen OR noun pl neut gen OR noun pl masc gen OR part sg pres act masc nom OR part sg fut act masc nom attic epic doric contr}
 \item \eintrag{ΤΗΣ}{ εἰς ὁ }{into}{prep proclitic indeclform OR prep proclitic indeclform OR article pl fem dat epic indeclform OR article sg fem gen attic epic ionic indeclform}
 \item \eintrag{Ταῦτα}{ οὗτος }{this}{adj pl neut voc indeclform OR adj pl neut acc indeclform OR adj pl neut nom indeclform OR adj dual fem acc indeclform}
 \item \eintrag{Τοιγάρτοι}{ τοιγάρ }{therefore, accordingly, well then,}{partic indeclform}
 \item \eintrag{Τότε}{ τότε τοτέ τοτέ2 }{at that time, then,}{adv indeclform OR adv indeclform OR adv indeclform}
 \item \eintrag{Τὸ}{ ὁ }{the following}{article sg neut nom indeclform OR article sg neut voc indeclform OR article sg neut acc indeclform}
 \item \eintrag{Φίλιππον}{ φίλιππος }{fond of horses, horse-loving,}{adj sg neut nom OR adj sg neut voc OR adj sg neut acc OR adj sg fem acc OR adj sg masc acc}
 \item \eintrag{Φίλιππος}{ φίλιππος }{fond of horses, horse-loving,}{adj sg masc nom OR adj sg fem nom}
 \item \eintrag{Φίλᾳ}{ φίλος φίλος φιλέω }{loved, beloved, dear}{adj sg fem dat doric aeolic OR adj pl fem voc OR adj pl fem nom OR adj sg fem dat doric aeolic OR adj pl fem voc OR adj pl fem nom OR verb aor inf act epic OR verb 2nd sg aor imperat mid epic}
 \item \eintrag{Φιλίππου}{ φίλιππος φιλιππέω }{fond of horses, horse-loving,}{adj sg neut gen OR adj sg masc gen OR adj sg fem gen OR verb 2nd sg pres imperat mp attic contr OR verb 2nd sg imperf ind mp attic poetic contr unaugmented}
 \item \eintrag{Φιλίππῳ}{ φίλιππος }{fond of horses, horse-loving,}{adj sg masc dat OR adj sg neut dat OR adj sg fem dat}
 \item \eintrag{Φλαμινίνου}{ }{keine Übersetzung gefunden}{Nichts gefunden}
 \item \eintrag{Φλαμινίνῳ}{ }{keine Übersetzung gefunden}{Nichts gefunden}
 \item \eintrag{Φλαμινῖνον}{ }{keine Übersetzung gefunden}{Nichts gefunden}
 \item \eintrag{Φλαμινῖνος}{ }{keine Übersetzung gefunden}{Nichts gefunden}
 \item \eintrag{Φωκίδος}{ Φωκίς φωκίς }{Phocis}{noun sg fem gen OR noun sg fem gen}
 \item \eintrag{Χίον}{ Χίος Χῖος Χῖος χάω χῖον χιών }{Chios,}{noun sg fem acc OR adj sg neut voc OR adj sg neut nom OR adj sg neut acc OR adj sg masc acc OR adj sg neut voc OR adj sg neut nom OR adj sg neut acc OR adj sg masc acc OR verb 3rd pl imperf ind act epic doric ionic poetic unaugmented OR verb 1st sg imperf ind act epic doric ionic poetic unaugmented OR part sg pres act neut voc epic doric ionic OR part sg pres act neut nom epic doric ionic OR part sg pres act neut acc epic doric ionic OR part sg pres act masc voc epic doric ionic OR noun sg neut voc OR noun sg neut nom OR noun sg neut acc OR noun sg fem voc}
 \item \eintrag{Χίων}{ Χίος Χῖος Χῖος χάω χῖον χιάζω χιόω χιών }{Chios,}{noun pl fem gen OR adj pl neut gen OR adj pl masc gen OR adj pl fem gen OR adj pl neut gen OR adj pl masc gen OR adj pl fem gen OR part sg pres act masc nom epic doric ionic OR noun pl neut gen OR part sg fut act neut voc contr OR part sg fut act neut acc contr OR part sg fut act neut nom contr OR part sg fut act masc voc contr OR part sg fut act masc nom attic epic ionic contr OR verb pres inf act doric OR verb 3rd pl imperf ind act doric aeolic poetic contr unaugmented OR part sg pres act neut voc doric aeolic contr OR verb 1st sg imperf ind act doric aeolic poetic contr unaugmented OR part sg pres act neut acc doric aeolic contr OR part sg pres act neut nom doric aeolic contr OR part sg pres act masc voc doric aeolic contr OR part sg pres act masc nom contr OR noun sg fem nom OR noun sg fem voc}
 \item \eintrag{Χαλκίδι}{ Χαλκίς χαλκίς }{Chalcis}{noun sg fem dat OR noun sg fem dat}
 \item \eintrag{αἰδεῖσθαι}{ αἰδέομαι }{to be ashamed}{verb pres inf mp attic epic contr}
 \item \eintrag{αἰδεῖσθε}{ αἰδέομαι }{to be ashamed}{verb 2nd pl pres opt mp epic ionic OR verb 2nd pl pres ind mp attic epic contr OR verb 2nd pl pres imperat mp attic epic contr OR verb 2nd pl imperf ind mp attic epic contr unaugmented}
 \item \eintrag{αἰδουμένη}{ αἰδέομαι }{to be ashamed}{part sg pres mp fem nom attic epic contr OR part sg pres mp fem voc attic epic contr}
 \item \eintrag{αἰδούμενος}{ αἰδέομαι }{to be ashamed}{part sg pres mp masc nom attic epic doric contr}
 \item \eintrag{αἰκισάμενοι}{ αἰκίζω }{maltreat}{part pl aor mid masc voc attic epic OR part pl aor mid masc nom attic epic}
 \item \eintrag{αἰσθόμενος}{ αἰσθάνομαι }{perceive, apprehend by the senses}{part sg pres mp masc nom attic later OR part sg aor mid masc nom}
 \item \eintrag{αἰσχροῦ}{ αἰσχρός }{causing shame, dishonouring, reproachful}{adj sg neut gen OR adj sg masc gen OR adj sg fem gen}
 \item \eintrag{αἰσχρὸν}{ αἰσχρός }{causing shame, dishonouring, reproachful}{adj sg neut nom OR adj sg neut voc OR adj sg neut acc OR adj sg fem acc OR adj sg masc acc}
 \item \eintrag{αἰσχύνην}{ αἰσχύνη αἰσχύνω }{shame, dishonour}{noun sg fem acc attic epic ionic OR verb fut inf act epic doric contr OR verb pres inf act doric aeolic contr}
 \item \eintrag{αἰτίας}{ αἴτιος αἰτία }{culpable, responsible}{adj sg fem gen attic doric aeolic OR adj pl fem acc OR noun sg fem gen attic doric aeolic OR noun pl fem acc}
 \item \eintrag{αἰτιωμένη}{ αἰτιάομαι }{accuse, censure}{part sg pres mp fem voc attic epic ionic contr OR part sg pres mp fem nom attic epic ionic contr}
 \item \eintrag{αἰτῶν}{ αἰτέω }{ask, beg}{part sg pres act masc nom attic epic doric contr}
 \item \eintrag{αἰφνίδιον}{ αἰφνίδιος }{unforeseen, sudden}{adj sg neut voc OR adj sg neut nom OR adj sg neut acc OR adj sg masc acc OR adj sg fem acc}
 \item \eintrag{αἰφνιδίῳ}{ αἰφνίδιος }{unforeseen, sudden}{adj sg masc dat OR adj sg neut dat OR adj sg fem dat}
 \item \eintrag{αἰχμάλωτα}{ αἰχμάλωτος }{taken by the spear, captive, prisoner}{adj pl neut acc OR adj pl neut nom OR adj pl neut voc}
 \item \eintrag{αἱ}{ ὅς ὅς ὁ }{yas, yā, yad,}{pron pl fem nom indeclform OR pron pl fem nom indeclform OR article pl fem voc proclitic indeclform OR article pl fem nom proclitic indeclform}
 \item \eintrag{αἴρεσθε}{ αἴρω }{to take up, raise, lift up}{verb 2nd pl pres ind mp OR verb 2nd pl pres imperat mp OR verb 2nd pl imperf ind mp homeric ionic unaugmented}
 \item \eintrag{αἴτιον}{ αἴτιος αἰτέω }{culpable, responsible}{adj sg neut voc OR adj sg neut nom OR adj sg masc acc OR adj sg neut acc OR adj sg fem acc OR verb 3rd pl imperf ind act doric poetic unaugmented OR verb 1st sg imperf ind act doric poetic unaugmented OR part sg pres act neut voc doric OR part sg pres act neut nom doric OR part sg pres act masc voc doric OR part sg pres act neut acc doric}
 \item \eintrag{αὐξανόμενον}{ αὐξάνω }{increase,}{part sg pres mp neut nom OR part sg pres mp neut voc OR part sg pres mp neut acc OR part sg pres mp masc acc}
 \item \eintrag{αὐτίκα}{ αὐτίκα }{forthwith, at once, in a moment,}{adv indeclform}
 \item \eintrag{αὐταῖς}{ αὐτός }{self}{adj pl fem dat}
 \item \eintrag{αὐτοὶ}{ αὐτός }{self}{adj pl masc voc OR adj pl masc nom}
 \item \eintrag{αὐτοὺς}{ αὐτός }{self}{adj pl masc acc}
 \item \eintrag{αὐτοῖς}{ αὐτός }{self}{adj pl neut dat OR adj pl masc dat}
 \item \eintrag{αὐτοῦ}{ αὐτός αὐτοῦ }{self}{adj sg masc gen OR adj sg neut gen OR adv indeclform}
 \item \eintrag{αὐτόμολα}{ αὐτόμολος }{going of oneself, without bidding,}{adj pl neut acc OR adj pl neut nom OR adj pl neut voc}
 \item \eintrag{αὐτόν}{ αὐτός }{self}{adj sg masc acc}
 \item \eintrag{αὐτὰ}{ αὐτός }{self}{adj sg fem voc doric aeolic OR adj sg fem nom doric aeolic OR adj pl neut voc OR adj pl neut nom OR adj pl neut acc OR adj dual fem voc OR adj dual fem nom OR adj dual fem acc}
 \item \eintrag{αὐτὸν}{ αὐτός }{self}{adj sg masc acc}
 \item \eintrag{αὐτὸς}{ αὐτός }{self}{adj sg masc nom}
 \item \eintrag{αὐτῆς}{ αὐτός }{self}{adj sg fem gen attic epic ionic}
 \item \eintrag{αὐτῶν}{ αὐτός }{self}{adj pl neut gen OR adj pl masc gen OR adj pl fem gen}
 \item \eintrag{αὐτῷ}{ αὐτός }{self}{adj sg neut dat OR adj sg masc dat}
 \item \eintrag{αὐχοῦντες}{ αὐχέω }{boast, plume oneself,}{part pl pres act masc voc attic epic doric contr OR part pl pres act masc nom attic epic doric contr}
 \item \eintrag{αὑτόν}{ ἑαυτοῦ }{Stadtrecht von Gortyn}{adj sg masc acc}
 \item \eintrag{αὑτὸν}{ ἑαυτοῦ }{Stadtrecht von Gortyn}{adj sg masc acc}
 \item \eintrag{αὑτῷ}{ ἑαυτοῦ }{Stadtrecht von Gortyn}{adj sg neut dat OR adj sg masc dat}
 \item \eintrag{αὔξετε}{ αὐξάνω }{increase,}{verb 2nd pl pres ind act OR verb 2nd pl pres imperat act OR verb 2nd pl imperf ind act doric}
 \item \eintrag{αὖθις}{ αὖθις }{back, back again,}{adv indeclform}
 \item \eintrag{βάρβαρα}{ βάρβαρος βαρβάρα }{barbarous,}{adj pl neut voc OR adj pl neut nom OR adj pl neut acc OR noun sg fem voc attic doric aeolic OR noun sg fem nom attic doric aeolic OR noun dual fem voc OR noun dual fem nom OR noun dual fem acc}
 \item \eintrag{βίαν}{ Βίας βία βιάω βιάζω }{keine Übersetzung gefunden}{noun sg masc voc OR noun sg fem acc attic doric aeolic OR noun pl fem gen doric aeolic OR verb pres inf act epic doric contr OR verb 3rd pl imperf ind act doric aeolic poetic contr unaugmented OR verb 1st sg imperf ind act attic poetic r e i alpha unaugmented OR part sg pres act neut voc doric aeolic contr OR part sg pres act neut nom doric aeolic contr OR part sg pres act masc voc doric aeolic contr OR part sg pres act neut acc doric aeolic contr OR part sg pres act masc nom doric aeolic contr OR part sg fut act neut voc doric aeolic contr OR verb fut inf act OR part sg fut act neut nom doric aeolic contr OR part sg fut act neut acc doric aeolic contr OR part sg fut act masc voc doric aeolic contr OR part sg fut act masc nom doric aeolic contr}
 \item \eintrag{βαρβάρους}{ βάρβαρος βαρβαρόομαι βαρβαρόω }{barbarous,}{adj pl masc acc OR adj pl fem acc OR verb 2nd sg pres ind act doric contr OR verb 2nd sg imperf ind act homeric ionic contr unaugmented OR verb 2nd sg imperf ind act homeric ionic contr unaugmented OR verb 2nd sg pres ind act doric contr}
 \item \eintrag{βαρβάρων}{ βάρβαρος βαρβάρα βαρβαρόομαι βαρβαρόω }{barbarous,}{adj pl neut gen OR adj pl masc gen OR adj pl fem gen OR noun pl fem gen OR verb pres inf act doric OR verb 3rd pl imperf ind act doric aeolic poetic contr unaugmented OR verb 1st sg imperf ind act doric aeolic poetic contr unaugmented OR part sg pres act neut acc doric aeolic contr OR part sg pres act neut nom doric aeolic contr OR part sg pres act neut voc doric aeolic contr OR part sg pres act masc nom contr OR part sg pres act masc voc doric aeolic contr OR verb pres inf act doric OR verb 3rd pl imperf ind act doric aeolic poetic contr unaugmented OR verb 1st sg imperf ind act doric aeolic poetic contr unaugmented OR part sg pres act neut voc doric aeolic contr OR part sg pres act neut acc doric aeolic contr OR part sg pres act neut nom doric aeolic contr OR part sg pres act masc nom contr OR part sg pres act masc voc doric aeolic contr}
 \item \eintrag{βαρυσυμφορώτατος}{ βαρυσύμφορος }{weighed down by ill-luck,}{adj sg masc nom superl}
 \item \eintrag{βαρὺς}{ βαρύς }{heavy in weight,}{adj sg masc nom}
 \item \eintrag{βασιλέα}{ βασιλεύς }{king, chief,}{noun sg masc acc}
 \item \eintrag{βασιλέας}{ βασιλεύς }{king, chief,}{noun pl masc acc}
 \item \eintrag{βασιλέων}{ βασίλη βασιλεύς }{queen, princess,}{noun pl fem gen epic ionic OR noun pl masc gen contr}
 \item \eintrag{βασιλέως}{ βασιλεύς }{king, chief,}{noun sg masc gen epic doric ionic aeolic OR noun sg masc nom epic ionic}
 \item \eintrag{βασιλεύοντος}{ βασιλεύω }{to be king, rule, reign,}{part sg pres act masc gen OR part sg pres act neut gen}
 \item \eintrag{βασιλεύς}{ βασιλεύς }{king, chief,}{noun sg masc nom OR noun sg masc gen epic ionic contr}
 \item \eintrag{βασιλεὺς}{ βασιλεύς }{king, chief,}{noun sg masc nom OR noun sg masc gen epic ionic contr}
 \item \eintrag{βασιλεῦσι}{ βασιλεύς }{king, chief,}{noun pl masc dat}
 \item \eintrag{βασιλικὰς}{ βασιλικός }{royal, kingly,}{adj pl fem acc OR adj sg fem gen doric aeolic}
 \item \eintrag{βεβαίους}{ βέβαιος βεβαιόω }{firm, steady}{adj pl masc acc OR adj pl fem acc OR verb 2nd sg pres ind act doric contr OR verb 2nd sg imperf ind act homeric ionic contr unaugmented}
 \item \eintrag{βιαίως}{ βίαιος }{forcible, violent}{adv OR adj pl fem acc doric OR adj pl masc acc doric}
 \item \eintrag{βλάβας}{ βλάβη }{harm, damage}{noun sg fem gen doric aeolic OR noun pl fem acc}
 \item \eintrag{βλαβερώτατος}{ βλαβερός }{harmful}{adj sg masc nom superl}
 \item \eintrag{βοήθειαν}{ βοήθεια }{help, aid}{noun sg fem acc OR noun pl fem gen doric aeolic}
 \item \eintrag{βοηθείας}{ βοήθεια }{help, aid}{noun sg fem gen attic doric aeolic OR noun pl fem acc}
 \item \eintrag{βουλευτήριον}{ βουλευτήριον βουλευτήριος }{council-chamber}{noun sg neut voc OR noun sg neut nom OR noun sg neut acc OR adj sg neut nom OR adj sg neut voc OR adj sg neut acc OR adj sg fem acc OR adj sg masc acc}
 \item \eintrag{βουλευτῶν}{ βουλευτής βουλευτός }{councillor, senator}{noun pl masc gen OR adj pl masc gen OR adj pl neut gen OR adj pl fem gen}
 \item \eintrag{βουλομένους}{ βούλομαι }{will, wish, be willing}{part pl pres mp masc acc}
 \item \eintrag{βουλὴ}{ βουλεύς βουλή }{keine Übersetzung gefunden}{noun sg masc acc contr OR noun dual masc nom contr OR noun dual masc voc contr OR noun dual masc acc contr OR noun sg fem nom attic epic ionic OR noun sg fem voc attic epic ionic}
 \item \eintrag{βουλὴν}{ βουλή }{will, determination}{noun sg fem acc attic epic ionic}
 \item \eintrag{βουλῆς}{ βουλεύς βουλή }{keine Übersetzung gefunden}{noun pl masc nom contr OR noun pl masc voc contr OR noun sg fem gen attic epic ionic}
 \item \eintrag{βουλῇ}{ βούλομαι βουλεύς βουλή }{will, wish, be willing}{verb 2nd sg pres subj mp OR verb 2nd sg pres ind mp OR noun sg masc dat epic ionic OR noun sg fem dat attic epic ionic}
 \item \eintrag{βοῆς}{ βόα βόειος βοάω βοείη βοεύς βοή }{fish}{noun sg fem gen attic epic ionic OR adj sg fem gen attic epic ionic contr OR verb 2nd sg pres ind act doric contr OR verb 2nd sg imperf ind act homeric ionic unaugmented OR noun sg fem gen attic epic ionic contr OR noun pl masc nom contr OR noun pl masc voc contr OR noun sg fem gen attic epic doric ionic}
 \item \eintrag{βοῶν}{ βάζω βόα βόειος βοάω βοή βοόω βοῦς βοών }{speak, say,}{part sg fut act neut nom epic OR part sg fut act neut voc epic OR part sg fut act neut acc epic OR part sg fut act masc nom epic OR part sg fut act masc voc epic OR noun pl fem gen OR adj pl neut gen attic epic ionic contr OR adj pl masc gen attic epic ionic contr OR adj pl fem gen attic epic ionic contr OR verb 3rd pl imperf ind act homeric ionic contr unaugmented OR part sg pres act masc nom attic epic doric ionic contr OR part sg pres act masc voc contr OR part sg pres act neut acc contr OR part sg pres act neut nom contr OR part sg pres act neut voc contr OR verb 1st sg imperf ind act homeric ionic contr unaugmented OR noun pl fem gen doric OR part sg pres act neut voc doric aeolic contr OR verb pres inf act doric OR part sg pres act neut acc doric aeolic contr OR part sg pres act neut nom doric aeolic contr OR part sg pres act masc nom contr OR part sg pres act masc voc doric aeolic contr OR noun pl masc gen indeclform OR noun pl fem gen indeclform OR noun sg masc voc OR noun sg masc nom}
 \item \eintrag{βραχέος}{ βραχύς }{short}{adj sg neut gen epic doric ionic aeolic OR adj sg masc gen epic doric ionic aeolic}
 \item \eintrag{βραχεῖ}{ βραχύς }{short}{adj sg neut dat OR adj sg masc dat}
 \item \eintrag{γένοιτο}{ γίγνομαι }{come into a new state of being}{verb 3rd sg aor opt mid}
 \item \eintrag{γε}{ γε }{at least, at any rate}{partic enclitic indeclform}
 \item \eintrag{γείτονας}{ γείτων }{neighbour, borderer}{noun pl fem acc OR noun pl masc acc}
 \item \eintrag{γείτονι}{ γείτων }{neighbour, borderer}{noun sg fem dat OR noun sg masc dat}
 \item \eintrag{γεγονέναι}{ γίγνομαι }{come into a new state of being}{verb perf inf act}
 \item \eintrag{γεγονότος}{ γίγνομαι }{come into a new state of being}{part sg perf act neut gen OR part sg perf act masc gen}
 \item \eintrag{γεγονότων}{ γίγνομαι }{come into a new state of being}{part pl perf act neut gen OR part pl perf act masc gen}
 \item \eintrag{γεγονώς}{ γίγνομαι }{come into a new state of being}{part sg perf act masc voc OR part sg perf act masc nom}
 \item \eintrag{γεγυμνασμένην}{ γυμνάζω }{train naked, train in gymnastic exercise}{part sg perf mp fem acc attic epic ionic redupl}
 \item \eintrag{γειτνίασιν}{ γειτνίασις γειτνιάω γειτνιάζω }{neighbourhood, proximity}{noun sg fem acc OR verb 3rd sg pres subj act epic contr nu movable OR verb 3rd pl pres subj act doric aeolic contr nu movable OR part pl pres act neut dat doric nu movable OR verb 2nd sg pres subj mp epic contr nu movable OR part pl pres act masc dat doric nu movable OR part pl fut act neut dat doric nu movable OR part pl fut act masc dat doric nu movable}
 \item \eintrag{γειτνίασις}{ γειτνίασις }{neighbourhood, proximity}{noun sg fem nom OR noun pl fem acc epic doric ionic aeolic}
 \item \eintrag{γειτονεύοντα}{ γειτονέω γειτονεύω }{keine Übersetzung gefunden}{part sg pres act masc acc OR part pl pres act neut nom OR part pl pres act neut voc OR part pl pres act neut acc OR part sg pres act masc acc OR part pl pres act neut nom OR part pl pres act neut voc OR part pl pres act neut acc}
 \item \eintrag{γειτόνων}{ γείτων γειτονέω }{neighbour, borderer}{noun pl masc gen OR noun pl fem gen OR part sg pres act masc nom attic epic doric contr}
 \item \eintrag{γενέσθαι}{ γίγνομαι }{come into a new state of being}{verb aor inf mid}
 \item \eintrag{γενήσεσθαι}{ γίγνομαι }{come into a new state of being}{verb fut inf mid}
 \item \eintrag{γενομένας}{ γίγνομαι }{come into a new state of being}{part pl aor mid fem acc OR part sg aor mid fem gen doric aeolic}
 \item \eintrag{γενομένην}{ γίγνομαι }{come into a new state of being}{part sg aor mid fem acc attic epic ionic}
 \item \eintrag{γενομένης}{ γίγνομαι }{come into a new state of being}{part sg aor mid fem gen attic epic ionic}
 \item \eintrag{γενομένοις}{ γίγνομαι }{come into a new state of being}{part pl aor mid neut dat OR part pl aor mid masc dat}
 \item \eintrag{γενομένου}{ γίγνομαι }{come into a new state of being}{part sg aor mid masc gen OR part sg aor mid neut gen}
 \item \eintrag{γενομένων}{ γίγνομαι }{come into a new state of being}{part pl aor mid neut gen OR part pl aor mid masc gen OR part pl aor mid fem gen}
 \item \eintrag{γενομένῳ}{ γίγνομαι }{come into a new state of being}{part sg aor mid neut dat OR part sg aor mid masc dat}
 \item \eintrag{γενόμεναι}{ γίγνομαι }{come into a new state of being}{part sg aor mid fem dat doric aeolic OR part pl aor mid fem voc OR part pl aor mid fem nom}
 \item \eintrag{γενόμενον}{ γίγνομαι }{come into a new state of being}{part sg aor mid neut acc OR part sg aor mid neut nom OR part sg aor mid neut voc OR part sg aor mid masc acc}
 \item \eintrag{γενόμενος}{ γίγνομαι }{come into a new state of being}{part sg aor mid masc nom}
 \item \eintrag{γιγνομένους}{ γίγνομαι }{come into a new state of being}{part pl pres mp masc acc pres redupl}
 \item \eintrag{γιγνόμενον}{ γίγνομαι }{come into a new state of being}{part sg pres mp neut voc pres redupl OR part sg pres mp neut nom pres redupl OR part sg pres mp masc acc pres redupl OR part sg pres mp neut acc pres redupl}
 \item \eintrag{γνωρίζοντος}{ γνωρίζω }{make known, point out}{part sg pres act masc gen OR part sg pres act neut gen}
 \item \eintrag{γνώμαις}{ γνώμη }{means of knowing}{noun pl fem dat}
 \item \eintrag{γνώμην}{ γνώμη }{means of knowing}{noun sg fem acc attic epic ionic}
 \item \eintrag{γυναικῶν}{ γυναικόω γυναικών γυναικωνῖτις γυνή }{make effeminate,}{verb pres inf act doric OR verb 3rd pl imperf ind act doric aeolic poetic contr unaugmented OR verb 1st sg imperf ind act doric aeolic poetic contr unaugmented OR part sg pres act neut voc doric aeolic contr OR part sg pres act neut nom doric aeolic contr OR part sg pres act neut acc doric aeolic contr OR part sg pres act masc nom contr OR part sg pres act masc voc doric aeolic contr OR noun sg masc voc OR noun sg masc nom OR noun sg masc voc OR noun sg masc nom OR noun pl fem gen indeclform}
 \item \eintrag{γὰρ}{ γάρ }{for}{partic indeclform}
 \item \eintrag{γῆς}{ γῆ }{earth}{noun sg fem gen attic epic ionic contr}
 \item \eintrag{δʼ}{ }{keine Übersetzung gefunden}{Nichts gefunden}
 \item \eintrag{δέ}{ δέ }{but}{partic indeclform}
 \item \eintrag{δέκα}{ δέκα δεκάς δεκάω δεκάζω }{ten,}{numeral indeclform OR noun sg fem voc OR verb 3rd sg imperf ind act homeric ionic contr unaugmented OR verb 1st sg pres subj act doric aeolic contr OR verb 2nd sg pres imperat act contr OR verb 1st sg pres ind act doric aeolic contr OR verb 1st sg fut ind act doric aeolic contr}
 \item \eintrag{δέον}{ δέον δέω δέω2 δεῖ }{that which is binding, needful, right,}{noun sg neut voc OR verb 3rd pl imperf ind act homeric ionic unaugmented OR part sg pres act neut voc attic epic doric ionic aeolic OR verb 1st sg imperf ind act homeric ionic unaugmented OR part sg pres act neut nom attic epic doric ionic aeolic OR part sg pres act masc voc attic epic doric ionic aeolic OR part sg pres act neut acc attic epic doric ionic aeolic OR part sg pres act neut nom OR part sg pres act neut voc OR verb 1st sg imperf ind act homeric ionic unaugmented OR verb 3rd pl imperf ind act homeric ionic unaugmented OR part sg pres act neut acc OR part sg pres act masc voc OR part sg pres act neut voc impersonal OR verb 1st sg imperf ind act homeric ionic unaugmented impersonal OR verb 3rd pl imperf ind act epic doric ionic aeolic unaugmented impersonal OR part sg pres act masc voc impersonal OR part sg pres act neut acc impersonal OR part sg pres act neut nom impersonal}
 \item \eintrag{δέους}{ δέος δέος }{fear, alarm, affright}{noun sg neut gen attic epic doric contr OR noun sg neut gen attic epic doric contr}
 \item \eintrag{δήμῳ}{ δῆμος δημός }{a country-district, country, land}{noun sg masc dat OR noun sg masc dat}
 \item \eintrag{δίδωσι}{ δίδημι δίδωμι }{bind, fetter,}{verb 3rd pl pres subj act pres redupl OR verb 3rd sg pres ind act pres redupl OR verb 2nd sg pres subj mp epic contr pres redupl OR verb 3rd pl pres subj act contr pres redupl}
 \item \eintrag{δίκαιον}{ δίκαιος }{observant of custom}{adj sg neut voc OR adj sg neut acc OR adj sg neut nom OR adj sg masc acc OR adj sg fem acc}
 \item \eintrag{δίκῃ}{ δίκη δικάζω δικεῖν }{custom, usage,}{noun sg fem dat attic epic ionic OR verb 3rd sg fut ind act doric contr OR verb 2nd sg fut ind mid doric contr OR verb 2nd sg aor subj mp poetic OR verb 3rd sg aor subj act poetic}
 \item \eintrag{δίς}{ δίς }{twice, doubly,}{adv indeclform}
 \item \eintrag{δαίμων}{ δαίμων }{god, goddess,}{noun sg fem nom OR noun sg fem voc OR noun sg masc nom OR noun sg masc voc}
 \item \eintrag{δαιμόνιον}{ δαιμόνιον δαιμόνιος δαιμονάω }{divine Power, Divinity,}{noun sg neut voc OR noun sg neut nom OR noun sg neut acc OR adj sg neut voc OR adj sg neut nom OR adj sg neut acc OR adj sg masc acc OR adj sg fem acc OR verb 3rd pl imperf ind act epic doric ionic poetic unaugmented OR verb 1st sg imperf ind act epic doric ionic poetic unaugmented OR part sg pres act neut nom epic doric ionic OR part sg pres act neut voc epic doric ionic OR part sg pres act neut acc epic doric ionic OR part sg pres act masc voc epic doric ionic}
 \item \eintrag{δαπάνην}{ δαπάνη δαπανάω }{cost, expenditure,}{noun sg fem acc attic epic ionic OR verb pres inf act doric ionic contr OR verb 1st sg imperf ind act homeric ionic unaugmented OR verb 3rd pl imperf ind act epic doric aeolic unaugmented}
 \item \eintrag{δεδιὼς}{ δείδω }{to fear}{part sg perf act masc voc OR part sg perf act masc nom}
 \item \eintrag{δειλίαν}{ δειλία δειλιάω }{timidity, cowardice,}{noun sg fem acc attic doric aeolic OR noun pl fem gen doric aeolic OR verb pres inf act epic doric contr OR verb 3rd pl imperf ind act doric aeolic poetic contr unaugmented OR verb 1st sg imperf ind act doric aeolic poetic contr unaugmented OR part sg pres act neut nom doric aeolic contr OR part sg pres act neut voc doric aeolic contr OR part sg pres act masc nom doric aeolic contr OR part sg pres act masc voc doric aeolic contr OR part sg pres act neut acc doric aeolic contr}
 \item \eintrag{δεινὸν}{ δεῖνος δεῖνος δεινός }{keine Übersetzung gefunden}{noun sg masc acc OR noun sg masc acc OR adj sg neut voc OR adj sg neut nom OR adj sg neut acc OR adj sg masc acc}
 \item \eintrag{δεξιὸς}{ δέξις δεξιός }{reception,}{noun sg fem gen epic doric ionic aeolic OR adj sg masc nom}
 \item \eintrag{δεομένοις}{ δέομαι δέω δέω2 }{lack}{part pl pres mp neut dat epic doric ionic aeolic OR part pl pres mp masc dat epic doric ionic aeolic OR part pl pres mp neut dat OR part pl pres mp masc dat OR part pl pres mp neut dat OR part pl pres mp masc dat OR part pl pres mid neut dat epic doric ionic aeolic OR part pl pres mid masc dat epic doric ionic aeolic}
 \item \eintrag{δεύτερος}{ δεύτερος }{second,}{adj sg masc nom}
 \item \eintrag{δεῖσαι}{ δείδω δεῖσα }{to fear}{verb 2nd sg aor imperat mid OR verb 3rd sg aor opt act OR verb aor inf act OR noun sg fem dat doric aeolic OR noun pl fem nom OR noun pl fem voc}
 \item \eintrag{δηλώσων}{ δηλόω }{to make visible}{part sg fut act masc nom doric contr}
 \item \eintrag{διʼ}{ }{keine Übersetzung gefunden}{Nichts gefunden}
 \item \eintrag{διά}{ Δίη Δίον Δῖα Ζεύς δῖος διά }{keine Übersetzung gefunden}{noun sg fem voc attic doric aeolic OR noun sg fem nom attic doric aeolic OR noun pl neut voc OR noun pl neut nom OR noun pl neut acc OR noun pl neut voc OR noun pl neut nom OR noun pl neut acc OR noun sg masc acc indeclform OR adj sg fem voc epic indeclform OR adj sg fem nom epic indeclform OR adj pl neut voc OR adj pl neut nom OR adj pl neut acc OR adj dual fem voc OR adj dual fem nom OR adj dual fem acc OR prep indeclform}
 \item \eintrag{διάλυσις}{ διάλυσις }{separating, parting,}{noun pl fem acc epic doric ionic aeolic OR noun sg fem nom}
 \item \eintrag{διέβαλλε}{ διαβάλλω }{throw}{verb 3rd sg imperf ind act}
 \item \eintrag{διέτριβον}{ διατρίβω }{rub hard,}{verb 1st sg imperf ind act OR verb 3rd pl imperf ind act}
 \item \eintrag{διέφθειραν}{ διαφθείρω }{destroy utterly,}{verb 3rd pl aor ind act}
 \item \eintrag{διέφθειρεν}{ διαφθείρω }{destroy utterly,}{verb 3rd sg aor ind act nu movable OR verb 3rd sg imperf ind act nu movable}
 \item \eintrag{διαίτης}{ δίαιτα διαιτάω διαιτέω }{way of living, mode of life,}{noun sg fem gen attic epic ionic OR verb 2nd sg pres ind act epic doric ionic contr OR verb 2nd sg imperf ind act homeric ionic unaugmented OR verb 2nd sg pres ind act doric contr}
 \item \eintrag{διαβάλλοντος}{ διαβάλλω }{throw}{part sg pres act neut gen OR part sg pres act masc gen}
 \item \eintrag{διαβαλεῖν}{ διαβάλλω }{throw}{verb fut inf act attic epic doric contr OR verb aor inf act attic epic doric contr}
 \item \eintrag{διαβαλλόντων}{ διαβάλλω }{throw}{verb 3rd pl pres imperat act OR part pl pres act neut gen OR part pl pres act masc gen}
 \item \eintrag{διαβολὴν}{ διαβολή }{false accusation, slander,}{noun sg fem acc attic epic ionic}
 \item \eintrag{διαβολῆς}{ διαβολή }{false accusation, slander,}{noun sg fem gen attic epic ionic}
 \item \eintrag{διακριθῆναι}{ διακρίνω }{separate one from another,}{verb aor inf pass}
 \item \eintrag{διακόπτων}{ διακόπτω }{cut in two, cut through,}{part sg pres act masc nom}
 \item \eintrag{διακόσια}{ διακόσιοι }{two hundred,}{adj sg fem voc attic doric aeolic OR adj sg fem nom attic doric aeolic OR adj pl neut nom OR adj pl neut voc OR adj dual fem acc OR adj dual fem nom OR adj dual fem voc OR adj pl neut acc}
 \item \eintrag{διαλλαγαῖς}{ διαλλαγή }{interchange,}{noun pl fem dat}
 \item \eintrag{διαλλαγῇ}{ διαλλάσσω διαλλαγή }{interchange,}{verb 3rd sg aor subj pass contr OR noun sg fem dat attic epic ionic}
 \item \eintrag{διαλλαγῶν}{ διαλλαγή }{interchange,}{noun pl fem gen}
 \item \eintrag{διαλλακτήρων}{ διαλλακτήρ }{mediator,}{noun pl masc gen}
 \item \eintrag{διαλύσεις}{ διάλυσις διαλύω }{separating, parting,}{noun pl fem voc attic epic contr OR noun pl fem nom attic epic contr OR noun pl fem acc attic OR verb 2nd sg fut ind act doric contr OR verb 2nd sg aor subj act epic short subj}
 \item \eintrag{διαλύσεως}{ διάλυσις }{separating, parting,}{noun sg fem gen attic}
 \item \eintrag{διαλῦσαι}{ διαλύω }{loose one from another, part asunder,}{verb aor inf act OR verb 2nd sg aor imperat mid OR verb 3rd sg aor opt act}
 \item \eintrag{διαπλεῦσαι}{ διαπλέω διαπλόω }{sail through}{verb 2nd sg aor imperat mid OR verb 3rd sg aor opt act OR verb aor inf act OR part pl pres act fem voc epic doric ionic contr OR part sg pres act fem dat doric contr OR part pl pres act fem nom epic doric ionic contr OR part pl pres act fem nom epic ionic contr OR part pl pres act fem voc epic ionic contr}
 \item \eintrag{διαστήματι}{ διάστημα }{interval,}{noun sg neut dat}
 \item \eintrag{διαστασιάσειε}{ διαστασιάζω }{form into separate factions,}{verb 3rd sg aor opt act}
 \item \eintrag{διασωθέντος}{ διασῴζω }{preserve through}{part sg aor pass neut gen OR part sg aor pass masc gen}
 \item \eintrag{διαφερόμενοι}{ διαφέρω }{carry over}{part pl pres mp masc voc OR part pl pres mp masc nom}
 \item \eintrag{διδούς}{ δίδωμι }{Aër.}{verb 2nd sg pres ind act epic doric contr pres redupl OR part sg pres act masc voc pres redupl OR verb 2nd sg imperf ind act homeric ionic unaugmented pres redupl OR part sg pres act masc nom pres redupl}
 \item \eintrag{διελέγχειν}{ διελέγχω }{refute,}{verb pres inf act attic epic contr}
 \item \eintrag{διεπρεσβεύετο}{ διαπρεσβεύομαι }{send embassies,}{verb 3rd sg imperf ind mp}
 \item \eintrag{διεστῶτες}{ διίστημι }{to set apart, to place separately, separate}{part pl perf act masc nom OR part pl perf act masc voc}
 \item \eintrag{διετάρασσεν}{ διαταράσσω }{throw into confusion,}{verb 3rd sg imperf ind act nu movable}
 \item \eintrag{διετίθετο}{ διατίθημι }{arrange}{verb 3rd sg imperf ind mp}
 \item \eintrag{δικαίους}{ δίκαιος δικαιόω }{observant of custom}{adj pl fem acc OR adj pl masc acc OR verb 2nd sg imperf ind act homeric ionic contr unaugmented OR verb 2nd sg pres ind act doric contr}
 \item \eintrag{δισμυρίους}{ δισμύριοι }{twenty thousand,}{adj pl masc acc}
 \item \eintrag{διὰ}{ Δίη Δίον Δῖα Ζεύς δῖος διά }{keine Übersetzung gefunden}{noun sg fem voc attic doric aeolic OR noun sg fem nom attic doric aeolic OR noun pl neut voc OR noun pl neut nom OR noun pl neut acc OR noun pl neut voc OR noun pl neut nom OR noun pl neut acc OR noun sg masc acc indeclform OR adj sg fem voc epic indeclform OR adj sg fem nom epic indeclform OR adj pl neut voc OR adj pl neut nom OR adj pl neut acc OR adj dual fem voc OR adj dual fem nom OR adj dual fem acc OR prep indeclform}
 \item \eintrag{δμηθεὶς}{ δαμάζω }{overpower}{part sg aor pass masc voc OR part sg aor pass masc nom}
 \item \eintrag{δοθῆναι}{ δίδωμι }{Aër.}{verb aor inf pass}
 \item \eintrag{δοκιμάζω}{ δοκιμάζω }{assay, test,}{verb 1st sg pres ind act OR verb 1st sg pres subj act}
 \item \eintrag{δουλείαν}{ δούλειος δουλεία }{slavish, servile,}{adj pl masc gen doric OR adj sg fem acc attic doric aeolic OR adj pl fem gen doric OR noun sg fem acc attic doric aeolic OR noun pl fem gen doric aeolic}
 \item \eintrag{δοῦναι}{ δίδωμι }{Aër.}{verb aor inf act}
 \item \eintrag{δρόμῳ}{ δρόμος }{course, race,}{noun sg masc dat}
 \item \eintrag{δυνάμενος}{ δύναμαι }{to be able, strong enough}{part sg pres mp masc nom}
 \item \eintrag{δυνάστην}{ δυνάστης }{lord, master, ruler}{noun sg masc acc attic epic ionic}
 \item \eintrag{δυναμένων}{ δύναμαι }{to be able, strong enough}{part pl pres mp neut gen OR part pl pres mp masc gen OR part pl pres mp fem gen}
 \item \eintrag{δυσμενὴς}{ δυσμενής }{hostile}{adj sg masc nom OR adj sg fem nom OR adj pl masc nom doric aeolic contr OR adj pl masc voc doric aeolic contr OR adj pl masc acc attic epic doric contr OR adj pl fem acc attic epic doric contr OR adj pl fem nom doric aeolic contr OR adj pl fem voc doric aeolic contr}
 \item \eintrag{δυσπόρους}{ δύσπορος }{scarcely passable}{adj pl masc acc OR adj pl fem acc}
 \item \eintrag{δυσχεραίνοντες}{ δυσχεραίνω }{to be unable to endure}{part pl pres act masc voc OR part pl pres act masc nom}
 \item \eintrag{δυσχεραινόντων}{ δυσχεραίνω }{to be unable to endure}{verb 3rd pl pres imperat act OR part pl pres act masc gen OR part pl pres act neut gen}
 \item \eintrag{δωρεὰν}{ δωρεά }{gift, present}{noun sg fem acc attic doric ionic aeolic OR noun pl fem gen doric ionic aeolic OR adv indeclform}
 \item \eintrag{δωροδοκίαν}{ δωροδοκία }{taking of bribes}{noun sg fem acc attic doric aeolic OR noun pl fem gen doric aeolic}
 \item \eintrag{δόξαν}{ δόξα δοξάζω δοκέω }{expectation,}{noun pl fem gen doric aeolic OR noun sg fem acc doric aeolic OR part sg fut act neut voc doric aeolic contr OR verb fut inf act OR part sg fut act neut acc doric aeolic contr OR part sg fut act neut nom doric aeolic contr OR part sg fut act masc voc doric aeolic contr OR part sg fut act masc nom doric aeolic contr OR part sg aor act neut voc OR verb 3rd pl aor ind act homeric ionic unaugmented OR part sg aor act neut acc OR part sg aor act neut nom}
 \item \eintrag{δύναιντο}{ δύναμαι }{to be able, strong enough}{verb 3rd pl pres opt mp}
 \item \eintrag{δύναμίν}{ δύναμις }{power, might}{noun sg fem acc}
 \item \eintrag{δύνασθαι}{ δύναμαι }{to be able, strong enough}{verb pres inf mp}
 \item \eintrag{δύο}{ δύο }{Acut.(Sp.}{numeral indeclform}
 \item \eintrag{δώσειν}{ δίδωμι }{Aër.}{verb fut inf act doric contr}
 \item \eintrag{δὲ}{ δέ }{but}{partic indeclform}
 \item \eintrag{δὴ}{ Δεύς δέω δέω2 δεῖ δή }{keine Übersetzung gefunden}{noun dual masc voc contr OR noun sg masc acc contr OR noun dual masc nom contr OR noun dual masc acc contr OR verb 3rd sg imperf ind act doric aeolic poetic contr unaugmented OR verb 2nd sg pres imperat act doric aeolic contr OR verb 3rd sg imperf ind act doric aeolic poetic contr unaugmented OR verb 2nd sg pres imperat act doric aeolic contr OR verb 3rd sg imperf ind act doric ionic aeolic poetic contr unaugmented impersonal OR partic indeclform}
 \item \eintrag{δῆμος}{ δῆμος δημός }{a country-district, country, land}{noun sg masc nom OR noun sg masc nom}
 \item \eintrag{εἰ}{ εἰ εἰ2 εἶμι εἶμι2 εἰμί }{if}{conj proclitic indeclform OR conj proclitic indeclform OR verb 2nd sg pres ind act OR verb 2nd sg pres ind act OR verb 2nd sg pres ind act}
 \item \eintrag{εἰάσατε}{ ἐάω εἰάζω }{suffer, permit}{verb 2nd pl aor ind act attic epic doric aeolic syll augment OR verb 2nd pl aor ind act homeric ionic unaugmented OR verb 2nd pl aor imperat act}
 \item \eintrag{εἰδὼς}{ εἰδοί οἶδα }{Idus}{noun pl masc acc doric OR part sg perf act masc voc OR part sg perf act masc nom}
 \item \eintrag{εἰκὸς}{ ἔοικα εἰκός }{as,}{part sg perf act neut nom OR part sg perf act neut voc OR part sg perf act neut acc OR part sg perf act neut nom OR part sg perf act neut voc OR part sg perf act neut acc}
 \item \eintrag{εἰλημμένα}{ λαμβάνω }{a}{part sg perf mp fem voc doric aeolic OR part pl perf mp neut voc OR part sg perf mp fem nom doric aeolic OR part pl perf mp neut nom OR part pl perf mp neut acc OR part dual perf mp fem voc OR part dual perf mp fem nom OR part dual perf mp fem acc}
 \item \eintrag{εἰμι}{ εἶμι εἶμι2 εἰμί }{to go}{verb 1st sg pres ind act OR verb 1st sg pres ind act OR verb 1st sg pres ind act enclitic}
 \item \eintrag{εἰπεῖν}{ εἶπον }{said}{verb aor inf act attic epic doric contr}
 \item \eintrag{εἰπόντος}{ εἶπον }{said}{part sg aor act neut gen OR part sg aor act masc gen}
 \item \eintrag{εἰπών}{ εἶπον εἶπος }{said}{part sg aor act masc nom OR noun pl masc gen}
 \item \eintrag{εἰπὼν}{ εἶπον εἶπος }{said}{part sg aor act masc nom OR noun pl masc gen}
 \item \eintrag{εἰρήνην}{ εἰρήνη εἰρηνέω }{peace}{noun sg fem acc attic epic ionic OR verb pres inf act epic doric contr}
 \item \eintrag{εἰρήνης}{ εἰρήνη εἰρηνέω }{peace}{noun sg fem gen attic epic ionic OR verb 2nd sg pres ind act doric contr OR verb 2nd sg imperf ind act doric aeolic poetic contr unaugmented}
 \item \eintrag{εἰρήνῃ}{ εἰρήνη εἰρηνέω }{peace}{noun sg fem dat attic epic ionic OR verb 3rd sg pres subj act contr OR verb 2nd sg pres subj mp contr OR verb 2nd sg pres ind mp contr}
 \item \eintrag{εἰς}{ εἶμι εἶμι2 εἰμί εἰς }{to go}{verb 2nd sg pres ind act epic ionic OR verb 2nd sg pres ind act epic ionic OR verb 2nd sg pres ind act ionic OR prep proclitic indeclform}
 \item \eintrag{εἰώθεσαν}{ ἔθω }{to be accustomed, to be wont}{verb 3rd pl plup ind act}
 \item \eintrag{εἴ}{ εἰ εἰ2 εἶμι εἶμι2 εἰμί }{if}{conj proclitic indeclform OR conj proclitic indeclform OR verb 2nd sg pres ind act OR verb 2nd sg pres ind act OR verb 2nd sg pres ind act}
 \item \eintrag{εἴθʼ}{ }{keine Übersetzung gefunden}{Nichts gefunden}
 \item \eintrag{εἴκαζον}{ εἰκάζω }{represent by an image}{verb 3rd pl imperf ind act homeric ionic unaugmented OR verb 1st sg imperf ind act homeric ionic unaugmented OR part sg pres act neut voc OR part sg pres act neut nom OR part sg pres act neut acc OR part sg pres act masc voc}
 \item \eintrag{εἴληφεν}{ λαμβάνω }{a}{verb perf inf act epic poetic OR verb 3rd pl plup ind act epic doric aeolic OR verb 3rd sg perf ind act nu movable}
 \item \eintrag{εἴποι}{ εἶπον εἶπος }{said}{verb 3rd sg aor opt act OR noun pl masc nom OR noun pl masc voc}
 \item \eintrag{εἴτʼ}{ }{keine Übersetzung gefunden}{Nichts gefunden}
 \item \eintrag{εἴτε}{ εἴτε εἰμί }{sive..sive.., either..or.., whether..or..,}{adv indeclform OR verb 2nd pl pres opt act}
 \item \eintrag{εἶναι}{ εἰμί }{sum}{verb pres inf act}
 \item \eintrag{εἶπεν}{ εἶπον }{said}{verb aor inf act doric OR verb 3rd sg aor ind act epic ionic nu movable}
 \item \eintrag{εἶπον}{ εἶπον εἶπος }{said}{verb 2nd sg aor imperat act late OR verb 3rd pl aor ind act OR part sg aor act neut voc OR verb 1st sg aor ind act homeric ionic syll augment unaugmented OR part sg aor act neut nom OR part sg aor act neut acc OR part sg aor act masc voc OR noun sg masc acc}
 \item \eintrag{εἶχον}{ ἔχω ἔχω2 }{check}{verb 3rd pl imperf ind act syll augment OR verb 1st sg imperf ind act syll augment OR verb 3rd pl imperf ind act syll augment OR verb 1st sg imperf ind act syll augment}
 \item \eintrag{εἷλε}{ αἱρέω }{take with the hand, grasp, seize}{verb 3rd sg aor ind act syll augment late}
 \item \eintrag{εὐβουλίαν}{ εὐβουλία }{good counsel, soundness of judgement, prudence}{noun pl fem gen doric aeolic OR noun sg fem acc attic doric aeolic}
 \item \eintrag{εὐδοκιμοῦντα}{ εὐδοκιμέω }{to be of good repute, highly esteemed, popular}{part sg pres act masc acc attic epic doric contr OR part pl pres act neut voc attic epic doric contr OR part pl pres act neut acc attic epic doric contr OR part pl pres act neut nom attic epic doric contr}
 \item \eintrag{εὐθὺς}{ εὐθύς εὐθύς }{straight, direct}{adj sg masc nom OR adv indeclform}
 \item \eintrag{εὐμαρῆ}{ εὐμαρέω εὐμαρής }{have abundance}{verb 3rd sg imperf ind act doric aeolic poetic contr unaugmented OR verb 2nd sg pres imperat act doric aeolic contr OR adj sg fem acc attic epic doric contr OR adj sg masc acc attic epic doric contr OR adj pl neut voc attic epic doric contr OR adj pl neut nom attic epic doric contr OR adj pl neut acc attic epic doric contr OR adj dual neut voc doric aeolic contr OR adj dual neut nom doric aeolic contr OR adj dual neut acc doric aeolic contr OR adj dual masc voc doric aeolic contr OR adj dual masc nom doric aeolic contr OR adj dual masc acc doric aeolic contr OR adj dual fem voc doric aeolic contr OR adj dual fem acc doric aeolic contr OR adj dual fem nom doric aeolic contr}
 \item \eintrag{εὐμενεστέρους}{ εὐμενής }{well-disposed, favourable, gracious, kindly}{adj pl masc acc comp}
 \item \eintrag{εὐμετάβολος}{ εὐμετάβολος }{changeable}{adj sg fem nom OR adj sg masc nom}
 \item \eintrag{εὐμετάθετος}{ εὐμετάθετος }{easily changing}{adj sg fem nom OR adj sg masc nom}
 \item \eintrag{εὐνοίας}{ εὔνοια }{goodwill, favour}{noun sg fem gen attic doric aeolic OR noun pl fem acc}
 \item \eintrag{εὐσχήμονα}{ εὐσχήμων }{elegant in figure, mien and bearing, graceful}{adj sg masc acc OR adj pl neut acc OR adj pl neut nom OR adj pl neut voc OR adj sg fem acc}
 \item \eintrag{εὐτολμότατος}{ εὔτολμος }{brave-spirited, courageous}{adj sg masc nom superl}
 \item \eintrag{εὐτυχίας}{ εὐτυχία }{good luck, success}{noun sg fem gen attic doric aeolic OR noun pl fem acc}
 \item \eintrag{εὐχεροῦς}{ εὐχερής }{tolerant of}{adj sg neut gen attic epic doric contr OR adj sg masc gen attic epic doric contr OR adj sg fem gen attic epic doric contr}
 \item \eintrag{εὐχερὴς}{ εὐχερής }{tolerant of}{adj sg masc nom OR adj sg fem nom OR adj pl masc voc doric aeolic contr OR adj pl masc nom doric aeolic contr OR adj pl masc acc attic epic doric contr OR adj pl fem voc doric aeolic contr OR adj pl fem nom doric aeolic contr OR adj pl fem acc attic epic doric contr}
 \item \eintrag{εὑρεῖν}{ εὑρίσκω }{find}{verb aor inf act attic epic doric contr}
 \item \eintrag{εὔβουλόν}{ εὔβουλος }{well-advised, prudent}{adj sg neut nom OR adj sg neut voc OR adj sg neut acc OR adj sg masc acc OR adj sg fem acc}
 \item \eintrag{εὔζωνοι}{ εὔζωνος }{well-girdled}{adj pl masc voc OR adj pl fem nom OR adj pl fem voc OR adj pl masc nom}
 \item \eintrag{εὔζωνον}{ εὔζωνος }{well-girdled}{adj sg neut nom OR adj sg neut voc OR adj sg neut acc OR adj sg fem acc OR adj sg masc acc}
 \item \eintrag{εὖ}{ εὖ }{well}{adv indeclform}
 \item \eintrag{ζήλου}{ ζῆλος ζηλέω ζηλόω }{jealousy}{noun sg masc gen OR verb 2nd sg pres imperat mp attic contr OR verb 3rd sg imperf ind act homeric ionic contr unaugmented OR verb 2nd sg pres imperat mp contr OR verb 2nd sg pres imperat act contr OR verb 2nd sg imperf ind mp homeric ionic contr unaugmented}
 \item \eintrag{ζευγνὺς}{ ζεύγνυμι }{yoke, put to}{part sg pres act masc nom OR part sg pres act masc voc OR verb 2nd sg imperf ind act homeric ionic unaugmented OR verb 2nd sg pres ind act}
 \item \eintrag{ζητεῖ}{ ζητέω }{seek, seek for}{verb 2nd sg pres imperat act attic epic contr OR verb 2nd sg pres ind mp attic epic doric ionic contr OR verb 3rd sg imperf ind act attic epic contr unaugmented OR verb 3rd sg pres ind act attic epic doric ionic contr}
 \item \eintrag{θάλασσαν}{ θάλασσα }{sea}{noun sg fem acc OR noun pl fem gen doric aeolic}
 \item \eintrag{θέρους}{ θέρος }{summer}{noun sg neut gen attic epic doric ionic contr}
 \item \eintrag{θέσιν}{ θέσις }{setting, placing}{noun sg fem acc}
 \item \eintrag{θήσει}{ τίθημι }{p}{verb 2nd sg fut ind mid doric contr OR verb 3rd sg fut ind act doric contr}
 \item \eintrag{θαλάσσης}{ θάλασσα θαλασσεύς }{sea}{noun sg fem gen attic epic ionic OR noun pl masc voc contr OR noun pl masc nom contr}
 \item \eintrag{θαμινὰ}{ θαμινός }{crowded, close-set}{adj dual fem acc OR adj dual fem nom OR adj dual fem voc OR adj pl neut acc OR adj pl neut nom OR adj pl neut voc OR adj sg fem nom doric aeolic OR adj sg fem voc doric aeolic}
 \item \eintrag{θαρρεῖν}{ θαρσέω }{to be of good courage}{verb pres inf act attic epic doric contr}
 \item \eintrag{θεοβλαβείας}{ θεοβλάβεια }{infatuation sent by the gods, madness}{noun pl fem acc OR noun sg fem gen attic doric aeolic}
 \item \eintrag{θεοὺς}{ θεός }{God, the Deity}{noun pl masc acc}
 \item \eintrag{θεοῦ}{ θεάομαι θεάω θεός }{gaze at, behold}{verb 2nd sg pres imperat mp attic epic ionic contr OR verb 2nd sg imperf ind mp attic epic ionic poetic contr unaugmented OR verb 2nd sg pres imperat mp attic epic ionic contr OR verb 2nd sg imperf ind mp attic epic ionic poetic contr unaugmented OR noun sg masc gen}
 \item \eintrag{θεραπεύοι}{ θεραπεύω }{to be an attendant, do service}{verb 3rd sg pres opt act}
 \item \eintrag{θεωρίαν}{ θεωρία }{sending of}{noun sg fem acc attic doric aeolic OR noun pl fem gen doric aeolic}
 \item \eintrag{θεῷ}{ θεάω θεός }{keine Übersetzung gefunden}{verb 3rd sg pres opt act contr OR noun sg masc dat}
 \item \eintrag{θορυβούμενον}{ θορυβέω }{make a noise, uproar}{part sg pres mp masc acc attic epic doric contr OR part sg pres mp neut acc attic epic doric contr OR part sg pres mp neut nom attic epic doric contr OR part sg pres mp neut voc attic epic doric contr}
 \item \eintrag{θριάμβου}{ θρίαμβος }{hymn to Dionysus}{noun sg masc gen}
 \item \eintrag{θόρυβος}{ θόρυβος }{noise}{noun sg masc nom}
 \item \eintrag{θύσων}{ θύω θύω2 }{offer by burning}{part sg fut act masc nom doric contr OR part sg fut act masc nom doric poetic contr}
 \item \eintrag{θῦσαι}{ θύω θύω2 }{offer by burning}{verb 2nd sg aor imperat mid OR verb 3rd sg aor opt act OR verb aor inf act OR verb 2nd sg aor imperat mid poetic OR verb 3rd sg aor opt act poetic OR verb aor inf act poetic}
 \item \eintrag{κήρυκα}{ κῆρυξ }{herald, pursuivant}{noun sg masc acc}
 \item \eintrag{καί}{ καί καί2 καί3 καί4 }{and}{conj indeclform OR conj indeclform OR conj indeclform OR conj indeclform}
 \item \eintrag{καίπερ}{ καίπερ }{even}{partic indeclform}
 \item \eintrag{καθʼ}{ }{keine Übersetzung gefunden}{Nichts gefunden}
 \item \eintrag{καθάπερ}{ καθά }{according as, just as,}{adv indeclform}
 \item \eintrag{καθίσταντο}{ καθίστημι καθιστάω }{set down}{verb 3rd pl imperf ind mp causal pres redupl OR verb 3rd pl imperf ind mp doric ionic aeolic comp only contr}
 \item \eintrag{καθίστασθαι}{ καθίστημι καθιστάω }{set down}{verb pres inf mp causal pres redupl OR verb pres inf mp contr}
 \item \eintrag{καθαιρεθέντος}{ καθαιρέω }{take down}{part sg aor pass masc gen OR part sg aor pass neut gen}
 \item \eintrag{καθιστάμενοι}{ καθίστημι καθιστάω }{set down}{part pl pres mp masc voc causal pres redupl OR part pl pres mp masc nom causal pres redupl OR part pl pres mp masc voc doric aeolic contr OR part pl pres mp masc nom doric aeolic contr}
 \item \eintrag{καθὰ}{ καθά }{according as, just as,}{adv indeclform}
 \item \eintrag{καιρὸν}{ καῖρος καῖρος καιρός }{the row of thrums}{noun sg masc acc OR noun sg masc acc OR noun sg masc acc}
 \item \eintrag{κακοῦ}{ κακός κακόω }{bad}{adj sg neut gen OR adj sg masc gen OR verb 2nd sg imperf ind mp homeric ionic contr unaugmented OR verb 2nd sg pres imperat act contr OR verb 2nd sg pres imperat mp contr OR verb 3rd sg imperf ind act homeric ionic contr unaugmented}
 \item \eintrag{καλῶς}{ κάλως κάλως καλός καλός }{a reefing rope, reef}{noun sg masc nom attic epic ionic OR noun pl masc acc epic doric ionic OR noun sg masc nom attic epic ionic OR noun pl masc acc epic doric ionic OR adv OR adj pl masc acc doric OR adv OR adj pl masc acc doric}
 \item \eintrag{κατʼ}{ }{keine Übersetzung gefunden}{Nichts gefunden}
 \item \eintrag{κατάρχειν}{ κατάρχω }{make beginning of}{verb pres inf act attic epic contr}
 \item \eintrag{κατέλαβον}{ καταλαμβάνω }{seize, lay hold of}{verb 3rd pl aor ind act OR verb 1st sg aor ind act}
 \item \eintrag{καταγιγνωσκόντων}{ καταγιγνώσκω }{remark, observe}{verb 3rd pl pres imperat act pres redupl OR part pl pres act neut gen pres redupl OR part pl pres act masc gen pres redupl}
 \item \eintrag{καταθήσειν}{ καταθέω κατατίθημι }{run down}{verb fut inf act doric contr OR verb fut inf act doric contr}
 \item \eintrag{καταθαυμαζόμενος}{ θαυμάζω }{wonder, marvel}{part sg pres mp masc nom}
 \item \eintrag{κατακτῷτο}{ κατακτάομαι }{get for oneself, win}{verb 3rd sg pres opt mp contr}
 \item \eintrag{καταλέγειν}{ καταλέγω καταλέγω2 }{lay down}{verb pres inf act attic epic contr OR verb pres inf act attic epic contr}
 \item \eintrag{καταποντισμὸν}{ καταποντισμός }{drowning}{noun sg masc acc}
 \item \eintrag{καταφανὴς}{ καταφαίνω καταφανής }{declare, make known}{verb 2nd sg fut ind act doric contr OR verb 2nd sg aor ind pass homeric ionic unaugmented OR adj sg masc nom OR adj sg fem nom OR adj pl masc voc doric aeolic contr OR adj pl masc nom doric aeolic contr OR adj pl masc acc attic epic doric contr OR adj pl fem voc doric aeolic contr OR adj pl fem acc attic epic doric contr OR adj pl fem nom doric aeolic contr}
 \item \eintrag{καταφράκτων}{ κατάφρακτος καταφράκτης }{covered, shut up}{adj pl neut gen attic OR adj pl masc gen attic OR adj pl fem gen attic OR noun pl masc gen}
 \item \eintrag{κατελογίσατο}{ καταλογίζομαι }{count up, reckon}{verb 3rd sg aor ind mp}
 \item \eintrag{κατεργάσασθαι}{ κατεργάζομαι }{effect by labour, achieve}{verb aor inf mp}
 \item \eintrag{κατεφρόνει}{ καταφρονέω }{look down upon, think slightly of}{verb 3rd sg imperf ind act attic epic contr}
 \item \eintrag{κατεψεύσατο}{ καταψεύδομαι }{tell lies against, speak falsely of}{verb 3rd sg aor ind mid}
 \item \eintrag{κατηγορούμενον}{ κατηγορέω }{speak against}{part sg pres mp neut acc attic epic doric contr OR part sg pres mp neut nom attic epic doric contr OR part sg pres mp neut voc attic epic doric contr OR part sg pres mp masc acc attic epic doric contr}
 \item \eintrag{κατηγορούντων}{ κατηγορέω }{speak against}{verb 3rd pl pres imperat act attic epic doric contr OR part pl pres act masc gen attic epic doric contr OR part pl pres act neut gen attic epic doric contr}
 \item \eintrag{κατηγόρει}{ κατηγορέω }{speak against}{verb 3rd sg imperf ind act attic epic comp only contr unaugmented OR verb 3rd sg pres ind act attic epic doric ionic contr OR verb 2nd sg pres ind mp attic epic doric ionic comp only contr OR verb 2nd sg pres imperat act attic epic contr}
 \item \eintrag{κατηγόρουν}{ κατηγορέω }{speak against}{verb 3rd pl imperf ind act attic epic doric comp only contr OR verb 1st sg imperf ind act attic epic doric comp only contr unaugmented OR part sg pres act neut voc attic epic doric contr OR part sg pres act neut nom attic epic doric contr OR part sg pres act neut acc attic epic doric contr OR part sg pres act masc voc attic epic doric comp only contr}
 \item \eintrag{κατὰ}{ κάτος κατά κατά2 }{following}{adj pl neut voc OR adj pl neut nom OR adj pl neut acc OR prep indeclform OR prep indeclform}
 \item \eintrag{κατῆρξαν}{ κατάρχω }{make beginning of}{verb 3rd pl aor ind act attic epic ionic}
 \item \eintrag{καὶ}{ καί καί2 καί3 καί4 }{and}{conj indeclform OR conj indeclform OR conj indeclform OR conj indeclform}
 \item \eintrag{κεκομίκασι}{ κομίζω }{take care of, provide for}{verb 3rd pl perf ind act redupl}
 \item \eintrag{κεκριμένον}{ κρίνω }{separate, put asunder, distinguish}{part sg perf mp neut voc redupl OR part sg perf mp neut nom redupl OR part sg perf mp neut acc redupl OR part sg perf mp masc acc redupl}
 \item \eintrag{κεκτῆσθαι}{ κτάομαι κτέομαι }{procure for oneself, get, acquire}{verb perf inf mp OR verb perf inf mp redupl}
 \item \eintrag{κεχαρισμένος}{ χαρίζομαι χαρίζω }{to say}{part sg perf mp masc nom redupl OR part sg perf mp masc nom redupl}
 \item \eintrag{κηρύγματι}{ κήρυγμα }{that which is cried by a herald, proclamation}{noun sg neut dat}
 \item \eintrag{κοινῷ}{ κοινός κοινόω }{common}{adj sg neut dat OR adj sg masc dat OR adj sg fem dat rare OR verb 3rd sg pres opt act doric contr}
 \item \eintrag{κοιρανέων}{ κοιρανέω }{to be lord}{part sg pres act masc nom epic doric ionic aeolic}
 \item \eintrag{κρίσιν}{ κρίσις }{separating, distinguishing}{noun sg fem acc}
 \item \eintrag{κρατῆσαι}{ κρατέω }{to be strong, powerful}{verb aor inf act OR verb 3rd sg aor opt act OR verb 2nd sg aor imperat mid}
 \item \eintrag{κρινεῖν}{ κρίνω }{separate, put asunder, distinguish}{verb pres inf act attic epic contr OR verb fut inf act attic epic doric contr}
 \item \eintrag{κρύφα}{ κρύφα κρυφᾶ }{without the knowledge of}{adv indeclform OR adv indeclform}
 \item \eintrag{κρῖναι}{ κρίνω }{separate, put asunder, distinguish}{verb aor inf act OR verb 2nd sg aor imperat mid OR verb 3rd sg aor opt act}
 \item \eintrag{κυρίου}{ κύριος κύριος κυριόω }{having power}{noun sg masc gen OR adj sg neut gen OR adj sg masc gen OR adj sg fem gen OR noun sg masc gen OR adj sg neut gen OR adj sg masc gen OR adj sg fem gen OR verb 2nd sg pres imperat mp contr OR verb 3rd sg imperf ind act homeric ionic contr unaugmented OR verb 2nd sg imperf ind mp homeric ionic contr unaugmented OR verb 2nd sg pres imperat act contr}
 \item \eintrag{κωλύειν}{ κωλύω }{hinder, prevent}{verb pres inf act attic epic contr}
 \item \eintrag{κωλύετε}{ κωλύω }{hinder, prevent}{verb 2nd pl pres ind act OR verb 2nd pl imperf ind act homeric ionic unaugmented OR verb 2nd pl pres imperat act}
 \item \eintrag{κόλπον}{ κόλπος }{bosom, lap}{noun sg masc acc}
 \item \eintrag{κἀκεῖνος}{ ἐκεῖνος }{the person there, that person}{adj sg masc nom}
 \item \eintrag{λέγειν}{ λέγω λέγω2 λέγω3 }{lay}{verb pres inf act attic epic contr OR verb pres inf act attic epic contr OR verb pres inf act attic epic contr}
 \item \eintrag{λέγοιεν}{ λέγω λέγω2 λέγω3 }{lay}{verb 3rd pl pres opt act OR verb 3rd pl pres opt act OR verb 3rd pl pres opt act}
 \item \eintrag{λέγοντος}{ λέγω λέγω2 λέγω3 }{lay}{part sg pres act neut gen OR part sg pres act masc gen OR part sg pres act neut gen OR part sg pres act masc gen OR part sg pres act neut gen OR part sg pres act masc gen}
 \item \eintrag{λήγουσι}{ λήγω }{stay, abate}{part pl pres act neut dat attic epic doric ionic OR verb 3rd pl pres ind act attic epic doric ionic OR part pl pres act masc dat attic epic doric ionic}
 \item \eintrag{λανθάνων}{ λανθάνω }{escape notice}{part sg pres act masc nom n infix}
 \item \eintrag{λαοῖσί}{ λαός }{men}{noun pl masc dat epic ionic aeolic}
 \item \eintrag{λείαν}{ λεία λεία2 λεῖος }{tool for smoothing stone}{noun sg fem acc attic doric aeolic OR noun pl fem gen doric aeolic OR noun sg fem acc attic doric aeolic OR noun pl fem gen doric aeolic OR adj sg fem acc attic doric aeolic OR adj pl masc gen doric OR adj pl fem gen doric}
 \item \eintrag{λησόμενα}{ λανθάνω }{escape notice}{part sg fut mid fem voc doric aeolic OR part sg fut mid fem nom doric aeolic OR part pl fut mid neut voc OR part pl fut mid neut nom OR part pl fut mid neut acc OR part dual fut mid fem voc OR part dual fut mid fem nom OR part dual fut mid fem acc}
 \item \eintrag{λογίσασθαι}{ λογίζομαι }{count, reckon}{verb aor inf mp}
 \item \eintrag{λογισμὸν}{ λογισμός }{counting, calculation}{noun sg masc acc}
 \item \eintrag{λογισμῷ}{ λογισμός }{counting, calculation}{noun sg masc dat}
 \item \eintrag{λοιποὶ}{ λοιπός }{remaining over}{adj pl masc voc OR adj pl masc nom}
 \item \eintrag{λοιπὰ}{ λοιπάς λοιπάζω λοιπός }{remainder,}{noun sg fem voc OR verb 1st sg fut ind act doric aeolic contr OR adj sg fem nom doric aeolic OR adj sg fem voc doric aeolic OR adj pl neut voc OR adj pl neut acc OR adj pl neut nom OR adj dual fem voc OR adj dual fem acc OR adj dual fem nom}
 \item \eintrag{λοιπὸν}{ λοιπός }{remaining over}{adj sg neut voc OR adj sg neut nom OR adj sg neut acc OR adj sg masc acc}
 \item \eintrag{λουομένῳ}{ λούω }{l[acaron]vo}{part sg pres mp neut dat OR part sg pres mp masc dat}
 \item \eintrag{λόγον}{ λόγος }{computation, reckoning}{noun sg masc acc}
 \item \eintrag{λόγος}{ λόγος }{computation, reckoning}{noun sg masc nom}
 \item \eintrag{λόγους}{ λόγος λογόω }{computation, reckoning}{noun pl masc acc OR verb 2nd sg pres ind act doric contr OR verb 2nd sg imperf ind act homeric ionic contr unaugmented}
 \item \eintrag{λόγῳ}{ λόγος λογάω }{computation, reckoning}{noun sg masc dat OR verb 3rd sg pres opt act contr}
 \item \eintrag{μάλιστα}{ μάλιστα }{keine Übersetzung gefunden}{adv indeclform}
 \item \eintrag{μάχας}{ μάχη μαχάω }{battle, combat}{noun sg fem gen doric aeolic OR noun pl fem acc OR verb 2nd sg pres ind act doric contr OR verb 2nd sg imperf ind act homeric ionic contr unaugmented}
 \item \eintrag{μάχης}{ μάχη μαχάω συμμαχέω }{battle, combat}{noun sg fem gen attic epic ionic OR verb 2nd sg pres ind act doric contr OR verb 2nd sg imperf ind act homeric ionic unaugmented OR verb 2nd sg pres ind act doric contr}
 \item \eintrag{μέγα}{ Μέγης μέγας }{keine Übersetzung gefunden}{noun sg masc voc OR noun sg masc nom epic OR noun sg masc gen doric aeolic contr OR noun dual masc nom OR noun dual masc voc OR noun dual masc acc OR adj sg neut voc indeclform OR adj sg neut nom indeclform OR adj sg neut acc indeclform}
 \item \eintrag{μέγαν}{ Μέγης μέγας }{keine Übersetzung gefunden}{noun sg masc acc epic doric aeolic OR noun pl masc gen doric aeolic OR adj sg masc acc indeclform}
 \item \eintrag{μέθης}{ μέθη }{strong drink}{noun sg fem gen attic epic ionic}
 \item \eintrag{μέμφονται}{ μέμφομαι }{blame, censure}{verb 3rd pl pres ind mp}
 \item \eintrag{μέν}{ μέν }{indeed, of a truth}{partic indeclform}
 \item \eintrag{μέρει}{ μέρος }{share, portion}{noun dual neut voc attic epic contr OR noun sg neut dat epic ionic OR noun dual neut acc attic epic contr OR noun dual neut nom attic epic contr}
 \item \eintrag{μέρος}{ μέρος }{share, portion}{noun sg neut voc OR noun sg neut acc OR noun sg neut nom}
 \item \eintrag{μέσαις}{ μέση μέσης μέσος μεσάζω }{mese}{noun pl fem dat OR noun pl masc dat OR adj pl fem dat OR verb 2nd sg fut ind act epic contr}
 \item \eintrag{μέχρι}{ μέχρι }{as far as}{prep indeclform OR conj indeclform OR adv indeclform}
 \item \eintrag{μή}{ μή μή2 μή3 μής }{mā´}{conj indeclform OR conj indeclform OR conj indeclform OR noun sg masc voc}
 \item \eintrag{μήθʼ}{ }{keine Übersetzung gefunden}{Nichts gefunden}
 \item \eintrag{μήτε}{ μήτε }{and not}{partic indeclform}
 \item \eintrag{μαθοῦσα}{ μανθάνω }{learn}{part sg aor act fem voc attic epic doric ionic OR part sg aor act fem nom attic epic doric ionic OR part dual aor act fem voc attic epic doric ionic OR part dual aor act fem nom attic epic doric ionic OR part dual aor act fem acc attic epic doric ionic}
 \item \eintrag{μαθὼν}{ μάθη μάθος μανθάνω }{keine Übersetzung gefunden}{noun pl fem gen OR noun pl neut gen attic epic doric contr OR part sg aor act masc nom}
 \item \eintrag{μείζονος}{ μέγας }{big}{adj sg gen comp}
 \item \eintrag{μεγάλοις}{ μέγας }{big}{adj pl neut dat OR adj pl masc dat}
 \item \eintrag{μεθʼ}{ }{keine Übersetzung gefunden}{Nichts gefunden}
 \item \eintrag{μεθειμένων}{ μεθίημι }{set loose, let go}{part pl perf mp neut gen OR part pl perf mp masc gen OR part pl perf mp fem gen}
 \item \eintrag{μετʼ}{ }{keine Übersetzung gefunden}{Nichts gefunden}
 \item \eintrag{μετέπιπτον}{ μεταπίπτω }{fall differently, undergo a change}{verb 1st sg imperf ind act pres redupl OR verb 3rd pl imperf ind act pres redupl}
 \item \eintrag{μετέφερον}{ μεταφέρω }{carry across, transfer}{verb 3rd pl imperf ind act OR verb 1st sg imperf ind act}
 \item \eintrag{μεταβολῆς}{ μεταβολεύς μεταβολή }{one who exchanges}{noun pl masc voc contr OR noun pl masc nom contr OR noun sg fem gen attic epic ionic}
 \item \eintrag{μεταγιγνώσκοντες}{ μεταγιγνώσκω }{find out after}{part pl pres act masc voc pres redupl OR part pl pres act masc nom pres redupl}
 \item \eintrag{μεταθέσθαι}{ μετατίθημι }{place among}{verb aor inf mp}
 \item \eintrag{μεταθήσεσθαι}{ μεταθέω μετατίθημι }{run after}{verb fut inf mid OR verb fut inf mid}
 \item \eintrag{μετακαλούντων}{ μετακαλέω }{recall}{verb 3rd pl pres imperat act attic epic doric contr OR part pl pres act neut gen attic epic doric contr OR part pl pres act masc gen attic epic doric contr OR part pl fut act neut gen attic epic doric contr OR part pl fut act masc gen attic epic doric contr}
 \item \eintrag{μεταστάντος}{ μεθίστημι }{place in another way, change}{part sg aor act neut gen OR part sg aor act masc gen}
 \item \eintrag{μετεπέμπετο}{ μεταπέμπω }{send after}{verb 3rd sg imperf ind mp}
 \item \eintrag{μετὰ}{ μετά }{mip}{prep indeclform}
 \item \eintrag{μηδʼ}{ }{keine Übersetzung gefunden}{Nichts gefunden}
 \item \eintrag{μηδετέρους}{ μηδέτερος }{neither of the two}{adj pl masc acc}
 \item \eintrag{μηδὲν}{ μηδείς }{not one, not even one, nobody}{adj sg neut nom indeclform OR adj sg neut voc indeclform OR adj sg neut acc indeclform}
 \item \eintrag{μηχανήματα}{ μηχάνημα }{machine}{noun pl neut nom OR noun pl neut voc OR noun pl neut acc}
 \item \eintrag{μικρολόγος}{ μικρολόγος }{counting trifles, careful about trifles}{adj sg masc nom OR adj sg fem nom}
 \item \eintrag{μισθοφόρους}{ μισθόφορος μισθοφόρος }{serving for hire}{adj pl masc acc OR adj pl fem acc OR adj pl masc acc OR adj pl fem acc}
 \item \eintrag{μιᾶς}{ εἷς }{sem}{adj sg fem gen attic doric aeolic}
 \item \eintrag{μοι}{ ἐγώ }{I at least, for my part, indeed, for myself}{pron 1st sg masc dat enclitic indeclform OR pron 1st sg fem dat enclitic indeclform}
 \item \eintrag{μυρίους}{ μυρίος }{numberless, countless, infinite}{adj pl masc acc}
 \item \eintrag{μυριάδων}{ μυριάς }{number of}{noun pl fem gen}
 \item \eintrag{μόλις}{ μόλις }{only just}{adv indeclform}
 \item \eintrag{μόνου}{ μόνος μονόω }{alone, solitary}{adj sg neut gen OR adj sg masc gen OR verb 3rd sg imperf ind act homeric ionic contr unaugmented OR verb 2nd sg pres imperat mp contr OR verb 2nd sg imperf ind mp homeric ionic contr unaugmented OR verb 2nd sg pres imperat act contr}
 \item \eintrag{μὲν}{ μέν }{indeed, of a truth}{partic indeclform}
 \item \eintrag{μὴ}{ μή μή2 μή3 μής }{mā´}{conj indeclform OR conj indeclform OR conj indeclform OR noun sg masc voc}
 \item \eintrag{μᾶλλον}{ μᾶλλον μαλλός }{keine Übersetzung gefunden}{adv indeclform OR noun sg masc acc}
 \item \eintrag{νέον}{ νέος νέω νέω2 νέω3 νέω4 νειός }{young, youthful}{adj sg neut voc attic OR adj sg neut acc attic OR adj sg neut nom attic OR adj sg masc acc OR adj sg fem acc attic OR verb 3rd pl imperf ind act epic doric ionic aeolic unaugmented OR verb 1st sg imperf ind act epic doric ionic aeolic unaugmented OR part sg pres act neut voc epic doric ionic aeolic OR part sg pres act neut nom epic doric ionic aeolic OR part sg pres act neut acc epic doric ionic aeolic OR part sg pres act masc voc epic doric ionic aeolic OR verb 3rd pl imperf ind act epic doric ionic aeolic unaugmented OR verb 1st sg imperf ind act epic doric ionic aeolic unaugmented OR part sg pres act neut voc epic doric ionic aeolic OR part sg pres act neut nom epic doric ionic aeolic OR part sg pres act neut acc epic doric ionic aeolic OR part sg pres act masc voc epic doric ionic aeolic OR verb 3rd pl imperf ind act epic doric ionic aeolic unaugmented OR verb 1st sg imperf ind act epic doric ionic aeolic unaugmented OR part sg pres act neut voc epic doric ionic aeolic OR part sg pres act neut nom epic doric ionic aeolic OR part sg pres act neut acc epic doric ionic aeolic OR part sg pres act masc voc epic doric ionic aeolic OR verb 3rd pl imperf ind act epic doric ionic aeolic unaugmented OR verb 1st sg imperf ind act epic doric ionic aeolic unaugmented OR part sg pres act neut voc epic doric ionic aeolic OR part sg pres act neut nom epic doric ionic aeolic OR part sg pres act neut acc epic doric ionic aeolic OR part sg pres act masc voc epic doric ionic aeolic OR noun sg fem acc indeclform}
 \item \eintrag{νέου}{ νέος νεάω νεόω }{young, youthful}{adj sg neut gen attic OR adj sg masc gen attic OR adj sg fem gen attic OR verb 2nd sg pres imperat mp attic epic ionic contr OR verb 3rd sg imperf ind act homeric ionic contr unaugmented OR verb 2nd sg pres imperat mp contr OR verb 2nd sg pres imperat act contr OR verb 2nd sg imperf ind mp homeric ionic contr unaugmented}
 \item \eintrag{νήσους}{ νῆσος }{island,}{noun pl fem acc}
 \item \eintrag{νίκην}{ νίκη νίκη2 νῖκος νικάω }{victory,}{noun sg fem acc attic epic ionic OR noun sg fem acc attic epic ionic OR noun sg neut acc OR verb pres inf act doric ionic contr OR verb 3rd pl imperf ind act epic doric aeolic unaugmented OR verb 1st sg imperf ind act homeric ionic unaugmented}
 \item \eintrag{νίκης}{ Νικεύς νίκη νίκη2 νικάω }{keine Übersetzung gefunden}{noun pl masc voc contr OR noun pl masc nom contr OR noun sg fem gen attic epic ionic OR noun sg fem gen attic epic ionic OR verb 2nd sg pres ind act doric contr OR verb 2nd sg imperf ind act homeric ionic unaugmented}
 \item \eintrag{ναυαρχοῦντος}{ ναυαρχέω }{command a fleet}{part sg pres act neut gen attic epic doric contr OR part sg pres act masc gen attic epic doric contr}
 \item \eintrag{ναῦς}{ ναῦς }{ship,}{noun pl fem acc poetic indeclform OR noun sg fem nom doric indeclform OR noun sg nom attic indeclform OR noun pl acc attic indeclform}
 \item \eintrag{νεωτέρους}{ νέος νεώτερος }{young, youthful}{adj pl masc acc comp OR adj pl masc acc}
 \item \eintrag{νεωτεριζόντων}{ νεωτερίζω }{makeinnovations,}{verb 3rd pl pres imperat act OR part pl pres act neut gen OR part pl pres act masc gen}
 \item \eintrag{νεότητα}{ νεότης }{youth,}{noun sg fem acc}
 \item \eintrag{νεώτερον}{ νέος νεώτερος }{young, youthful}{adv comp OR adv attic OR adj sg neut nom comp OR adj sg neut voc comp OR adj sg neut acc comp OR adj sg masc acc comp OR adj sg neut voc OR adj sg neut nom OR adj sg neut acc OR adj sg masc acc}
 \item \eintrag{νεῶν}{ ναός ναός ναῦς νέα νέος νέω νέω2 νέω3 νέω4 νεάω νεάζω νεόω νεών νεώς }{2 Ma.)}{noun sg masc acc attic epic ionic OR noun pl masc gen attic epic ionic OR noun sg masc acc attic epic ionic OR noun pl masc gen attic epic ionic OR noun pl gen attic indeclform OR noun pl fem gen epic doric ionic indeclform OR noun pl fem gen OR adj pl neut gen OR adj pl fem gen OR adj pl masc gen OR part sg pres act masc nom epic doric ionic aeolic OR part sg pres act masc nom epic doric ionic aeolic OR part sg pres act masc nom epic doric ionic aeolic OR part sg pres act masc nom epic doric ionic aeolic OR part sg pres act neut voc contr OR part sg pres act neut nom contr OR part sg pres act masc voc contr OR part sg pres act neut acc contr OR part sg pres act masc nom attic epic ionic contr OR part sg fut act neut nom contr OR part sg fut act neut voc contr OR part sg fut act neut acc contr OR part sg fut act masc voc contr OR verb pres inf act doric OR part sg pres act neut voc doric aeolic contr OR part sg pres act neut nom doric aeolic contr OR part sg pres act neut acc doric aeolic contr OR part sg pres act masc voc doric aeolic contr OR part sg pres act masc nom contr OR noun sg masc voc OR noun sg masc nom OR noun sg masc acc attic epic ionic OR noun pl masc gen attic epic ionic}
 \item \eintrag{νηφάλιον}{ νηφάλιος }{unmixed with wine,}{adj sg neut voc OR adj sg neut nom OR adj sg neut acc OR adj sg masc acc}
 \item \eintrag{νυμφαγωγίας}{ νυμφαγωγία }{bridal procession,}{noun sg fem gen attic doric aeolic OR noun pl fem acc}
 \item \eintrag{νόμοις}{ νόμος νόμος νομός νομός }{anything assigned, a usage, custom, law, ordinance}{noun pl masc dat OR noun pl masc dat OR noun pl masc dat OR noun pl masc dat}
 \item \eintrag{νόμῳ}{ νόμος νόμος νομάζω νομός νομός }{anything assigned, a usage, custom, law, ordinance}{noun sg masc dat OR noun sg masc dat OR verb 3rd sg fut opt act contr OR noun sg masc dat OR noun sg masc dat}
 \item \eintrag{οἰκείοις}{ οἰκέω οἰκείω οἰκεῖος οἰκειόω }{inhabit,}{verb 2nd sg pres opt act epic OR verb 2nd sg pres opt act OR adj pl neut dat OR adj pl masc dat OR adj pl fem dat OR verb 2nd sg pres subj act contr OR verb 2nd sg pres ind act contr OR verb 2nd sg pres opt act contr}
 \item \eintrag{οἰκείως}{ οἰκεῖος οἰκειόω }{in}{adv OR adj pl masc acc doric OR adj pl fem acc doric OR verb 2nd sg pres ind act doric contr OR verb 2nd sg imperf ind act doric aeolic poetic contr unaugmented}
 \item \eintrag{οἰκείῳ}{ οἰκεῖος }{in}{adj sg neut dat OR adj sg masc dat OR adj sg fem dat}
 \item \eintrag{οἰκτιζόμενος}{ οἰκτίζω }{pity, have pity upon,}{part sg pres mp masc nom}
 \item \eintrag{οἱ}{ ἕ ὅς ὅς ὁ }{sui.}{pron sg masc dat epic ionic enclitic indeclform OR pron sg fem dat epic ionic enclitic indeclform OR pron pl masc nom indeclform OR pron pl masc nom indeclform OR article pl masc voc proclitic indeclform OR article pl masc nom proclitic indeclform}
 \item \eintrag{οἳ}{ ἕ ὅς ὅς ὁ }{sui.}{pron sg fem dat epic ionic enclitic indeclform OR pron sg masc dat epic ionic enclitic indeclform OR pron pl masc nom indeclform OR pron pl masc nom indeclform OR article pl masc voc proclitic indeclform OR article pl masc nom proclitic indeclform}
 \item \eintrag{οἵδε}{ ὅδε }{this,}{pron pl masc nom indeclform}
 \item \eintrag{οἷ}{ ἕ ὅς ὅς ὁ }{sui.}{pron sg masc dat epic ionic enclitic indeclform OR pron sg fem dat epic ionic enclitic indeclform OR pron pl masc nom indeclform OR pron pl masc nom indeclform OR article pl masc voc proclitic indeclform OR article pl masc nom proclitic indeclform}
 \item \eintrag{οἷς}{ ὅς ὅς }{yas, yā, yad,}{pron pl neut dat indeclform OR pron pl masc dat indeclform OR pron pl neut dat indeclform OR pron pl masc dat indeclform}
 \item \eintrag{οὐ}{ οὐ οὐ10 οὐ11 οὐ12 οὐ13 οὐ14 οὐ15 οὐ16 οὐ17 οὐ18 οὐ2 οὐ3 οὐ4 οὐ5 οὐ6 οὐ7 οὐ8 οὐ9 }{fact}{adv proclitic indeclform OR adv proclitic indeclform OR adv proclitic indeclform OR adv proclitic indeclform OR adv proclitic indeclform OR adv proclitic indeclform OR adv proclitic indeclform OR adv proclitic indeclform OR adv proclitic indeclform OR adv proclitic indeclform OR adv proclitic indeclform OR adv proclitic indeclform OR adv proclitic indeclform OR adv proclitic indeclform OR adv proclitic indeclform OR adv proclitic indeclform OR adv proclitic indeclform OR adv proclitic indeclform}
 \item \eintrag{οὐδʼ}{ }{keine Übersetzung gefunden}{Nichts gefunden}
 \item \eintrag{οὐδέν}{ οὐδείς }{not one}{adj sg neut voc indeclform OR adj sg neut nom indeclform OR adj sg neut acc indeclform}
 \item \eintrag{οὐδένα}{ οὐδείς }{not one}{adj sg masc acc indeclform OR adj sg fem acc indeclform OR adj pl neut voc indeclform OR adj pl neut nom indeclform OR adj pl neut acc indeclform}
 \item \eintrag{οὐδέτεροι}{ οὐδέτερος }{not either, neither of the two}{adj pl masc voc OR adj pl masc nom}
 \item \eintrag{οὐδενὸς}{ οὐδείς }{not one}{adj sg gen indeclform}
 \item \eintrag{οὐδὲ}{ οὐδέ οὐδέ2 οὐδέ3 οὐδός οὐδός }{but not}{partic indeclform OR partic indeclform OR partic indeclform OR noun sg masc voc OR noun sg fem voc ionic}
 \item \eintrag{οὐδὲν}{ οὐδείς }{not one}{adj sg neut voc indeclform OR adj sg neut nom indeclform OR adj sg neut acc indeclform}
 \item \eintrag{οὐκ}{ οὐ οὐ10 οὐ11 οὐ12 οὐ13 οὐ14 οὐ15 οὐ16 οὐ17 οὐ18 οὐ2 οὐ3 οὐ4 οὐ5 οὐ6 οὐ7 οὐ8 οὐ9 }{fact}{adv proclitic indeclform OR adv proclitic indeclform OR adv proclitic indeclform OR adv proclitic indeclform OR adv proclitic indeclform OR adv proclitic indeclform OR adv proclitic indeclform OR adv proclitic indeclform OR adv proclitic indeclform OR adv proclitic indeclform OR adv proclitic indeclform OR adv proclitic indeclform OR adv proclitic indeclform OR adv proclitic indeclform OR adv proclitic indeclform OR adv proclitic indeclform OR adv proclitic indeclform OR adv proclitic indeclform}
 \item \eintrag{οὐκέτι}{ οὐκέτι }{no more, no longer, no further}{adv indeclform}
 \item \eintrag{οὓς}{ ὅς ὅς }{yas, yā, yad,}{pron pl masc acc indeclform OR pron pl masc acc indeclform}
 \item \eintrag{οὔπω}{ οὔπω }{not yet}{adv indeclform}
 \item \eintrag{οὔτε}{ οὔτε }{and not}{adv indeclform}
 \item \eintrag{οὕτω}{ οὕτως }{in this way}{adv indeclform}
 \item \eintrag{οὕτως}{ οὕτως }{in this way}{adv indeclform}
 \item \eintrag{οὖν}{ οὖν }{certainly, in fact}{partic indeclform}
 \item \eintrag{οὖσι}{ εἰμί ὄντα }{sum}{part pl pres act neut dat attic epic doric ionic OR part pl pres act masc dat attic epic doric ionic OR noun pl neut dat attic epic doric ionic indeclform}
 \item \eintrag{πάθοι}{ πάσχω }{have}{verb 3rd sg aor opt act}
 \item \eintrag{πάλαι}{ πάλα πάλαι πάλη πάλη2 πάλλω }{nugget}{noun pl fem voc OR noun sg fem dat doric aeolic OR noun pl fem nom OR adv indeclform OR noun sg fem dat doric aeolic OR noun pl fem voc OR noun pl fem nom OR noun sg fem dat doric aeolic OR noun pl fem voc OR noun pl fem nom OR verb aor inf act doric OR verb 3rd sg aor opt act doric OR verb 2nd sg aor imperat mid doric}
 \item \eintrag{πάλιν}{ πάλιν }{back, backwards}{adv indeclform}
 \item \eintrag{πάμπαν}{ πάμπαν }{wholly, altogether}{adv indeclform}
 \item \eintrag{πάντα}{ πάντῃ πᾶς πᾶς πᾶϲ }{every way, on every side}{adv doric indeclform OR adj sg masc acc indeclform OR adj pl neut voc indeclform OR adj pl neut nom indeclform OR adj pl neut acc indeclform OR adj sg masc acc indeclform OR adj pl neut voc indeclform OR adj pl neut nom indeclform OR adj pl neut acc indeclform OR adj sg masc acc indeclform OR adj pl neut voc indeclform OR adj pl neut nom indeclform OR adj pl neut acc indeclform}
 \item \eintrag{πάντων}{ πᾶς πᾶς πᾶϲ }{all}{adj pl neut gen indeclform OR adj pl masc gen indeclform OR adj pl neut gen indeclform OR adj pl masc gen indeclform OR adj pl neut gen indeclform OR adj pl masc gen indeclform}
 \item \eintrag{πάσας}{ πάσσω πᾶς πᾶς πᾶϲ }{sprinkle}{verb 2nd sg aor ind act homeric ionic unaugmented OR part sg aor act masc voc attic epic ionic OR part sg aor act masc nom attic epic ionic OR adj sg fem gen doric indeclform OR adj pl fem acc doric indeclform OR adj sg fem gen doric indeclform OR adj pl fem acc doric indeclform OR adj sg fem gen doric indeclform OR adj pl fem acc doric indeclform}
 \item \eintrag{πέδας}{ πέδη πεδάω }{fetter}{noun pl fem acc OR noun sg fem gen doric aeolic OR verb 2nd sg imperf ind act homeric ionic contr unaugmented OR verb 2nd sg pres ind act doric contr}
 \item \eintrag{πέμψαντας}{ πέμπω }{send}{part pl aor act masc acc}
 \item \eintrag{πέντε}{ πέντε }{five}{numeral indeclform}
 \item \eintrag{πέρι}{ περί }{round about, all round}{prep indeclform}
 \item \eintrag{παίδων}{ παῖς }{child}{noun pl masc gen indeclform OR noun pl fem gen indeclform}
 \item \eintrag{παθεῖν}{ πάσχω }{have}{verb aor inf act attic epic doric contr}
 \item \eintrag{παθόντες}{ πάσχω }{have}{part pl aor act masc voc OR part pl aor act masc nom}
 \item \eintrag{παρʼ}{ }{keine Übersetzung gefunden}{Nichts gefunden}
 \item \eintrag{παρά}{ παρά πῆρος πηρός }{beside}{prep indeclform OR noun pl neut voc doric aeolic contr OR noun pl neut acc doric aeolic contr OR noun pl neut nom doric aeolic contr OR adj sg fem nom attic doric aeolic OR adj sg fem voc attic doric aeolic OR adj pl neut voc doric OR adj pl neut nom doric OR adj pl neut acc doric OR adj dual fem voc doric OR adj dual fem nom doric OR adj dual fem acc doric}
 \item \eintrag{παράδοξον}{ παράδοξος }{contrary to expectation, incredible}{adj sg neut voc OR adj sg neut nom OR adj sg neut acc OR adj sg masc acc OR adj sg fem acc}
 \item \eintrag{παρέπεμπεν}{ παραπέμπω }{send past}{verb 3rd sg imperf ind act nu movable}
 \item \eintrag{παραγενέσθαι}{ παραγίγνομαι }{to be beside, by}{verb aor inf mid}
 \item \eintrag{παραδοῦναι}{ παραδίδωμι }{give, hand over to another, transmit}{verb aor inf act}
 \item \eintrag{παρακαλεῖν}{ παρακαλέω }{call to}{verb pres inf act attic epic doric contr OR verb fut inf act attic epic doric contr}
 \item \eintrag{παρακαλοῦντι}{ παρακαλέω }{call to}{verb 3rd pl pres ind act doric contr OR verb 3rd pl fut ind act attic doric contr OR part sg pres act neut dat attic epic doric contr OR part sg pres act masc dat attic epic doric contr OR part sg fut act neut dat attic epic doric contr OR part sg fut act masc dat attic epic doric contr}
 \item \eintrag{παραλαβεῖν}{ παραλαμβάνω }{receive from}{verb aor inf act attic epic doric contr}
 \item \eintrag{παραλόγως}{ παράλογος }{beyond calculation, unexpected, unlooked for}{noun pl masc acc doric OR adv OR adj pl masc acc doric OR adj pl fem acc doric}
 \item \eintrag{παρανομήματα}{ παρανόμημα }{unlawful act, transgression}{noun pl neut voc OR noun pl neut nom OR noun pl neut acc}
 \item \eintrag{παραπεμφθείσας}{ παραπέμπω }{send past}{part pl aor pass fem acc OR part sg aor pass fem gen doric aeolic}
 \item \eintrag{παρασκευήν}{ παρασκευή }{preparation}{noun sg fem acc attic epic ionic}
 \item \eintrag{παρασκευὴν}{ παρασκευή }{preparation}{noun sg fem acc attic epic ionic}
 \item \eintrag{παρελθούσης}{ παρέρχομαι }{ibo}{part sg aor act fem gen attic epic ionic}
 \item \eintrag{παρελθὼν}{ παρέρχομαι }{ibo}{part sg aor act masc nom}
 \item \eintrag{παρεσκευασμένων}{ παρασκευάζω }{get ready, prepare}{part pl perf mp neut gen redupl OR part pl perf mp masc gen redupl OR part pl perf mp fem gen redupl}
 \item \eintrag{παρρησίᾳ}{ παρρησία παρρησιάζομαι }{outspokenness, frankness, freedom of speech}{noun sg fem dat attic doric aeolic OR noun pl fem voc OR noun pl fem nom OR verb 2nd sg fut ind mp epic contr}
 \item \eintrag{παρόντος}{ πάρειμι }{sum}{part sg pres act neut gen OR part sg pres act masc gen}
 \item \eintrag{παρὰ}{ παρά πῆρος πηρός }{beside}{prep indeclform OR noun pl neut voc doric aeolic contr OR noun pl neut acc doric aeolic contr OR noun pl neut nom doric aeolic contr OR adj sg fem nom attic doric aeolic OR adj sg fem voc attic doric aeolic OR adj pl neut voc doric OR adj pl neut nom doric OR adj pl neut acc doric OR adj dual fem voc doric OR adj dual fem nom doric OR adj dual fem acc doric}
 \item \eintrag{πατρικὸν}{ πατρικός }{derived from one's fathers, hereditary}{adj sg neut voc OR adj sg neut nom OR adj sg neut acc OR adj sg masc acc}
 \item \eintrag{πατρὸς}{ πατήρ }{pitṛ[snull ]u}{noun sg masc gen syncope}
 \item \eintrag{πατρῴας}{ πατρώιος πατρῷος }{of one's fathers, ancestral}{adj sg fem gen attic doric aeolic OR adj pl fem acc OR adj sg fem gen doric aeolic OR adj pl fem acc}
 \item \eintrag{πατὴρ}{ πατήρ }{pitṛ[snull ]u}{noun sg masc nom}
 \item \eintrag{παῖς}{ πᾶς πᾶς πᾶϲ παῖς }{all}{adj sg masc voc doric aeolic contr indeclform OR adj sg masc nom doric aeolic contr indeclform OR adj sg masc voc doric aeolic contr indeclform OR adj sg masc nom doric aeolic contr indeclform OR adj sg masc voc doric aeolic contr indeclform OR adj sg masc nom doric aeolic contr indeclform OR noun sg masc nom epic poetic indeclform OR noun sg fem nom epic poetic indeclform}
 \item \eintrag{πείρᾳ}{ πείρω πεῖρα πειρά2 πειράω πειράζω }{pierce, run through}{verb aor inf act OR verb 2nd sg aor imperat mid OR noun sg fem dat attic doric aeolic OR noun pl fem voc OR noun pl fem nom OR noun sg fem dat attic doric aeolic OR noun pl fem voc OR noun pl fem nom OR verb 3rd sg pres subj act contr OR verb 3rd sg pres ind act epic contr OR verb 2nd sg pres subj mp contr OR verb 2nd sg pres ind mp epic contr OR verb 3rd sg fut ind act epic contr OR verb 2nd sg fut ind mid epic contr}
 \item \eintrag{πεδίοις}{ μέτειμι2 πέδιον πεδάω πεδίον πεδίον2 πεδίζω }{ibo}{verb 2nd sg pres opt act doric aeolic meta to peda OR noun pl neut dat OR verb 2nd sg pres opt act epic doric ionic OR noun pl neut dat OR noun pl neut dat OR verb 2nd sg fut opt act attic epic doric contr}
 \item \eintrag{πεζοὺς}{ πεζός }{on foot, walking}{adj pl masc acc}
 \item \eintrag{πεζῷ}{ πεζός }{on foot, walking}{adj sg masc dat OR adj sg neut dat}
 \item \eintrag{πειρώμενος}{ πειράω πειράζω }{attempt, endeavour, try}{part sg pres mp masc nom contr OR part sg fut mid masc nom contr}
 \item \eintrag{πειρᾶσθαι}{ πείρω πειράω πειράζω }{pierce, run through}{verb aor inf mid OR verb pres inf mp contr OR verb fut inf mid}
 \item \eintrag{πεμπομένην}{ πέμπω }{send}{part sg pres mp fem acc attic epic ionic}
 \item \eintrag{πεμφθέντα}{ πέμπω }{send}{part pl aor pass neut voc OR part sg aor pass masc acc OR part pl aor pass neut nom OR part pl aor pass neut acc}
 \item \eintrag{πεντακόσια}{ πεντακόσιοι }{five hundred}{adj sg fem voc attic doric aeolic OR adj sg fem nom attic doric aeolic OR adj pl neut nom OR adj pl neut voc OR adj pl neut acc OR adj dual fem nom OR adj dual fem voc OR adj dual fem acc}
 \item \eintrag{πεντεκαίδεκα}{ πεντεκαίδεκα }{fifteen}{numeral indeclform}
 \item \eintrag{πεποιῆσθαι}{ ποιέω }{make}{verb perf inf mp redupl}
 \item \eintrag{πεπολεμωμένον}{ πολεμόω }{make hostile, make an enemy of,}{part sg perf mp neut voc redupl OR part sg perf mp neut nom redupl OR part sg perf mp neut acc redupl OR part sg perf mp masc acc redupl}
 \item \eintrag{πεπραγμένα}{ πράσσω }{pass through, pass over,}{part sg perf mp fem voc doric aeolic redupl OR part sg perf mp fem nom doric aeolic redupl OR part pl perf mp neut voc redupl OR part pl perf mp neut acc redupl OR part pl perf mp neut nom redupl OR part dual perf mp fem nom redupl OR part dual perf mp fem voc redupl OR part dual perf mp fem acc redupl}
 \item \eintrag{περ}{ πέρ }{all}{partic enclitic indeclform}
 \item \eintrag{περίοικα}{ περίοικος }{dwelling round}{adj pl neut voc OR adj pl neut nom OR adj pl neut acc}
 \item \eintrag{περαίαν}{ πέραιος περαίας περαίη περαῖος }{on the further side}{adj sg fem acc attic doric aeolic OR adj pl masc gen doric OR adj pl fem gen doric OR noun sg masc acc attic epic doric aeolic OR noun pl masc gen doric aeolic OR noun pl fem gen doric aeolic OR noun sg fem acc attic doric aeolic OR adj sg fem acc attic doric aeolic OR adj pl masc gen doric OR adj pl fem gen doric}
 \item \eintrag{περασάντων}{ περάω περάω2 }{drive right through}{verb 3rd pl aor imperat act doric aeolic OR part pl aor act masc gen doric aeolic OR part pl aor act neut gen doric aeolic OR verb 3rd pl aor imperat act OR part pl aor act neut gen OR part pl aor act masc gen}
 \item \eintrag{περιέπεμπον}{ περιπέμπω }{send round}{verb 3rd pl imperf ind act raw preverb OR verb 1st sg imperf ind act raw preverb}
 \item \eintrag{περικάθηται}{ περικάθημαι }{to be seated all round}{verb 3rd sg pres ind mid}
 \item \eintrag{περικαθήμενοι}{ κάθημαι }{to be seated, sit}{part pl perf mid masc nom OR part pl perf mid masc voc}
 \item \eintrag{περιπέμπουσιν}{ περιπέμπω }{send round}{part pl pres act masc dat attic epic doric ionic nu movable OR part pl pres act neut dat attic epic doric ionic nu movable OR verb 3rd pl pres ind act attic epic doric ionic nu movable}
 \item \eintrag{περιποιήσας}{ περιποιέω }{cause to remain over and above, keep safe, preserve}{part sg aor act masc nom attic epic ionic OR part sg aor act masc voc attic epic ionic OR verb 2nd sg aor ind act homeric ionic unaugmented}
 \item \eintrag{περὶ}{ περί }{round about, all round}{prep indeclform}
 \item \eintrag{περῶντάς}{ περάω περάω2 }{drive right through}{part pl pres act masc acc contr OR part pl fut act masc acc contr OR part pl pres act masc acc contr}
 \item \eintrag{πεῖρα}{ πείρω πεῖρα πειρά2 πειράω πειράζω }{pierce, run through}{verb 1st sg aor ind act homeric ionic unaugmented OR noun sg fem voc OR noun sg fem nom OR noun dual fem voc OR noun dual fem nom OR noun dual fem acc OR noun sg fem voc attic doric aeolic OR noun sg fem nom attic doric aeolic OR noun dual fem voc OR noun dual fem nom OR noun dual fem acc OR verb 3rd sg imperf ind act homeric ionic contr unaugmented OR verb 1st sg pres subj act doric aeolic contr OR verb 2nd sg pres imperat act contr OR verb 1st sg pres ind act doric aeolic contr OR verb 1st sg fut ind act doric aeolic contr}
 \item \eintrag{πεῖσαι}{ πείθω πεῖσα }{persuade}{verb aor inf act OR verb 3rd sg aor opt act OR verb 2nd sg aor imperat mid OR noun sg fem dat doric aeolic OR noun pl fem voc OR noun pl fem nom}
 \item \eintrag{πιθανώτατος}{ πιθανός }{persuasive, plausible}{adj sg masc nom superl}
 \item \eintrag{πιμελής}{ πιμελή πιμελής }{soft fat, lard}{noun sg fem gen attic epic ionic OR adj pl fem acc attic epic doric contr OR adj pl fem nom doric aeolic contr OR adj pl fem voc doric aeolic contr OR adj pl masc acc attic epic doric contr OR adj pl masc nom doric aeolic contr OR adj pl masc voc doric aeolic contr OR adj sg fem nom OR adj sg masc nom}
 \item \eintrag{πιστεύσειεν}{ πιστεύω }{trust, put faith in, rely on}{verb 3rd sg aor opt act nu movable}
 \item \eintrag{πλέον}{ πλέος πλέω πλέως πλείων }{full.}{adj sg masc acc OR adj sg neut acc OR adj sg neut nom OR adj sg neut voc OR part sg pres act masc voc OR part sg pres act neut acc OR part sg pres act neut nom OR part sg pres act neut voc OR verb 1st sg imperf ind act homeric ionic unaugmented OR verb 3rd pl imperf ind act homeric ionic unaugmented OR adj sg masc acc ionic OR adj sg neut acc ionic OR adj sg neut nom ionic OR adj sg neut voc ionic OR adj sg fem voc comp OR adj sg masc voc comp OR adj sg neut acc comp OR adj sg neut nom comp OR adj sg neut voc comp}
 \item \eintrag{πλείονες}{ πλείων }{more,}{adj pl fem nom comp OR adj pl fem voc comp OR adj pl masc nom comp OR adj pl masc voc comp}
 \item \eintrag{πλευραῖς}{ πλευρά }{rib,}{noun pl fem dat}
 \item \eintrag{πληθύοντος}{ πληθύω }{to be}{part sg pres act masc gen OR part sg pres act neut gen}
 \item \eintrag{πλησιάσας}{ πλησιάζω }{bring near,}{part pl fut act fem acc doric OR part sg aor act masc nom attic epic ionic OR part sg aor act masc voc attic epic ionic OR part sg fut act fem gen doric OR verb 2nd sg aor ind act homeric ionic unaugmented}
 \item \eintrag{πλὴν}{ πλέω πλήν }{sail, go by sea,}{verb pres inf act epic doric contr OR prep indeclform}
 \item \eintrag{πλῆθος}{ πλῆθος }{great number, multitude,}{noun sg neut voc OR noun sg neut nom OR noun sg neut acc}
 \item \eintrag{ποιεῖν}{ ποιέω }{make}{verb pres inf act attic epic doric contr}
 \item \eintrag{ποιεῖσθαι}{ ποιέω }{make}{verb pres inf mp attic epic contr}
 \item \eintrag{ποιεῖσθε}{ ποιέω }{make}{verb 2nd pl imperf ind mp attic epic contr unaugmented OR verb 2nd pl pres imperat mp attic epic contr OR verb 2nd pl pres ind mp attic epic contr OR verb 2nd pl pres opt mp epic ionic}
 \item \eintrag{ποιησάμενοι}{ ποιέω }{make}{part pl aor mid masc nom OR part pl aor mid masc voc}
 \item \eintrag{ποιούμενοι}{ ποιέω ποιόω }{make}{part pl pres mp masc nom attic epic doric contr OR part pl pres mp masc voc attic epic doric contr OR part pl pres mp masc nom contr OR part pl pres mp masc voc contr}
 \item \eintrag{ποιούμενος}{ ποιέω ποιόω }{make}{part sg pres mp masc nom attic epic doric contr OR part sg pres mp masc nom contr}
 \item \eintrag{πολέμοις}{ πόλεμος πολεμέω πολεμόω }{war,}{noun pl masc dat OR verb 2nd sg pres opt act attic epic doric contr OR verb 2nd sg pres ind act contr OR verb 2nd sg pres opt act contr OR verb 2nd sg pres subj act contr}
 \item \eintrag{πολέμου}{ πόλεμος πολεμέω πολεμόω }{war,}{noun sg masc gen OR verb 2nd sg pres imperat mp attic contr OR verb 3rd sg imperf ind act homeric ionic contr unaugmented OR verb 2nd sg pres imperat mp contr OR verb 2nd sg pres imperat act contr OR verb 2nd sg imperf ind mp homeric ionic contr unaugmented}
 \item \eintrag{πολέμῳ}{ πόλεμος }{war,}{noun sg masc dat}
 \item \eintrag{πολεμήσαντος}{ πολεμέω }{to be at war}{part sg aor act masc gen OR part sg aor act neut gen}
 \item \eintrag{πολεμεῖν}{ πολεμέω }{to be at war}{verb pres inf act attic epic doric contr}
 \item \eintrag{πολιορκίας}{ πολιορκία }{siege of a city,}{noun pl fem acc OR noun sg fem gen attic doric aeolic}
 \item \eintrag{πολλά}{ πολύς }{many,}{adj sg fem nom doric aeolic OR adj sg fem voc doric aeolic OR adj pl neut voc OR adj pl neut nom OR adj dual fem voc OR adj pl neut acc OR adj dual fem nom OR adj dual fem acc}
 \item \eintrag{πολλοὶ}{ πολύς }{many,}{adj pl masc voc OR adj pl masc nom}
 \item \eintrag{πολλοὺς}{ πολύς }{many,}{adj pl masc acc}
 \item \eintrag{πολλοῖς}{ πολύς }{many,}{adj pl masc dat OR adj pl neut dat}
 \item \eintrag{πολλοῦ}{ πολύς }{many,}{adj sg neut gen ionic OR adj sg masc gen ionic}
 \item \eintrag{πολλὰ}{ πολύς }{many,}{adj sg fem nom doric aeolic OR adj sg fem voc doric aeolic OR adj pl neut voc OR adj pl neut nom OR adj dual fem voc OR adj pl neut acc OR adj dual fem nom OR adj dual fem acc}
 \item \eintrag{πολλῆς}{ πολύς }{many,}{adj sg fem gen attic epic ionic}
 \item \eintrag{πολλῶν}{ πολύς }{many,}{adj pl neut gen OR adj pl masc gen OR adj pl fem gen}
 \item \eintrag{πολὺ}{ πολύς }{many,}{adj sg neut voc attic epic indeclform OR adj sg neut nom attic epic indeclform OR adj sg neut acc attic epic indeclform}
 \item \eintrag{πολὺν}{ πολύς }{many,}{adj sg masc acc attic epic indeclform}
 \item \eintrag{πονουμένης}{ πονέομαι πονέω }{be engaged in
            toil, toil, labor, be busy}{part sg pres mp fem gen attic epic contr OR part sg pres mp fem gen attic epic contr}
 \item \eintrag{ποταμοὺς}{ ποταμός }{river, stream,}{noun pl masc acc}
 \item \eintrag{πράξαντας}{ πράσσω }{pass through, pass over,}{part pl aor act masc acc}
 \item \eintrag{πράξειν}{ πράσσω }{pass through, pass over,}{verb fut inf act doric contr}
 \item \eintrag{πρέσβεις}{ πρέσβις πρέσβις πρέσβυς πρεσβεύς }{ambassador,}{noun pl fem voc attic epic contr OR noun pl fem nom attic epic contr OR noun pl fem acc attic OR noun pl fem voc attic epic contr OR noun pl fem nom attic epic contr OR noun pl fem acc attic OR noun pl masc voc attic epic contr OR noun pl masc nom attic epic contr OR noun pl masc voc epic parad form contr OR noun pl masc acc epic contr late OR noun pl masc nom epic parad form contr}
 \item \eintrag{πρέσβεων}{ πρέσβα πρέσβεα πρέσβη πρέσβις πρέσβις πρέσβος πρέσβυς πρεσβεύς }{august, honoured}{noun pl fem gen epic ionic OR noun pl fem gen OR noun pl fem gen epic ionic OR noun pl fem gen OR noun pl fem gen OR noun pl neut gen epic doric ionic aeolic OR noun pl masc gen OR noun pl masc gen epic}
 \item \eintrag{πρεσβείας}{ πρέσβεια πρεσβεία πρεσβεία2 }{keine Übersetzung gefunden}{noun pl fem acc OR noun sg fem gen attic doric aeolic OR noun pl fem acc OR noun sg fem gen attic doric aeolic OR noun pl fem acc OR noun sg fem gen attic doric aeolic}
 \item \eintrag{πρεσβευσάντων}{ πρεσβεύω }{to be the elder}{part pl aor act masc gen OR part pl aor act neut gen OR verb 3rd pl aor imperat act}
 \item \eintrag{πρεσβευτήν}{ πρεσβευτής }{ambassador,}{noun sg masc acc attic epic ionic}
 \item \eintrag{πρεσβεῦσαι}{ πρεσβεύω }{to be the elder}{verb 2nd sg aor imperat mid OR verb 3rd sg aor opt act OR verb aor inf act}
 \item \eintrag{πρεσβυτέρους}{ πρέσβυς }{old man}{adj pl masc acc irreg comp}
 \item \eintrag{πρεσβύτερόν}{ πρέσβυς }{old man}{adj sg masc acc irreg comp OR adj sg neut acc irreg comp OR adj sg neut nom irreg comp OR adj sg neut voc irreg comp}
 \item \eintrag{προαποφήνασθαι}{ προαποφαίνω }{declare}{verb aor inf mid}
 \item \eintrag{προδώσειν}{ προδίδωμι }{give beforehand, pay in advance,}{verb fut inf act doric contr}
 \item \eintrag{προειλήφει}{ προλαμβάνω }{take}{verb 3rd sg plup ind act attic epic contr raw preverb}
 \item \eintrag{προειρημέναις}{ προερέω }{say beforehand,}{part pl perf mp fem dat epic ionic raw preverb}
 \item \eintrag{προηγόρευον}{ προαγορεύω }{tell beforehand,}{verb 1st sg imperf ind act attic epic ionic raw preverb OR verb 3rd pl imperf ind act attic epic ionic raw preverb}
 \item \eintrag{προθυμίαν}{ προθυμία }{readiness, willingness, eagerness,}{noun sg fem acc attic doric aeolic OR noun pl fem gen doric aeolic}
 \item \eintrag{προθυμίας}{ προθυμία }{readiness, willingness, eagerness,}{noun pl fem acc OR noun sg fem gen attic doric aeolic}
 \item \eintrag{προκαλεῖται}{ προκαλέω }{call forth,}{verb 3rd sg fut ind mid attic epic contr OR verb 3rd sg pres ind mp attic epic contr}
 \item \eintrag{προπαθών}{ προπάσχω προπαθής }{suffer first}{part sg aor act masc nom OR adj pl fem gen attic epic doric contr OR adj pl masc gen attic epic doric contr OR adj pl neut gen attic epic doric contr}
 \item \eintrag{προπολεμεῖν}{ προπολεμέω }{make war for}{verb pres inf act attic epic doric contr}
 \item \eintrag{προσέβαλεν}{ προσβάλλω }{strike, dash against}{verb 3rd sg aor ind act nu movable}
 \item \eintrag{προσέθηκε}{ προστίθημι }{put to}{verb 3rd sg aor ind act}
 \item \eintrag{προσέκειτο}{ πρόσκειμαι }{to be placed}{verb 3rd sg imperf ind mp}
 \item \eintrag{προσέπεμπεν}{ προσπέμπω }{send to}{verb 3rd sg imperf ind act nu movable}
 \item \eintrag{προσήκαντο}{ προσίημι }{let come to}{verb 3rd pl aor ind mid}
 \item \eintrag{προσαγαγέσθαι}{ προσάγω }{bring to}{verb aor inf mid redupl}
 \item \eintrag{προσγράφετε}{ προσγράφω }{write besides, add in writing}{verb 2nd pl imperf ind act homeric ionic unaugmented OR verb 2nd pl pres imperat act OR verb 2nd pl pres ind act}
 \item \eintrag{προσελάμβανον}{ προσλαμβάνω }{take}{verb 1st sg imperf ind act n infix OR verb 3rd pl imperf ind act n infix}
 \item \eintrag{προσεξεργάσαιτο}{ προσεξεργάζομαι }{work out, accomplish besides}{verb 3rd sg aor opt mp}
 \item \eintrag{προσεπεῖπε}{ προσεπεῖπον }{say besides}{verb 2nd sg aor imperat act unasp preverb OR verb 3rd sg aor ind act epic ionic}
 \item \eintrag{προσηκόντων}{ προσήκω }{to have come, be at hand, be present}{part pl pres act masc gen OR part pl pres act neut gen OR verb 3rd pl pres imperat act}
 \item \eintrag{προσιέναι}{ πρόσειμι προσίημι }{sum}{verb pres inf act OR verb pres inf act}
 \item \eintrag{προσιούσης}{ πρόσειμι προσίζω }{sum}{part sg pres act fem gen attic epic ionic OR part sg fut act fem gen attic epic contr}
 \item \eintrag{προσοίκου}{ πρόσοικος προσοικέω }{dwelling near to, neighbouring}{adj sg fem gen OR adj sg masc gen OR adj sg neut gen OR verb 2nd sg imperf ind mp attic poetic contr unaugmented OR verb 2nd sg pres imperat mp attic contr}
 \item \eintrag{προτάσεις}{ πρότασις }{putting forward}{noun pl fem acc attic OR noun pl fem nom attic OR noun pl fem voc attic epic contr}
 \item \eintrag{προφάσεως}{ πρόφασις }{motive}{noun sg fem gen attic ionic}
 \item \eintrag{προφέρει}{ προφέρω προφερής }{bring before}{verb 2nd sg pres ind mp OR verb 3rd sg pres ind act OR adj dual fem acc attic epic poetic contr OR adj dual fem nom attic epic poetic contr OR adj dual fem voc attic epic poetic contr OR adj dual masc acc attic epic poetic contr OR adj dual masc nom attic epic poetic contr OR adj dual masc voc attic epic poetic contr OR adj dual neut acc attic epic poetic contr OR adj dual neut nom attic epic poetic contr OR adj dual neut voc attic epic poetic contr OR adj sg dat epic poetic OR adj sg fem dat poetic OR adj sg masc dat poetic OR adj sg neut dat poetic}
 \item \eintrag{προύτεινεν}{ προτείνω }{stretch out before, hold before}{verb 3rd sg aor ind act nu movable raw preverb OR verb 3rd sg imperf ind act nu movable raw preverb}
 \item \eintrag{πρόεδρος}{ πρόεδρος }{one who sits in the first place, president,}{noun sg masc nom}
 \item \eintrag{πρόξενον}{ πρόξενος }{public}{noun sg masc acc}
 \item \eintrag{πρότερον}{ πρότερος }{before, in front}{adj sg neut voc irreg comp OR adj sg neut nom irreg comp OR adj sg neut acc irreg comp OR adj sg masc acc irreg comp}
 \item \eintrag{πρότερος}{ πρότερος }{before, in front}{adj sg masc nom irreg comp}
 \item \eintrag{πρόφασιν}{ πρόφασις προφάω }{motive}{noun sg fem acc OR verb 3rd sg pres subj act epic contr nu movable OR verb 3rd pl pres subj act doric aeolic contr nu movable OR verb 2nd sg pres subj mp epic contr nu movable OR part pl pres act neut dat doric nu movable OR part pl pres act masc dat doric nu movable}
 \item \eintrag{πρώτη}{ Πρωτεύς πρότερος πρῶτος πρωτεύς πρωτός }{Proteus}{noun sg masc acc contr OR adj sg fem voc attic epic ionic irreg superl OR adj sg fem nom attic epic ionic irreg superl OR adj sg fem voc attic epic ionic OR adj sg fem nom attic epic ionic OR noun sg masc acc contr OR noun dual masc voc contr OR noun dual masc nom contr OR noun dual masc acc contr OR adj sg fem nom attic epic ionic OR adj sg fem voc attic epic ionic}
 \item \eintrag{πρώτων}{ πρότερος πρῶτος πρωτός }{before, in front}{adj pl masc gen irreg superl OR adj pl neut gen irreg superl OR adj pl fem gen irreg superl OR adj pl masc gen OR adj pl neut gen OR adj pl fem gen OR adj pl masc gen OR adj pl neut gen OR adj pl fem gen}
 \item \eintrag{πρὸ}{ πρό }{before, forth}{prep indeclform}
 \item \eintrag{πρὸς}{ πρός }{on the side of, in the direction of,}{prep indeclform}
 \item \eintrag{πρῶτοι}{ Πρωτώ πρότερος πρῶτος πρωτός }{keine Übersetzung gefunden}{noun sg fem dat epic poetic OR adj pl masc voc irreg superl OR adj pl masc nom irreg superl OR adj pl masc voc OR adj pl masc nom OR adj pl masc voc OR adj pl masc nom}
 \item \eintrag{πρῶτον}{ πρότερος πρῶτος πρωτός }{before, in front}{adj sg neut voc irreg superl OR adj sg neut nom irreg superl OR adj sg masc acc irreg superl OR adj sg neut acc irreg superl OR adj sg neut nom OR adj sg neut voc OR adj sg masc acc OR adj sg neut acc OR adj sg neut nom OR adj sg neut voc OR adj sg masc acc OR adj sg neut acc}
 \item \eintrag{πρῶτος}{ πρότερος πρῶτος πρωτός }{before, in front}{adj sg masc nom irreg superl OR adj sg masc nom OR adj sg masc nom}
 \item \eintrag{πταῖσμα}{ πταῖσμα }{stumble, trip, false step, mistake}{noun sg neut acc OR noun sg neut nom OR noun sg neut voc}
 \item \eintrag{πυθόμενοι}{ πύθω πυνθάνομαι }{cause to rot}{part pl pres pass masc voc OR part pl pres pass masc nom OR part pl aor mid masc voc OR part pl aor mid masc nom}
 \item \eintrag{πυθόμενος}{ πύθω πυνθάνομαι }{cause to rot}{part sg pres pass masc nom OR part sg aor mid masc nom}
 \item \eintrag{πυνθανόμενος}{ πυνθάνομαι }{learn}{part sg pres mp masc nom n infix}
 \item \eintrag{πω}{ πω πω2 πῶ πῶ2 }{up to this time, yet}{partic enclitic indeclform OR partic enclitic indeclform OR adv doric indeclform OR adv doric indeclform}
 \item \eintrag{πόλεις}{ πόλις πολέω πολύς }{city,}{noun pl fem voc attic epic doric ionic contr OR noun pl fem nom attic epic doric ionic OR noun pl fem acc attic epic doric ionic OR verb 2nd sg pres ind act attic epic doric ionic contr OR verb 2nd sg imperf ind act attic epic contr unaugmented OR adj pl masc nom epic OR adj pl masc acc epic}
 \item \eintrag{πόλεμον}{ πόλεμος }{war,}{noun sg masc acc}
 \item \eintrag{πόλεμος}{ πόλεμος }{war,}{noun sg masc nom}
 \item \eintrag{πόλεσιν}{ πόλις πολύς }{city,}{noun pl fem dat nu movable OR adj pl neut poetic nu movable indeclform OR adj pl masc poetic nu movable indeclform OR adj pl masc dat epic nu movable}
 \item \eintrag{πόλεως}{ πόλις }{city,}{noun sg fem gen attic epic doric ionic}
 \item \eintrag{πόνου}{ πόνος πονέομαι πονέω }{work,}{noun sg masc gen OR verb 2nd sg pres imperat mp attic contr OR verb 2nd sg imperf ind mp attic poetic contr unaugmented OR verb 2nd sg pres imperat mp attic contr OR verb 2nd sg imperf ind mp attic poetic contr unaugmented}
 \item \eintrag{πόσων}{ πόσος ποσός ποσόω }{of what quantity?}{adj pl fem gen OR adj pl masc gen OR adj pl neut gen OR adj pl fem gen OR adj pl masc gen OR adj pl neut gen OR part sg pres act masc nom contr OR part sg pres act masc voc doric aeolic contr OR part sg pres act neut acc doric aeolic contr OR part sg pres act neut nom doric aeolic contr OR part sg pres act neut voc doric aeolic contr OR verb 1st sg imperf ind act doric aeolic poetic contr unaugmented OR verb 3rd pl imperf ind act doric aeolic poetic contr unaugmented OR verb pres inf act doric}
 \item \eintrag{πύλαις}{ Πύλαι πύλη }{keine Übersetzung gefunden}{noun pl fem dat OR noun pl fem dat}
 \item \eintrag{πᾶσαν}{ πάσσω πᾶς πᾶς πᾶϲ }{sprinkle}{verb 3rd pl aor ind act homeric ionic unaugmented OR part sg aor act neut voc OR part sg aor act neut nom OR part sg aor act neut acc OR adj sg fem acc doric indeclform OR adj pl fem gen doric indeclform OR adj sg fem acc doric indeclform OR adj pl fem gen doric indeclform OR adj sg fem acc doric indeclform OR adj pl fem gen doric indeclform}
 \item \eintrag{πᾶσιν}{ πᾶς πᾶς πᾶϲ πᾶσις }{all}{adj pl neut dat attic epic ionic nu movable indeclform OR adj pl masc dat attic epic ionic nu movable indeclform OR adj pl neut dat attic epic ionic nu movable indeclform OR adj pl masc dat attic epic ionic nu movable indeclform OR adj pl neut dat attic epic ionic nu movable indeclform OR adj pl masc dat attic epic ionic nu movable indeclform OR noun sg fem acc}
 \item \eintrag{πῆμα}{ πῆμα }{misery, calamity}{noun sg neut acc OR noun sg neut nom OR noun sg neut voc}
 \item \eintrag{σάλπιγγι}{ σάλπιγξ }{war-trumpet}{noun sg fem dat}
 \item \eintrag{σητει}{ σητάω }{fret}{verb 2nd sg pres imperat act attic epic ionic contr OR verb 2nd sg pres ind mp attic epic doric ionic contr OR verb 3rd sg imperf ind act attic epic ionic contr unaugmented OR verb 3rd sg pres ind act attic epic doric ionic contr}
 \item \eintrag{σιτολογίαν}{ σιτολογία }{collecting of corn, foraging}{noun pl fem gen doric aeolic OR noun sg fem acc attic doric aeolic}
 \item \eintrag{σιωπήν}{ σιωπάω σιωπή }{keep silence}{verb pres inf act doric ionic contr OR verb 3rd pl imperf ind act epic doric aeolic unaugmented OR verb 1st sg imperf ind act homeric ionic unaugmented OR noun sg fem acc attic epic ionic}
 \item \eintrag{σκαιὸς}{ σκαιός }{left, on the left hand}{adj sg masc nom}
 \item \eintrag{σκαφῶν}{ σκάφη σκάφος σκάφος σκαφή2 }{trough, tub, basin}{noun pl fem gen OR noun pl masc gen OR noun pl neut gen attic epic doric contr OR noun pl fem gen}
 \item \eintrag{σμικρολογία}{ μικρολογία }{meanness, stinginess}{noun dual fem acc OR noun dual fem nom OR noun dual fem voc OR noun sg fem nom attic doric aeolic OR noun sg fem voc attic doric aeolic}
 \item \eintrag{σμικρύνασα}{ }{keine Übersetzung gefunden}{Nichts gefunden}
 \item \eintrag{σπονδαὶ}{ σπονδή }{drink-offering}{noun sg fem dat doric aeolic OR noun pl fem voc OR noun pl fem nom}
 \item \eintrag{σπουδῆς}{ σπουδάζω σπουδή }{to be busy, eager}{verb 2nd sg fut ind act doric contr OR noun sg fem gen attic epic ionic}
 \item \eintrag{σταδίου}{ στάδιον στάδιος }{stade}{noun sg neut gen OR adj sg masc gen OR adj sg neut gen}
 \item \eintrag{σταθμούς}{ σταθμός σταθμόω }{standing-place}{noun pl masc acc OR verb 2nd sg pres ind act doric contr OR verb 2nd sg imperf ind act homeric ionic contr unaugmented}
 \item \eintrag{στατῆρας}{ στατήρ }{standard coin}{noun pl masc acc}
 \item \eintrag{στεφάνους}{ στέφανος στεφανόω }{that which surrounds}{noun pl masc acc OR verb 2nd sg imperf ind act homeric ionic contr unaugmented OR verb 2nd sg pres ind act doric contr}
 \item \eintrag{στεφάνων}{ στέφανος στεφάνη στεφανόω }{that which surrounds}{noun pl masc gen OR noun pl fem gen OR part sg pres act masc nom contr OR part sg pres act masc voc doric aeolic contr OR part sg pres act neut acc doric aeolic contr OR part sg pres act neut nom doric aeolic contr OR part sg pres act neut voc doric aeolic contr OR verb 1st sg imperf ind act doric aeolic poetic contr unaugmented OR verb 3rd pl imperf ind act doric aeolic poetic contr unaugmented OR verb pres inf act doric}
 \item \eintrag{στρατηγοί}{ στρατηγέω στρατηγός }{to be general}{verb 3rd sg pres opt act attic epic doric contr OR noun pl masc nom OR noun pl masc voc}
 \item \eintrag{στρατηγοῖς}{ στρατηγέω στρατηγός }{to be general}{verb 2nd sg pres opt act attic epic doric contr OR noun pl masc dat}
 \item \eintrag{στρατηγοῦ}{ στρατηγέω στρατηγός }{to be general}{verb 2nd sg imperf ind mp attic poetic contr unaugmented OR verb 2nd sg pres imperat mp attic contr OR noun sg masc gen}
 \item \eintrag{στρατηγοῦντος}{ στρατηγέω }{to be general}{part sg pres act neut gen attic epic doric contr OR part sg pres act masc gen attic epic doric contr}
 \item \eintrag{στρατηγός}{ στρατηγός }{leader}{noun sg masc nom}
 \item \eintrag{στρατηγῷ}{ στρατηγός }{leader}{noun sg masc dat}
 \item \eintrag{στρατιὰ}{ Στρατίη στράτιος στρατία στρατιά }{keine Übersetzung gefunden}{noun sg fem voc attic doric aeolic geog name OR noun sg fem nom attic doric aeolic geog name OR noun dual fem nom geog name OR noun dual fem voc geog name OR noun dual fem acc geog name OR adj sg fem voc attic doric aeolic OR adj sg fem nom attic doric aeolic OR adj pl neut nom OR adj pl neut voc OR adj pl neut acc OR adj dual fem nom OR adj dual fem voc OR adj dual fem acc OR noun sg fem voc attic doric aeolic OR noun sg fem nom attic doric aeolic OR noun dual fem voc OR noun dual fem nom OR noun dual fem acc OR noun sg fem voc attic doric aeolic OR noun sg fem nom attic doric aeolic OR noun dual fem voc OR noun dual fem nom OR noun dual fem acc}
 \item \eintrag{στρατιὰν}{ στράτιος στρατία στρατιά }{of an army}{adj pl fem gen doric OR adj pl masc gen doric OR adj sg fem acc attic doric aeolic OR noun pl fem gen doric aeolic OR noun sg fem acc attic doric aeolic OR noun pl fem gen doric aeolic OR noun sg fem acc attic doric aeolic}
 \item \eintrag{στρατιᾶς}{ στράτιος στρατία στρατιά }{of an army}{adj pl fem acc OR adj sg fem gen attic doric aeolic OR noun pl fem acc OR noun sg fem gen attic doric aeolic OR noun pl fem acc OR noun sg fem gen attic doric aeolic}
 \item \eintrag{στρατοπέδῳ}{ στρατόπεδον }{camp, encampment}{noun sg neut dat}
 \item \eintrag{στρατοῖς}{ στρατός στρατόω }{army, host.}{noun pl masc dat OR verb 2nd sg pres ind act contr OR verb 2nd sg pres opt act contr OR verb 2nd sg pres subj act contr}
 \item \eintrag{στρατοῦ}{ στρατός στρατόω }{army, host.}{noun sg masc gen OR verb 2nd sg imperf ind mp homeric ionic contr unaugmented OR verb 2nd sg pres imperat act contr OR verb 2nd sg pres imperat mp contr OR verb 3rd sg imperf ind act homeric ionic contr unaugmented}
 \item \eintrag{στρατόν}{ στρατός }{army, host.}{noun sg masc acc}
 \item \eintrag{στρατόπεδον}{ στρατόπεδον }{camp, encampment}{noun sg neut acc OR noun sg neut nom OR noun sg neut voc}
 \item \eintrag{στρατὸν}{ στρατός }{army, host.}{noun sg masc acc}
 \item \eintrag{στρατῷ}{ στρατός }{army, host.}{noun sg masc dat}
 \item \eintrag{στόλον}{ στόλος }{equipment}{noun sg masc acc}
 \item \eintrag{στόλῳ}{ στόλος }{equipment}{noun sg masc dat}
 \item \eintrag{συγγιγνώσκειν}{ συγγιγνώσκω }{think with, agree with}{verb pres inf act attic epic contr pres redupl}
 \item \eintrag{συγγνώμης}{ συγγνώμη }{fellow-feeling, forbearance, lenient judgement, allowance}{noun sg fem gen attic epic ionic}
 \item \eintrag{συγκειμένων}{ σύγκειμαι }{lie together}{part pl perf mp fem gen OR part pl perf mp masc gen OR part pl perf mp neut gen OR part pl pres mp fem gen OR part pl pres mp masc gen OR part pl pres mp neut gen}
 \item \eintrag{συμβάντος}{ συμβαίνω }{stand with the feet together}{part sg aor act masc gen OR part sg aor act neut gen}
 \item \eintrag{συμβάσεις}{ σύμβασις συμβαίνω }{bringing one foot up to the other}{noun pl fem voc attic epic ionic contr OR noun pl fem nom attic epic ionic contr OR noun pl fem acc attic ionic OR verb 2nd sg aor subj act epic doric short subj causal}
 \item \eintrag{συμβάσεων}{ σύμβασις συμβαίνω }{bringing one foot up to the other}{noun pl fem gen OR part pl aor act fem gen epic ionic}
 \item \eintrag{συμβαίνοντες}{ συμβαίνω }{stand with the feet together}{part pl pres act masc nom OR part pl pres act masc voc}
 \item \eintrag{συμβούλους}{ σύμβουλος }{adviser, counsellor}{noun pl masc acc}
 \item \eintrag{συμμάχους}{ σύμμαχος }{fighting along with, leagued}{adj pl fem acc OR adj pl masc acc}
 \item \eintrag{συμμάχων}{ σύμμαχος συμμαχέω }{fighting along with, leagued}{adj pl neut gen OR adj pl masc gen OR adj pl fem gen OR part sg pres act masc nom attic epic doric comp only contr}
 \item \eintrag{συμμαχήσας}{ συμμαχέω }{to be an ally, to be in alliance}{part sg aor act masc nom attic epic ionic comp only OR part sg aor act masc voc attic epic ionic comp only OR verb 2nd sg aor ind act homeric ionic comp only unaugmented}
 \item \eintrag{συμμαχήσειν}{ συμμαχέω }{to be an ally, to be in alliance}{verb fut inf act doric comp only contr}
 \item \eintrag{συμμαχίαν}{ συμμαχία }{alliance, offensive and defensive}{noun pl fem gen doric aeolic OR noun sg fem acc attic doric aeolic}
 \item \eintrag{συμμαχίας}{ συμμαχία }{alliance, offensive and defensive}{noun sg fem gen attic doric aeolic OR noun pl fem acc}
 \item \eintrag{συμμαχῶν}{ σύμμαχος συμμαχέω }{fighting along with, leagued}{adj pl fem gen OR adj pl masc gen OR adj pl neut gen OR part sg pres act masc nom attic epic doric comp only contr}
 \item \eintrag{συμπέσοι}{ συμπίτνω }{fall}{verb 3rd sg aor opt act}
 \item \eintrag{συμπράξειε}{ συμπράσσω }{join}{verb 3rd sg aor opt act}
 \item \eintrag{συμφέρει}{ συμφέρω }{bring together, gather, collect}{verb 2nd sg pres ind mp OR verb 3rd sg pres ind act}
 \item \eintrag{συμφέρειν}{ συμφέρω }{bring together, gather, collect}{verb pres inf act attic epic contr}
 \item \eintrag{συμφορὰν}{ συμφορά συμφοράζω }{bringing together, collecting}{noun pl fem gen doric aeolic OR noun sg fem acc attic doric aeolic OR part sg fut act masc nom doric aeolic contr OR part sg fut act masc voc doric aeolic contr OR part sg fut act neut acc doric aeolic contr OR part sg fut act neut nom doric aeolic contr OR part sg fut act neut voc doric aeolic contr OR verb fut inf act}
 \item \eintrag{συνέβη}{ συμβαίνω }{stand with the feet together}{verb 3rd sg aor ind act}
 \item \eintrag{συνέβησαν}{ συμβαίνω }{stand with the feet together}{verb 3rd pl aor ind act}
 \item \eintrag{συνέθεντο}{ συντίθημι }{place}{verb 3rd pl aor ind mid}
 \item \eintrag{συνέπραξε}{ συμπράσσω }{join}{verb 3rd sg aor ind act}
 \item \eintrag{συνίστορας}{ συνίστωρ }{knowing along with}{noun pl fem acc OR noun pl masc acc}
 \item \eintrag{συναγαγόντι}{ συνάγω }{bring together, gather together}{part sg aor act masc dat redupl OR part sg aor act neut dat redupl}
 \item \eintrag{συναγαγὼν}{ συνάγω }{bring together, gather together}{part sg aor act masc nom redupl}
 \item \eintrag{συναγόντων}{ συνάγω }{bring together, gather together}{part pl pres act masc gen OR part pl pres act neut gen OR verb 3rd pl pres imperat act}
 \item \eintrag{συνελθεῖν}{ συνέρχομαι }{ibo}{verb aor inf act attic epic doric contr}
 \item \eintrag{συνεμάχει}{ συμμαχέω }{to be an ally, to be in alliance}{verb 3rd sg imperf ind act attic epic comp only contr}
 \item \eintrag{συνεμάχησε}{ συμμαχέω }{to be an ally, to be in alliance}{verb 3rd sg aor ind act comp only}
 \item \eintrag{συνεμάχησεν}{ συμμαχέω }{to be an ally, to be in alliance}{verb 3rd sg aor ind act comp only nu movable}
 \item \eintrag{συνετίθεσθε}{ συντίθημι }{place}{verb 2nd pl imperf ind mp}
 \item \eintrag{συνεχώρει}{ συγχωρέω }{come together, meet}{verb 3rd sg imperf ind act attic epic contr}
 \item \eintrag{συνηδόμενοι}{ συνήδομαι }{rejoice together}{part pl pres mp masc nom OR part pl pres mp masc voc}
 \item \eintrag{συνθέμενος}{ συντίθημι }{place}{part sg aor mid masc nom}
 \item \eintrag{συνθήκας}{ συνθήκη συντίθημι }{compounding}{noun pl fem acc OR noun sg fem gen doric aeolic OR verb 2nd sg aor ind act homeric ionic unaugmented}
 \item \eintrag{συνθήκῃ}{ συνθήκη }{compounding}{noun sg fem dat attic epic ionic}
 \item \eintrag{συνθηκῶν}{ συνθήκη }{compounding}{noun pl fem gen}
 \item \eintrag{συνθῆκαι}{ συνθήκη }{compounding}{noun sg fem dat doric aeolic OR noun pl fem nom OR noun pl fem voc}
 \item \eintrag{συνοίσει}{ συμφέρω }{bring together, gather, collect}{verb 2nd sg fut ind mid doric contr OR verb 3rd sg fut ind act doric contr}
 \item \eintrag{συντυχίας}{ συντυχία }{occurrence, happening, incident}{noun pl fem acc OR noun sg fem gen attic doric aeolic}
 \item \eintrag{συνῆλθεν}{ συνέρχομαι }{ibo}{verb 3rd sg aor ind act nu movable late}
 \item \eintrag{συνῆλθον}{ συνέρχομαι }{ibo}{verb 1st sg aor ind act OR verb 3rd pl aor ind act}
 \item \eintrag{συστρατεύσειν}{ συστρατεύω }{join}{verb fut inf act doric contr}
 \item \eintrag{σφάλλοιτο}{ σφάλλω }{make to fall, overthrow}{verb 3rd sg pres opt mp}
 \item \eintrag{σφέτερα}{ σφέτερος }{their own, their}{adj dual fem acc OR adj dual fem nom OR adj dual fem voc OR adj pl neut acc OR adj pl neut nom OR adj pl neut voc OR adj sg fem nom attic doric aeolic OR adj sg fem voc attic doric aeolic}
 \item \eintrag{σφίσιν}{ σφεῖς }{Rendic.Pont. Accad.Rom. di Arch.}{pron pl fem dat enclitic nu movable indeclform OR pron pl masc dat enclitic nu movable indeclform}
 \item \eintrag{σφᾶς}{ σφάζω σφεῖς σφός }{slay, slaughter}{verb 2nd sg fut ind act doric contr OR pron pl fem acc enclitic indeclform OR pron pl masc acc enclitic indeclform OR adj pl fem acc poetic OR adj sg fem gen doric aeolic poetic}
 \item \eintrag{σωφροσύνην}{ σωφρόσυνος σωφροσύνη }{keine Übersetzung gefunden}{adj sg fem acc attic epic ionic OR noun sg fem acc attic epic doric ionic}
 \item \eintrag{σωφρόνως}{ σώφρων }{of sound mind}{adv}
 \item \eintrag{σύγκλητον}{ σύγκλητος συγκλάω συγκλάζω }{called together, summoned}{adj sg fem acc OR adj sg masc acc OR adj sg neut acc OR adj sg neut nom OR adj sg neut voc OR verb 2nd dual imperf ind act homeric ionic unaugmented OR verb 2nd dual pres imperat act epic doric ionic aeolic contr OR verb 2nd dual pres ind act doric contr OR verb 2nd dual pres subj act doric contr OR verb 3rd dual pres ind act doric contr OR verb 3rd dual pres subj act doric contr OR verb 2nd dual fut ind act doric contr OR verb 3rd dual fut ind act doric contr}
 \item \eintrag{σύγκλητος}{ σύγκλητος }{called together, summoned}{adj sg fem nom OR adj sg masc nom}
 \item \eintrag{σύλλογον}{ σύλλογος }{assembly, concourse, meeting}{noun sg masc acc}
 \item \eintrag{σύμπαν}{ σύμπας }{all together, all at once}{adj sg neut acc OR adj sg neut nom OR adj sg neut voc}
 \item \eintrag{σύνθοιτο}{ συνθέω συντίθημι }{run together with}{verb 3rd sg pres opt mid attic epic doric contr OR verb 3rd sg pres opt mp attic epic doric contr OR verb 3rd sg aor opt mid OR verb 3rd sg attic contr}
 \item \eintrag{σώφρονα}{ σώφρων }{of sound mind}{adj pl neut acc OR adj pl neut nom OR adj pl neut voc OR adj sg fem acc OR adj sg masc acc}
 \item \eintrag{σώφρονας}{ σώφρων }{of sound mind}{adj pl fem acc OR adj pl masc acc}
 \item \eintrag{σὺν}{ σύν ὗς ὗς }{with.}{prep indeclform OR noun sg masc acc indeclform OR noun sg fem acc indeclform OR noun sg masc acc indeclform OR noun sg fem acc indeclform}
 \item \eintrag{σῶμα}{ σῶμα }{body}{noun sg neut voc OR noun sg neut acc OR noun sg neut nom}
 \item \eintrag{σῶφρον}{ σώφρων }{of sound mind}{adj sg fem voc OR adj sg masc voc OR adj sg neut acc OR adj sg neut nom OR adj sg neut voc}
 \item \eintrag{τʼ}{ }{keine Übersetzung gefunden}{Nichts gefunden}
 \item \eintrag{τάδε}{ ὅδε }{this,}{pron pl neut nom indeclform OR pron dual fem voc indeclform OR pron pl neut acc indeclform OR pron dual fem nom indeclform OR pron dual fem acc indeclform}
 \item \eintrag{τάλαντα}{ τάλαντον ταλαντάω }{balance}{noun pl neut voc OR noun pl neut nom OR noun pl neut acc OR verb 3rd sg imperf ind act homeric ionic contr unaugmented OR verb 2nd sg pres imperat act contr OR verb 1st sg pres subj act doric aeolic contr OR verb 1st sg pres ind act doric aeolic contr}
 \item \eintrag{τάφων}{ Τάφος τάφος τάφος ταφή τέθηπα }{Taphos}{noun pl fem gen geog name OR noun pl masc gen OR noun pl neut gen attic epic doric contr OR noun pl fem gen OR part sg aor act masc nom}
 \item \eintrag{τάχα}{ τάχα τάχος ταχύς }{quickly, presently, forthwith}{adv indeclform OR noun pl neut nom doric aeolic contr OR noun pl neut voc doric aeolic contr OR noun pl neut acc doric aeolic contr OR adj pl neut voc doric aeolic contr OR adj pl neut nom doric aeolic contr OR adj pl neut acc doric aeolic contr}
 \item \eintrag{τέκνοις}{ τέκνον τεκνόω }{child}{noun pl neut dat OR verb 2nd sg pres ind act contr OR verb 2nd sg pres opt act contr OR verb 2nd sg pres subj act contr}
 \item \eintrag{τέλεσι}{ τέλεσις τέλος }{event, fulfilment,}{noun sg fem dat epic doric ionic aeolic contr OR noun sg fem voc OR noun pl neut dat}
 \item \eintrag{τέλος}{ τέλος }{coming to pass, performance, consummation}{noun sg neut nom OR noun sg neut voc OR noun sg neut acc}
 \item \eintrag{τέσσαρες}{ τέσσαρες }{four}{noun pl masc voc OR noun pl masc nom OR noun pl fem voc OR noun pl fem nom}
 \item \eintrag{τέταρτος}{ τέταρτος }{fourth}{adj sg masc nom}
 \item \eintrag{τήν}{ ὁ }{the following}{article sg fem acc attic epic ionic indeclform}
 \item \eintrag{τήνδε}{ ὅδε }{this,}{pron sg fem acc attic homeric ionic indeclform}
 \item \eintrag{τί}{ τίς τις τις }{who? which?}{irreg sg neut nom indeclform OR irreg sg neut voc indeclform OR irreg sg neut acc indeclform OR pron sg neut voc enclitic indeclform OR pron sg neut nom enclitic indeclform OR pron sg neut acc enclitic indeclform OR pron sg neut voc enclitic indeclform OR pron sg neut nom enclitic indeclform OR pron sg neut acc enclitic indeclform}
 \item \eintrag{τίς}{ τίς τις τις }{who? which?}{irreg sg masc nom indeclform OR irreg sg fem nom indeclform OR pron sg masc nom enclitic indeclform OR pron sg fem nom enclitic indeclform OR pron sg masc nom enclitic indeclform OR pron sg fem nom enclitic indeclform}
 \item \eintrag{ταινίας}{ ταινία ταινιάζω }{band, fillet}{noun pl fem acc OR noun sg fem gen attic doric aeolic OR verb 2nd sg fut ind act doric contr}
 \item \eintrag{ταλάντοις}{ τάλαντον ταλαντάω }{balance}{noun pl neut dat OR verb 2nd sg pres opt act attic epic doric ionic contr}
 \item \eintrag{ταχέως}{ ταχέως ταχύς }{quickly,
            speedily}{adv indeclform OR adv}
 \item \eintrag{ταχὺς}{ ταχύς }{swift, fleet}{adj sg masc nom}
 \item \eintrag{ταύτης}{ οὗτος }{this}{adj sg fem gen attic epic ionic indeclform}
 \item \eintrag{ταῖς}{ ὁ }{the following}{article pl fem dat indeclform}
 \item \eintrag{ταῦθʼ}{ }{keine Übersetzung gefunden}{Nichts gefunden}
 \item \eintrag{ταῦτʼ}{ }{keine Übersetzung gefunden}{Nichts gefunden}
 \item \eintrag{ταῦτα}{ οὗτος }{this}{adj pl neut voc indeclform OR adj pl neut acc indeclform OR adj pl neut nom indeclform OR adj dual fem acc indeclform}
 \item \eintrag{τε}{ σύ τε }{thou}{pron 2nd sg acc doric indeclform OR partic enclitic indeclform}
 \item \eintrag{τειχίον}{ τειχέω τειχίον }{build walls}{part sg pres act masc voc doric OR part sg pres act neut acc doric OR part sg pres act neut nom doric OR part sg pres act neut voc doric OR verb 1st sg imperf ind act doric poetic unaugmented OR verb 3rd pl imperf ind act doric poetic unaugmented OR noun sg neut acc OR noun sg neut nom OR noun sg neut voc}
 \item \eintrag{τεσσάρων}{ τέσσαρες }{four}{noun pl fem gen OR noun pl masc gen OR noun pl neut gen}
 \item \eintrag{τετραμήνους}{ τετράμηνος }{of four months, lasting four months}{adj pl fem acc OR adj pl masc acc}
 \item \eintrag{τετρῦσθαι}{ τρύω τρύζω }{Erster Bericht}{verb perf inf mp OR verb perf inf mp redupl}
 \item \eintrag{τι}{ τίς τις τις }{who? which?}{irreg sg neut voc indeclform OR irreg sg neut nom indeclform OR irreg sg neut acc indeclform OR pron sg neut voc enclitic indeclform OR pron sg neut nom enclitic indeclform OR pron sg neut acc enclitic indeclform OR pron sg neut voc enclitic indeclform OR pron sg neut nom enclitic indeclform OR pron sg neut acc enclitic indeclform}
 \item \eintrag{τιμὴν}{ τιμάω τιμέω τιμή }{honour, revere, reverence}{verb pres inf act doric ionic contr OR verb 3rd pl imperf ind act epic doric aeolic unaugmented OR verb 1st sg imperf ind act homeric ionic unaugmented OR verb pres inf act epic doric contr OR noun sg fem acc attic epic ionic}
 \item \eintrag{τινα}{ τίς τις τις }{who? which?}{irreg pl neut acc indeclform OR irreg pl neut nom indeclform OR irreg pl neut voc indeclform OR irreg sg fem acc indeclform OR irreg sg masc acc indeclform OR pron pl neut acc enclitic indeclform OR pron pl neut nom enclitic indeclform OR pron pl neut voc enclitic indeclform OR pron sg fem acc enclitic indeclform OR pron sg masc acc enclitic indeclform OR pron pl neut acc enclitic indeclform OR pron pl neut nom enclitic indeclform OR pron pl neut voc enclitic indeclform OR pron sg fem acc enclitic indeclform OR pron sg masc acc enclitic indeclform}
 \item \eintrag{τινας}{ τίς τις τις }{who? which?}{irreg pl fem acc indeclform OR irreg pl masc acc indeclform OR pron pl fem acc enclitic indeclform OR pron pl masc acc enclitic indeclform OR pron pl fem acc enclitic indeclform OR pron pl masc acc enclitic indeclform}
 \item \eintrag{τινες}{ τίνω τίς τις τις }{pay a price}{verb 2nd sg pres ind act doric n infix OR verb 2nd sg imperf ind act homeric ionic unaugmented n infix OR irreg pl masc voc indeclform OR irreg pl fem voc indeclform OR irreg pl masc nom indeclform OR irreg pl fem nom indeclform OR pron pl masc voc enclitic indeclform OR pron pl masc nom enclitic indeclform OR pron pl fem voc enclitic indeclform OR pron pl fem nom enclitic indeclform OR pron pl masc voc enclitic indeclform OR pron pl masc nom enclitic indeclform OR pron pl fem voc enclitic indeclform OR pron pl fem nom enclitic indeclform}
 \item \eintrag{τινὰ}{ τίς τις τις }{who? which?}{irreg pl neut acc indeclform OR irreg pl neut nom indeclform OR irreg pl neut voc indeclform OR irreg sg fem acc indeclform OR irreg sg masc acc indeclform OR pron pl neut acc enclitic indeclform OR pron pl neut nom enclitic indeclform OR pron pl neut voc enclitic indeclform OR pron sg fem acc enclitic indeclform OR pron sg masc acc enclitic indeclform OR pron pl neut acc enclitic indeclform OR pron pl neut nom enclitic indeclform OR pron pl neut voc enclitic indeclform OR pron sg fem acc enclitic indeclform OR pron sg masc acc enclitic indeclform}
 \item \eintrag{τινὶ}{ τίς τις τις }{who? which?}{irreg sg dat indeclform OR pron sg dat enclitic indeclform OR pron sg dat enclitic indeclform}
 \item \eintrag{τις}{ τίς τις τις }{who? which?}{irreg sg fem nom indeclform OR irreg sg masc nom indeclform OR pron sg fem nom enclitic indeclform OR pron sg masc nom enclitic indeclform OR pron sg fem nom enclitic indeclform OR pron sg masc nom enclitic indeclform}
 \item \eintrag{τοιαῦτα}{ τοιοῦτος }{such as this,}{adj sg fem voc doric aeolic OR adj sg fem nom doric aeolic OR adj pl neut voc OR adj pl neut nom OR adj pl neut acc OR adj dual fem voc OR adj dual fem acc OR adj dual fem nom}
 \item \eintrag{τοιοῦτος}{ τοιοῦτος }{such as this,}{adj sg masc nom}
 \item \eintrag{τοιῷδε}{ τοιόσδε }{such as this,}{pron sg masc dat indeclform OR pron sg neut dat indeclform}
 \item \eintrag{τοσήνδε}{ τοσόσδε }{sufficient}{pron sg fem acc attic ionic indeclform}
 \item \eintrag{τοσαύτης}{ τοσοῦτος }{so large, so tall,}{adj sg fem gen attic epic ionic}
 \item \eintrag{τοσοῦδε}{ τοσόσδε }{sufficient}{pron sg masc gen indeclform OR pron sg neut gen indeclform}
 \item \eintrag{τοσῆσδε}{ τοσόσδε }{sufficient}{pron sg fem gen attic ionic indeclform}
 \item \eintrag{τοσῶνδε}{ τοσόσδε }{sufficient}{pron pl fem gen indeclform}
 \item \eintrag{τούτοις}{ οὗτος }{this}{adj pl neut dat indeclform OR adj pl masc dat indeclform}
 \item \eintrag{τούτους}{ οὗτος }{this}{adj pl masc acc indeclform}
 \item \eintrag{τούτων}{ οὗτος }{this}{adj pl neut gen indeclform OR adj pl masc gen indeclform OR adj pl fem gen indeclform}
 \item \eintrag{τούτῳ}{ οὗτος }{this}{adj sg neut dat indeclform OR adj sg masc dat indeclform OR adj dual neut voc indeclform iota intens OR adj dual neut nom indeclform iota intens OR adj dual neut acc indeclform iota intens}
 \item \eintrag{τοὺς}{ ὁ }{the following}{article pl masc acc indeclform}
 \item \eintrag{τοῖς}{ ὁ }{the following}{article pl neut dat indeclform OR article pl masc dat indeclform}
 \item \eintrag{τοῖσδε}{ ὅδε }{this,}{pron pl neut dat indeclform OR pron pl masc dat indeclform}
 \item \eintrag{τοῦ}{ ὁ τίς τις τις }{the following}{article sg neut gen indeclform OR article sg masc gen indeclform OR irreg sg gen attic indeclform OR pron sg gen attic enclitic indeclform OR pron sg gen attic enclitic indeclform}
 \item \eintrag{τοῦδε}{ ὅδε }{this,}{pron sg masc gen indeclform OR pron sg neut gen indeclform}
 \item \eintrag{τοῦθʼ}{ }{keine Übersetzung gefunden}{Nichts gefunden}
 \item \eintrag{τοῦτʼ}{ }{keine Übersetzung gefunden}{Nichts gefunden}
 \item \eintrag{τοῦτο}{ οὗτος }{this}{adj sg neut voc indeclform OR adj sg neut nom indeclform OR adj sg neut acc indeclform}
 \item \eintrag{τρίτην}{ τρίτος τριτάω }{third,}{adj sg fem acc attic epic ionic OR verb 1st sg imperf ind act homeric ionic unaugmented OR verb 3rd pl imperf ind act epic doric aeolic unaugmented OR verb pres inf act doric ionic contr}
 \item \eintrag{τραπέντων}{ τέρπω τρέπω }{delight, gladden, cheer}{part pl aor pass masc gen short subj OR part pl aor pass neut gen short subj OR verb 3rd pl aor imperat pass short subj OR part pl aor pass masc gen OR part pl aor pass neut gen OR verb 3rd pl aor imperat pass}
 \item \eintrag{τρεῖς}{ τρέω τρεῖς }{flee from fear, flee away}{verb 2nd sg pres ind act attic epic doric ionic contr OR verb 2nd sg imperf ind act attic epic contr unaugmented OR numeral pl masc voc indeclform OR numeral pl masc nom indeclform OR numeral pl masc acc indeclform OR numeral pl fem voc indeclform OR numeral pl fem nom indeclform OR numeral pl fem acc indeclform}
 \item \eintrag{τριάκοντα}{ τριάκοντα τριακοντάς }{thirty,}{numeral indeclform OR noun sg fem voc}
 \item \eintrag{τριήρης}{ τριήρης }{a trireme,}{noun pl fem acc attic epic doric ionic contr OR noun pl fem nom doric ionic aeolic contr OR noun pl fem voc doric ionic aeolic contr OR noun sg fem nom ionic}
 \item \eintrag{τριακοσίοις}{ τριακόσιοι }{three hundred,}{adj pl masc dat OR adj pl neut dat}
 \item \eintrag{τρισὶν}{ τρεῖς }{three}{numeral pl fem dat nu movable indeclform OR numeral pl masc dat nu movable indeclform OR numeral pl neut dat nu movable indeclform}
 \item \eintrag{τριῶν}{ τρέω τρεῖς τρία τρίζω τριάξω τριάζω }{flee from fear, flee away}{part sg pres act masc nom doric OR numeral pl neut gen indeclform OR numeral pl masc gen indeclform OR numeral pl fem gen indeclform OR noun pl neut gen OR part sg fut act masc nom attic epic doric contr OR part sg fut act neut nom contr OR part sg fut act neut voc contr OR part sg fut act neut acc contr OR part sg fut act masc voc contr OR part sg fut act masc nom attic epic ionic contr OR part sg fut act neut voc contr OR part sg fut act neut acc contr OR part sg fut act neut nom contr OR part sg fut act masc voc contr OR part sg fut act masc nom attic epic ionic contr}
 \item \eintrag{τροφάς}{ τροφεύς τροφή }{one who brings up, foster-father,}{noun pl masc acc contr OR noun pl fem acc OR noun sg fem gen doric aeolic}
 \item \eintrag{τροφαῖς}{ τροφή }{nourishment, food,}{noun pl fem dat}
 \item \eintrag{τρυφῆς}{ θρύπτω τρυφάω τρυφή }{break in pieces, break small}{verb 2nd sg aor ind pass homeric ionic unaugmented OR verb 2nd sg pres ind act doric contr OR verb 2nd sg imperf ind act homeric ionic unaugmented OR noun sg fem gen attic epic ionic}
 \item \eintrag{τυράννου}{ τύραννος τυραννεύω }{an absolute ruler,}{noun sg masc gen OR verb 2nd sg pres imperat mp attic contr OR verb 2nd sg imperf ind mp attic poetic contr unaugmented}
 \item \eintrag{τωθάζων}{ τωθάζω }{mock, jeer at, flout,}{part sg pres act masc nom}
 \item \eintrag{τό}{ ὁ }{the following}{article sg neut nom indeclform OR article sg neut voc indeclform OR article sg neut acc indeclform}
 \item \eintrag{τότε}{ τότε τοτέ τοτέ2 }{at that time, then,}{adv indeclform OR adv indeclform OR adv indeclform}
 \item \eintrag{τύχης}{ τύχη }{act}{noun sg fem gen attic epic ionic}
 \item \eintrag{τὰ}{ ὁ τίς }{the following}{article pl neut voc indeclform OR article pl neut acc indeclform OR article pl neut nom indeclform OR article dual fem voc indeclform OR article dual fem nom indeclform OR article dual fem acc indeclform OR irreg pl neut voc indeclform OR irreg pl neut nom indeclform OR irreg pl neut acc indeclform}
 \item \eintrag{τὰς}{ ὁ }{the following}{article sg fem gen doric aeolic indeclform OR article pl fem acc indeclform}
 \item \eintrag{τὴν}{ ὁ }{the following}{article sg fem acc attic epic ionic indeclform}
 \item \eintrag{τὸ}{ ὁ }{the following}{article sg neut nom indeclform OR article sg neut voc indeclform OR article sg neut acc indeclform}
 \item \eintrag{τὸν}{ ὁ }{the following}{article sg masc acc indeclform}
 \item \eintrag{τῆς}{ ὁ }{the following}{article sg fem gen attic epic ionic indeclform}
 \item \eintrag{τῆσδε}{ ὅδε }{this,}{pron sg fem gen attic homeric ionic indeclform}
 \item \eintrag{τῇ}{ ὁ τῇ }{the following}{article sg fem dat attic epic ionic indeclform OR adv indeclform}
 \item \eintrag{τῳ}{ ὁ τίς τις τις τῷ }{the following}{article sg masc dat indeclform OR article sg neut dat indeclform OR irreg sg dat attic indeclform OR pron sg dat attic epic enclitic indeclform OR pron sg dat attic epic enclitic indeclform OR adv indeclform}
 \item \eintrag{τῶν}{ ὁ }{the following}{article pl neut gen indeclform OR article pl masc gen indeclform OR article pl fem gen indeclform}
 \item \eintrag{τῶνδε}{ ὅδε }{this,}{pron pl neut gen indeclform OR pron pl masc gen indeclform OR pron pl fem gen indeclform}
 \item \eintrag{τῷ}{ ὁ τίς τις τις τῷ }{the following}{article sg neut dat indeclform OR article sg masc dat indeclform OR irreg sg dat attic indeclform OR pron sg dat attic epic enclitic indeclform OR pron sg dat attic epic enclitic indeclform OR adv indeclform}
 \item \eintrag{τῷδε}{ ὅδε }{this,}{pron sg masc dat indeclform OR pron sg neut dat indeclform}
 \item \eintrag{υἱὸν}{ υἱός }{huihus}{noun sg masc acc}
 \item \eintrag{φέρειν}{ φέρω }{fero,}{verb pres inf act attic epic contr}
 \item \eintrag{φέροντες}{ φέρω }{fero,}{part pl pres act masc voc OR part pl pres act masc nom}
 \item \eintrag{φέρουσι}{ φέρω }{fero,}{part pl pres act neut dat attic epic doric ionic OR verb 3rd pl pres ind act attic epic doric ionic OR part pl pres act masc dat attic epic doric ionic}
 \item \eintrag{φίλοις}{ φίλος φίλος φιλέω φιλόω }{loved, beloved, dear}{adj pl neut dat OR adj pl masc dat OR adj pl neut dat OR adj pl masc dat OR verb 2nd sg pres opt act attic epic doric contr OR verb 2nd sg pres subj act contr OR verb 2nd sg pres opt act contr OR verb 2nd sg pres ind act contr}
 \item \eintrag{φίλον}{ φίλος φίλος φιλέω }{loved, beloved, dear}{adj sg neut voc OR adj sg neut nom OR adj sg neut acc OR adj sg masc acc OR adj sg neut voc OR adj sg neut nom OR adj sg neut acc OR adj sg masc acc OR verb 2nd sg aor imperat act epic}
 \item \eintrag{φίλου}{ φίλος φίλος φιλέω φιλόω }{loved, beloved, dear}{adj sg neut gen OR adj sg masc gen OR adj sg neut gen OR adj sg masc gen OR verb 2nd sg pres imperat mp attic contr OR verb 2nd sg pres imperat mp contr OR verb 3rd sg imperf ind act homeric ionic contr unaugmented OR verb 2nd sg pres imperat act contr OR verb 2nd sg imperf ind mp homeric ionic contr unaugmented}
 \item \eintrag{φίλους}{ φίλος φίλος φῖλος φιλόω }{loved, beloved, dear}{adj pl masc acc OR adj pl masc acc OR noun sg neut gen attic epic doric contr OR verb 2nd sg pres ind act doric contr OR verb 2nd sg imperf ind act homeric ionic contr unaugmented}
 \item \eintrag{φίλων}{ φίλος φίλος φίλων φῖλος φιλέω φιλόω }{loved, beloved, dear}{adj pl neut gen OR adj pl masc gen OR adj pl fem gen OR adj pl neut gen OR adj pl masc gen OR adj pl fem gen OR noun sg masc nom OR noun sg masc voc OR noun pl neut gen attic epic doric contr OR part sg pres act masc nom attic epic doric contr OR verb pres inf act doric OR verb 3rd pl imperf ind act doric aeolic poetic contr unaugmented OR verb 1st sg imperf ind act doric aeolic poetic contr unaugmented OR part sg pres act neut voc doric aeolic contr OR part sg pres act neut nom doric aeolic contr OR part sg pres act neut acc doric aeolic contr OR part sg pres act masc voc doric aeolic contr OR part sg pres act masc nom contr}
 \item \eintrag{φανερὸν}{ φανερός }{visible, manifest,}{adj sg neut voc OR adj sg neut nom OR adj sg neut acc OR adj sg masc acc OR adj sg fem acc}
 \item \eintrag{φανερῶς}{ φανερός φανερόω }{visible, manifest,}{adv OR adj pl masc acc doric OR adj pl fem acc doric OR verb 2nd sg pres ind act doric contr OR verb 2nd sg imperf ind act doric aeolic poetic contr unaugmented}
 \item \eintrag{φανεὶς}{ φαίνω φανάω }{A ren.}{verb 2nd sg aor subj pass epic contr short subj OR verb 2nd sg fut ind act attic epic doric ionic contr OR verb 2nd sg aor subj act epic doric short subj OR part sg aor pass masc nom OR part sg aor pass masc voc OR verb 2nd sg pres ind act attic epic doric ionic contr OR verb 2nd sg imperf ind act attic epic ionic contr unaugmented}
 \item \eintrag{φαρμακείαν}{ φαρμάκεια2 φαρμακεία }{keine Übersetzung gefunden}{noun sg fem acc OR noun pl fem gen doric aeolic OR noun pl fem gen doric aeolic OR noun sg fem acc attic doric aeolic}
 \item \eintrag{φαυλίσασα}{ φαυλίζω }{hold cheap,}{part sg aor act fem voc attic epic ionic OR part sg aor act fem nom attic epic ionic OR part dual aor act fem acc attic epic ionic OR part dual aor act fem nom attic epic ionic OR part dual aor act fem voc attic epic ionic}
 \item \eintrag{φαύλως}{ φαῦλος }{cheap, easy, slight, paltry,}{adv OR adj pl masc acc doric OR adj pl fem acc doric}
 \item \eintrag{φείδεσθαι}{ φείδομαι }{spare}{verb pres inf mp}
 \item \eintrag{φειδόμενος}{ φείδομαι }{spare}{part sg pres mp masc nom}
 \item \eintrag{φερομένην}{ φέρω }{fero,}{part sg pres mp fem acc attic epic ionic}
 \item \eintrag{φερομένους}{ φέρω }{fero,}{part pl pres mp masc acc}
 \item \eintrag{φεύγοντας}{ φεύγω }{flee, take flight,}{part pl pres act masc acc}
 \item \eintrag{φησίν}{ φημί }{Spir. Prooem., Eratosth.Prooem.}{verb 3rd sg subj act epic nu movable OR verb 3rd sg pres ind act enclitic nu movable}
 \item \eintrag{φθονερὸς}{ φθονερός }{envious, jealous,}{adj sg masc nom}
 \item \eintrag{φθόνον}{ φθόνος }{ill-will}{noun sg masc acc}
 \item \eintrag{φθόνου}{ φθόνος φθονέω }{ill-will}{noun sg masc gen OR verb 2nd sg pres imperat mp attic contr OR verb 2nd sg imperf ind mp attic poetic contr unaugmented}
 \item \eintrag{φιλάνθρωπα}{ φιλάνθρωπος }{loving mankind, humane, benevolent, tender-hearted,}{adj pl neut acc OR adj pl neut nom OR adj pl neut voc}
 \item \eintrag{φιλάνθρωπον}{ φιλάνθρωπος }{loving mankind, humane, benevolent, tender-hearted,}{adj sg neut nom OR adj sg neut voc OR adj sg fem acc OR adj sg masc acc OR adj sg neut acc}
 \item \eintrag{φιλέλλην}{ φιλέλλην }{fond of the Hellenes,}{noun sg masc voc OR noun sg masc nom OR noun sg fem voc OR noun sg fem nom}
 \item \eintrag{φιλέλληνι}{ φιλέλλην }{fond of the Hellenes,}{noun sg masc dat OR noun sg fem dat}
 \item \eintrag{φιλία}{ φίλιος φιλία φιλιάζω }{friendly,}{adj sg fem voc attic doric aeolic OR adj sg fem nom attic doric aeolic OR adj pl neut nom OR adj pl neut voc OR adj dual fem voc OR adj pl neut acc OR adj dual fem acc OR adj dual fem nom OR noun sg fem nom attic doric aeolic OR noun sg fem voc attic doric aeolic OR noun dual fem voc OR noun dual fem nom OR noun dual fem acc OR verb 1st sg fut ind act doric aeolic contr}
 \item \eintrag{φιλοπάτωρ}{ φιλοπάτωρ }{loving one's father,}{noun sg fem nom OR noun sg masc nom}
 \item \eintrag{φιλοπόνους}{ φιλόπονος }{laborious, industrious,}{adj pl masc acc OR adj pl fem acc}
 \item \eintrag{φιλόπονον}{ φιλόπονος }{laborious, industrious,}{adj sg neut nom OR adj sg neut voc OR adj sg neut acc OR adj sg masc acc OR adj sg fem acc}
 \item \eintrag{φρονήματος}{ φρόνημα }{mind, spirit,}{noun sg neut gen}
 \item \eintrag{φρονοίη}{ φρονέω }{to be minded,}{verb 3rd sg pres opt act}
 \item \eintrag{φροντίζων}{ φροντίζω }{consider, reflect, take thought,}{part sg pres act masc nom}
 \item \eintrag{φρουρὰς}{ φρουρά }{look-out, watch, guard,}{noun sg fem gen attic doric aeolic OR noun pl fem acc}
 \item \eintrag{φρουρῶν}{ φρουρά φρουρέω φρουρός }{look-out, watch, guard,}{noun pl fem gen OR part sg pres act masc nom attic epic doric contr OR noun pl masc gen}
 \item \eintrag{φυγήν}{ φεύγω φυγή }{flee, take flight,}{verb aor inf act epic doric contr OR noun sg fem acc attic epic ionic}
 \item \eintrag{φόβον}{ φόβος }{panic flight,}{noun sg masc acc}
 \item \eintrag{φόβου}{ φόβος φοβέω }{panic flight,}{noun sg masc gen OR verb 2nd sg pres imperat mp attic contr OR verb 2nd sg imperf ind mp attic poetic contr unaugmented}
 \item \eintrag{φύσεως}{ φύσις }{origin,}{noun sg fem gen attic}
 \item \eintrag{φύσιν}{ φύς φύσις φύω }{a son;}{part pl aor act masc dat nu movable OR noun sg fem acc OR part pl aor act neut dat nu movable OR part pl aor act masc dat nu movable}
 \item \eintrag{χάριν}{ Χάρις χάρις }{Graces}{noun sg fem acc OR prep indeclform OR noun sg fem acc OR adv indeclform}
 \item \eintrag{χίλια}{ χίλιοι χίλιος χιλιάς χιλιάζω }{a thousand,}{adj pl neut nom OR adj pl neut voc OR adj pl neut acc OR adj sg fem voc attic doric aeolic OR adj sg fem nom attic doric aeolic OR adj pl neut nom OR adj pl neut voc OR adj pl neut acc OR adj dual fem nom OR adj dual fem voc OR adj dual fem acc OR noun sg fem voc OR verb 1st sg fut ind act doric aeolic contr}
 \item \eintrag{χαρίτων}{ Χάρις χάρις χάριτος χαριτόω }{Graces}{noun pl fem gen OR noun pl fem gen OR adj pl neut gen OR adj pl masc gen OR adj pl fem gen OR verb pres inf act doric OR verb 3rd pl imperf ind act doric aeolic poetic contr unaugmented OR verb 1st sg imperf ind act doric aeolic poetic contr unaugmented OR part sg pres act neut voc doric aeolic contr OR part sg pres act neut nom doric aeolic contr OR part sg pres act neut acc doric aeolic contr OR part sg pres act masc voc doric aeolic contr OR part sg pres act masc nom contr}
 \item \eintrag{χαρᾶς}{ χαρά }{joy, delight,}{noun sg fem gen attic doric aeolic OR noun pl fem acc}
 \item \eintrag{χειμάζων}{ χειμάζω }{expose to the winter cold}{part sg pres act masc nom}
 \item \eintrag{χιλίους}{ χίλιοι χίλιος }{a thousand,}{adj pl masc acc OR adj pl masc acc}
 \item \eintrag{χλαμύδας}{ χλαμύς }{short mantle,}{noun pl fem acc}
 \item \eintrag{χρήμασιν}{ χρῆμα }{need,}{noun pl neut dat nu movable}
 \item \eintrag{χρήματα}{ χρῆμα }{need,}{noun pl neut voc OR noun pl neut nom OR noun pl neut acc}
 \item \eintrag{χρημάτων}{ χρῆμα }{need,}{noun pl neut gen}
 \item \eintrag{χρησιμώτατός}{ χρήσιμος }{useful, serviceable,}{adj sg masc nom superl}
 \item \eintrag{χρυσίον}{ χρύσεος χρυσίον }{golden,}{adj sg neut voc aeolic OR adj sg neut nom aeolic OR adj sg neut acc aeolic OR adj sg masc acc aeolic OR noun sg neut voc diminutive OR noun sg neut acc diminutive OR noun sg neut nom diminutive}
 \item \eintrag{χρυσίου}{ χρύσεος χρυσίον }{golden,}{adj sg neut gen aeolic OR adj sg masc gen aeolic OR noun sg neut gen diminutive}
 \item \eintrag{χρυσοῦς}{ χρύσεος χρυσός χρυσόω }{golden,}{adj pl masc acc attic epic contr OR adj sg masc nom attic epic contr OR noun pl masc acc OR verb 2nd sg imperf ind act homeric ionic contr unaugmented OR verb 2nd sg pres ind act doric contr}
 \item \eintrag{χρυσᾶ}{ Χρύση Χρύσης χρύσεος }{Chryse}{noun sg fem voc doric aeolic OR noun sg fem nom doric aeolic OR noun sg masc voc OR noun sg masc nom epic OR noun dual masc voc OR noun sg masc gen doric aeolic contr OR noun dual masc nom OR noun dual masc acc OR adj pl neut voc attic contr OR adj pl neut nom attic contr OR adj dual fem voc attic contr OR adj pl neut acc attic contr OR adj dual fem nom attic contr OR adj dual fem acc attic contr}
 \item \eintrag{χρυσῶν}{ χρύσεος χρυσός χρυσόω χρυσών }{golden,}{adj pl fem gen attic epic contr OR adj pl masc gen attic epic contr OR adj pl neut gen attic epic contr OR noun pl masc gen OR part sg pres act masc nom contr OR part sg pres act masc voc doric aeolic contr OR part sg pres act neut acc doric aeolic contr OR part sg pres act neut nom doric aeolic contr OR part sg pres act neut voc doric aeolic contr OR verb 1st sg imperf ind act doric aeolic poetic contr unaugmented OR verb 3rd pl imperf ind act doric aeolic poetic contr unaugmented OR verb pres inf act doric OR noun sg masc nom OR noun sg masc voc}
 \item \eintrag{χρόνον}{ χρόνος }{time,}{noun sg masc acc}
 \item \eintrag{χρώμενοι}{ χράομαι χράω χράω2 χραύω }{consulting}{part pl pres mp masc voc attic contr OR part pl pres mp masc nom attic contr OR part pl pres mp masc voc contr OR part pl pres mp masc nom contr OR part pl pres mp masc voc attic contr OR part pl pres mp masc nom attic contr OR part pl pres mid masc voc ionic contr OR part pl pres mid masc nom ionic contr OR part pl pres mp masc nom contr OR part pl pres mp masc voc contr}
 \item \eintrag{χρῆσθαι}{ χράομαι χράω χράω2 χραύω }{consulting}{verb pres inf mp attic epic ionic contr OR verb pres inf mp doric contr OR verb pres inf mp attic epic ionic contr OR verb pres inf mp doric contr}
 \item \eintrag{χωρίον}{ χωρέω χωρίον }{make room for another, give way, withdraw,}{verb 1st sg imperf ind act doric poetic unaugmented OR verb 3rd pl imperf ind act doric poetic unaugmented OR part sg pres act neut nom doric OR part sg pres act neut voc doric OR part sg pres act masc voc doric OR part sg pres act neut acc doric OR noun sg neut voc diminutive OR noun sg neut nom diminutive OR noun sg neut acc diminutive}
 \item \eintrag{χωρίων}{ χωρέω χωρίον χωρίζω }{make room for another, give way, withdraw,}{part sg pres act masc nom doric OR noun pl neut gen diminutive OR part sg fut act masc nom attic epic doric contr}
 \item \eintrag{χωρεῖν}{ χωρέω }{make room for another, give way, withdraw,}{verb pres inf act attic epic doric contr}
 \item \eintrag{χωρὶς}{ χωρίς }{separately, apart,}{adv indeclform}
 \item \eintrag{ψέλια}{ ψέλιον }{armlet}{noun pl neut acc OR noun pl neut nom OR noun pl neut voc}
 \item \eintrag{ἀγαθήν}{ ἀγάω ἀγαθός }{keine Übersetzung gefunden}{verb 1st sg aor ind pass doric aeolic OR verb 3rd pl aor ind pass epic doric aeolic OR adj sg fem acc attic epic ionic}
 \item \eintrag{ἀγαθοῖς}{ ἀγαθός ἀγαθόω }{good:}{adj pl neut dat OR adj pl masc dat OR verb 2nd sg pres subj act contr OR verb 2nd sg pres opt act contr OR verb 2nd sg pres ind act contr}
 \item \eintrag{ἀγαθὸν}{ ἀγαθός }{good:}{adj sg neut voc OR adj sg neut nom OR adj sg neut acc OR adj sg masc acc}
 \item \eintrag{ἀγανακτοῦντές}{ ἀγανακτέω }{feel a violent irritation,}{part pl pres act masc voc attic epic doric contr OR part pl pres act masc nom attic epic doric contr}
 \item \eintrag{ἀγανακτῶν}{ ἀγανακτέω }{feel a violent irritation,}{part sg pres act masc nom attic epic doric contr}
 \item \eintrag{ἀγαπᾶν}{ ἀγάπη ἀγαπάω ἀγαπάζω }{love,}{noun sg fem acc doric aeolic OR noun pl fem gen doric aeolic OR verb pres inf act epic doric contr OR verb 3rd pl imperf ind act doric aeolic contr OR verb 1st sg imperf ind act doric aeolic contr OR part sg pres act neut nom doric aeolic contr OR part sg pres act neut voc doric aeolic contr OR part sg pres act neut acc doric aeolic contr OR part sg pres act masc voc doric aeolic contr OR part sg pres act masc nom doric aeolic contr OR verb fut inf act OR part sg fut act neut voc doric aeolic contr OR part sg fut act neut nom doric aeolic contr OR part sg fut act neut acc doric aeolic contr OR part sg fut act masc voc doric aeolic contr OR part sg fut act masc nom doric aeolic contr}
 \item \eintrag{ἀγαπῷτο}{ ἀγαπάω }{greet with affection}{verb 3rd sg pres opt mp contr}
 \item \eintrag{ἀγνοεῖν}{ ἀγνοέω }{not to perceive}{verb pres inf act attic epic doric contr}
 \item \eintrag{ἀγνοεῖς}{ ἀγνοέω }{not to perceive}{verb 2nd sg pres ind act attic epic doric ionic aeolic contr OR verb 2nd sg imperf ind act attic epic doric aeolic contr}
 \item \eintrag{ἀγορὰν}{ ἀγορά ἀγοράζω }{assembly}{noun pl fem gen doric aeolic OR noun sg fem acc attic doric aeolic OR verb fut inf act OR part sg fut act neut voc doric aeolic contr OR part sg fut act neut nom doric aeolic contr OR part sg fut act masc voc doric aeolic contr OR part sg fut act neut acc doric aeolic contr OR part sg fut act masc nom doric aeolic contr}
 \item \eintrag{ἀγῶνα}{ ἄγωνος ἄγωνος ἀγών }{without angle}{adj pl neut voc OR adj pl neut nom OR adj pl neut acc OR adj pl neut voc OR adj pl neut nom OR adj pl neut acc OR noun sg masc acc}
 \item \eintrag{ἀγῶνος}{ ἄγωνος ἄγωνος ἀγών }{without angle}{adj sg masc nom OR adj sg fem nom OR adj sg masc nom OR adj sg fem nom OR noun sg masc gen}
 \item \eintrag{ἀδίκως}{ ἄδικος }{wrongdoing, unrighteous, unjust}{adv OR adj pl fem acc doric OR adj pl masc acc doric}
 \item \eintrag{ἀδελφὸν}{ ἀδελφός }{son of the same mother}{noun sg masc acc}
 \item \eintrag{ἀδικεῖν}{ ἀδικέω }{to be}{verb pres inf act attic epic doric contr}
 \item \eintrag{ἀεί}{ ἀεί }{ever, always}{adv indeclform}
 \item \eintrag{ἀεὶ}{ ἀεί }{ever, always}{adv indeclform}
 \item \eintrag{ἀθεμίστως}{ ἀθέμιστος }{unlawful}{adv poetic OR adj pl masc acc doric poetic OR adj pl fem acc doric poetic}
 \item \eintrag{ἀθρόως}{ ἀθρόος }{in crowds, heaps}{adv OR adj pl masc acc attic doric OR adj pl fem acc attic doric}
 \item \eintrag{ἀκοῦσαι}{ ἀέκων ἀκέω ἀκέω2 ἀκούω }{involuntary, constrained}{adj sg fem dat attic doric contr OR adj pl fem voc attic epic doric ionic contr OR adj pl fem nom attic epic doric ionic contr OR part sg pres act fem dat doric contr OR part pl pres act fem voc attic epic doric contr OR part pl pres act fem nom attic epic doric contr OR part sg pres act fem dat doric contr OR part pl pres act fem voc attic epic doric contr OR part pl pres act fem nom attic epic doric contr OR verb aor inf act OR verb 2nd sg aor imperat mid OR verb 3rd sg aor opt act}
 \item \eintrag{ἀκροάσασθαι}{ ἀκροάομαι ἀκροάζομαι }{hearken, listen to}{verb aor inf mid OR verb aor inf mp}
 \item \eintrag{ἀλλʼ}{ }{keine Übersetzung gefunden}{Nichts gefunden}
 \item \eintrag{ἀλλήλοις}{ ἀλλήλων }{of one another, to one another, one another}{adj pl neut dat OR adj pl masc dat}
 \item \eintrag{ἀλλήλους}{ ἀλλήλων }{of one another, to one another, one another}{adj pl masc acc}
 \item \eintrag{ἀλλὰ}{ ἄλλος ἄλλος ἀλλά }{y}{adj sg fem voc doric aeolic OR adj sg fem nom doric aeolic OR adj pl neut voc OR adj pl neut nom OR adj dual fem voc OR adj pl neut acc OR adj dual fem nom OR adj dual fem acc OR adj sg fem voc doric aeolic OR adj sg fem nom doric aeolic OR adj pl neut voc OR adj pl neut nom OR adj pl neut acc OR adj dual fem voc OR adj dual fem nom OR adj dual fem acc OR adv indeclform}
 \item \eintrag{ἀλογιστίαν}{ ἀλογιστία }{thoughtlessness,}{noun sg fem acc attic doric aeolic OR noun pl fem gen doric aeolic}
 \item \eintrag{ἀλογωτέρους}{ ἄλογος }{without}{adj pl masc acc comp}
 \item \eintrag{ἀμέριμνος}{ ἀμέριμνος }{free from care, unconcerned,}{adj sg masc nom OR adj sg fem nom}
 \item \eintrag{ἀμέτρως}{ ἄμετρος }{without}{adv OR adj pl fem acc doric OR adj pl masc acc doric}
 \item \eintrag{ἀμυνόμενος}{ ἀμύνω }{keep off, ward off,}{part sg pres mp masc nom}
 \item \eintrag{ἀμφοτέροις}{ ἀμφότερος }{either,}{adj pl neut dat OR adj pl masc dat}
 \item \eintrag{ἀμφοτέρους}{ ἀμφότερος }{either,}{adj pl masc acc}
 \item \eintrag{ἀμφοῖν}{ ἄμφω }{both,}{adj dual neut gen indeclform OR adj dual neut dat indeclform OR adj dual masc gen indeclform OR adj dual masc dat indeclform OR adj dual fem gen indeclform OR adj dual fem dat indeclform}
 \item \eintrag{ἀμφότεροι}{ ἀμφότερος }{either,}{adj pl masc voc OR adj pl masc nom}
 \item \eintrag{ἀνέβαινε}{ ἀναβαίνω }{go up, mount,}{verb 3rd sg imperf ind act}
 \item \eintrag{ἀνέλαβεν}{ ἀναλαμβάνω }{take up, take into one's hands,}{verb 3rd sg aor ind act nu movable}
 \item \eintrag{ἀνέλοι}{ ἀναιρέω }{take up,}{verb 3rd sg aor opt act OR verb 3rd sg fut opt act attic epic doric contr}
 \item \eintrag{ἀνέτρεψαν}{ ἀνατρέπω }{overturn, upset,}{verb 3rd pl aor ind act}
 \item \eintrag{ἀνίστατο}{ ἀνίστημι }{make to stand up, raise up,}{verb 3rd sg imperf ind mp causal pres redupl}
 \item \eintrag{ἀναγκαζομένους}{ ἀναγκάζω }{force, compel,}{part pl pres mp masc acc}
 \item \eintrag{ἀναγνῶναι}{ ἀναγιγνώσκω }{know well, know certainly,}{verb aor inf act}
 \item \eintrag{ἀναθαρρῶν}{ ἀναθαρσέω }{regain courage,}{part sg pres act masc nom attic epic doric contr}
 \item \eintrag{ἀναιρουμένους}{ ἀναιρέω ἀνιερόω }{take up,}{part pl pres mp masc acc attic epic doric contr OR part pl pres mp masc acc ionic poetic contr raw preverb}
 \item \eintrag{ἀναλαμβάνειν}{ ἀναλαμβάνω }{take up, take into one's hands,}{verb pres inf act attic epic contr n infix}
 \item \eintrag{ἀναλαμβάνοντι}{ ἀναλαμβάνω }{take up, take into one's hands,}{verb 3rd pl pres ind act doric n infix OR part sg pres act neut dat n infix OR part sg pres act masc dat n infix}
 \item \eintrag{ἀναλῶσαι}{ ἀναλίσκω ἀναλόω }{use up, spend,}{verb aor inf act OR verb 2nd sg perf ind mp no redupl OR verb 3rd sg aor opt act OR verb 2nd sg aor imperat mid OR verb 2nd sg aor imperat mid OR verb 2nd sg perf ind mp doric aeolic redupl OR verb 3rd sg aor opt act OR verb aor inf act}
 \item \eintrag{ἀναμιμνήσκει}{ ἀναμιμνήσκω }{remind}{verb 3rd sg pres ind act pres redupl OR verb 2nd sg pres ind mp pres redupl}
 \item \eintrag{ἀναπλέοντας}{ ἀναπλέω }{sail upwards, go up-stream,}{part pl pres act masc acc epic doric ionic aeolic}
 \item \eintrag{ἀναστρέφειν}{ ἀναστρέφω }{turn upside down,}{verb pres inf act attic epic contr}
 \item \eintrag{ἀνασχόμενος}{ ἀνέχω }{hold up, lift up,}{part sg aor mid masc nom}
 \item \eintrag{ἀναφέροντα}{ ἀναφέρω }{bring, carry up,}{part sg pres act masc acc OR part pl pres act neut voc OR part pl pres act neut acc OR part pl pres act neut nom}
 \item \eintrag{ἀνδράσιν}{ ἀνήρ }{nar-}{noun pl masc dat nu movable indeclform}
 \item \eintrag{ἀνδριάντας}{ ἀνδριάς }{image of a man, statue,}{noun pl masc acc}
 \item \eintrag{ἀνδρός}{ ἀνήρ }{nar-}{noun sg masc gen indeclform}
 \item \eintrag{ἀνεγράφοντο}{ ἀναγράφω }{engrave and set up publicly,}{verb 3rd pl imperf ind mp}
 \item \eintrag{ἀνειπεῖν}{ ἀνειπεῖν }{to say aloud, announce, proclaim}{verb aor inf act attic epic doric contr}
 \item \eintrag{ἀνεκαίνισεν}{ ἀνακαινίζω }{renew,}{verb 3rd sg aor ind act nu movable}
 \item \eintrag{ἀνελόντες}{ ἀναιρέω }{take up,}{part pl aor act masc nom OR part pl aor act masc voc}
 \item \eintrag{ἀνενεώσασθε}{ ἀνανεόομαι }{renew,}{verb 2nd pl aor ind mid}
 \item \eintrag{ἀνεπαύοντο}{ ἀναπαύω }{make to cease, stop}{verb 3rd pl imperf ind mp}
 \item \eintrag{ἀνθρώπων}{ ἄνθρωπος ἀνθρωπώ }{man,}{noun pl masc gen OR noun pl fem gen contr}
 \item \eintrag{ἀνιόντας}{ ἄνειμι ἀνέω }{go up,}{part pl pres act masc acc OR part pl pres act masc acc doric}
 \item \eintrag{ἀνοχὰς}{ ἀνοχεύς ἀνοχή }{suspensory membrane,}{noun pl masc acc contr OR noun pl fem acc OR noun sg fem gen doric aeolic}
 \item \eintrag{ἀντέλεγε}{ ἀντιλέγω }{speak against, gainsay, contradict,}{verb 3rd sg imperf ind act elide preverb}
 \item \eintrag{ἀντιλέξων}{ ἀντιλέγω }{speak against, gainsay, contradict,}{part sg fut act masc nom doric contr}
 \item \eintrag{ἀντιλογίαν}{ ἀντιλογία }{contradiction, controversy,}{noun sg fem acc attic doric aeolic OR noun pl fem gen doric aeolic}
 \item \eintrag{ἀντιπέμψαντος}{ ἀντιπέμπω }{send back an answer,}{part sg aor act neut gen OR part sg aor act masc gen}
 \item \eintrag{ἀντὶ}{ ἀντί }{over against.}{prep indeclform}
 \item \eintrag{ἀνωμαλίας}{ ἀνωμαλία }{unevenness, irregularity,}{noun pl fem acc a priv OR noun sg fem gen attic doric aeolic a priv}
 \item \eintrag{ἀξίας}{ ἄξιος ἀξία ἀξιάω }{counterbalancing,}{adj sg fem gen attic doric aeolic OR adj pl fem acc OR noun pl fem acc OR noun sg fem gen attic doric aeolic OR verb 2nd sg pres ind act doric aeolic contr OR verb 2nd sg imperf ind act doric aeolic contr}
 \item \eintrag{ἀξιολογώτατα}{ ἀξιόλογος }{worthy of mention, remarkable,}{adv superl OR adj sg fem voc superl doric aeolic OR adj sg fem nom superl doric aeolic OR adj pl neut voc superl OR adj pl neut nom superl OR adj pl neut acc superl OR adj dual fem voc superl OR adj dual fem nom superl OR adj dual fem acc superl}
 \item \eintrag{ἀξιούμενον}{ ἀξιόω }{think, deem worthy,}{part sg pres mp neut voc contr OR part sg pres mp neut nom contr OR part sg pres mp masc acc contr OR part sg pres mp neut acc contr}
 \item \eintrag{ἀξιοῦντες}{ ἀξιόω }{think, deem worthy,}{part pl pres act masc voc contr OR part pl pres act masc nom contr}
 \item \eintrag{ἀξιοῦσα}{ ἀξιόω }{think, deem worthy,}{part sg pres act fem voc attic ionic contr OR part dual pres act fem voc attic ionic contr OR part sg pres act fem nom attic ionic contr OR part dual pres act fem nom attic ionic contr OR part dual pres act fem acc attic ionic contr}
 \item \eintrag{ἀπʼ}{ }{keine Übersetzung gefunden}{Nichts gefunden}
 \item \eintrag{ἀπέθανεν}{ ἀποθνήσκω }{die,}{verb 3rd sg aor ind act nu movable}
 \item \eintrag{ἀπέκτεινεν}{ ἀποκτείνω }{kill, slay,}{verb 3rd sg imperf ind act nu movable OR verb 3rd sg aor ind act nu movable}
 \item \eintrag{ἀπέλυσε}{ ἀπολύω }{loose from,}{verb 3rd sg aor ind act}
 \item \eintrag{ἀπέστειλε}{ ἀποστέλλω }{send off}{verb 3rd sg aor ind act}
 \item \eintrag{ἀπέτυχε}{ ἀποτυγχάνω }{fail in hitting}{verb 3rd sg aor ind act}
 \item \eintrag{ἀπίστου}{ ἄπιστος ἀπιστέω }{not to be trusted,}{adj sg masc gen OR adj sg neut gen OR adj sg fem gen OR verb 2nd sg pres imperat mp attic contr OR verb 2nd sg imperf ind mp attic doric aeolic contr}
 \item \eintrag{ἀπαχθέντες}{ ἀπάγω }{lead away, carry off,}{part pl aor pass masc nom OR part pl aor pass masc voc}
 \item \eintrag{ἀπειρίαν}{ ἀπειρία ἀπειρία2 }{want of skill, inexperience, ignorance,}{noun sg fem acc attic doric aeolic OR noun pl fem gen doric aeolic OR noun sg fem acc attic doric aeolic OR noun pl fem gen doric aeolic}
 \item \eintrag{ἀπεκρίνατο}{ ἀποκρίνω }{set apart,}{verb 3rd sg aor ind mid}
 \item \eintrag{ἀπεπείρασε}{ ἀποπειράομαι }{make trial}{verb 3rd sg aor ind act attic r e i alpha}
 \item \eintrag{ἀπεπείρασεν}{ ἀποπειράομαι }{make trial}{verb 3rd sg aor ind act attic r e i alpha nu movable}
 \item \eintrag{ἀπεσταλμένοι}{ ἀποστέλλω }{send off}{part pl perf mp masc voc OR part pl perf mp masc nom}
 \item \eintrag{ἀπεστρέφοντο}{ ἀποστρέφω }{turn back}{verb 3rd pl imperf ind mp}
 \item \eintrag{ἀπεχώρουν}{ ἀποχωρέω }{go from}{verb 1st sg imperf ind act attic epic doric contr OR verb 3rd pl imperf ind act attic epic doric contr}
 \item \eintrag{ἀπιστούμενον}{ ἀπιστέω }{to be}{part sg pres mp neut voc attic epic doric contr OR part sg pres mp neut nom attic epic doric contr OR part sg pres mp neut acc attic epic doric contr OR part sg pres mp masc acc attic epic doric contr}
 \item \eintrag{ἀποδοθῆναι}{ ἀποδίδωμι }{give up}{verb aor inf pass}
 \item \eintrag{ἀποδοῦναι}{ ἀποδίδωμι }{give up}{verb aor inf act}
 \item \eintrag{ἀποδόντες}{ ἀποδίδωμι }{give up}{part pl aor act masc nom epic OR part pl aor act masc voc epic}
 \item \eintrag{ἀποθανεῖν}{ ἀποθνήσκω }{die,}{verb aor inf act attic epic doric contr}
 \item \eintrag{ἀποθανόντων}{ ἀποθνήσκω }{die,}{verb 3rd pl aor imperat act OR part pl aor act neut gen OR part pl aor act masc gen}
 \item \eintrag{ἀποκλείουσαν}{ ἀποκλείω }{shut off from}{part sg pres act fem acc attic epic doric ionic OR part pl pres act fem gen doric}
 \item \eintrag{ἀποκρινάμενοι}{ ἀποκρίνω }{set apart,}{part pl aor mid masc voc OR part pl aor mid masc nom}
 \item \eintrag{ἀποκριναμένων}{ ἀποκρίνω }{set apart,}{part pl aor mid neut gen OR part pl aor mid masc gen OR part pl aor mid fem gen}
 \item \eintrag{ἀπολιπών}{ ἀπολιμπάνω }{to leave}{part sg aor act masc nom}
 \item \eintrag{ἀποπειρώμενος}{ ἀποπειράομαι ἀποπειράζω }{make trial}{part sg pres mp masc nom contr OR part sg fut mid masc nom contr}
 \item \eintrag{ἀποπλευσάντων}{ ἀποπλέω }{sail away, sail off,}{verb 3rd pl aor imperat act OR part pl aor act masc gen OR part pl aor act neut gen}
 \item \eintrag{ἀπορῶν}{ ἄπορος ἀφοράω ἀπορέω ἀπορέω2 ἀπορραίνω }{without passage, having no way in, out,}{adj pl neut gen OR adj pl masc gen OR adj pl fem gen OR verb 3rd pl imperf ind act ionic contr unaugmented unasp preverb OR verb 1st sg imperf ind act ionic contr unaugmented unasp preverb OR part sg pres act neut voc attic epic doric ionic aeolic parad form prose contr unasp preverb OR part sg pres act neut nom attic epic doric ionic aeolic parad form prose contr unasp preverb OR part sg pres act neut acc attic epic doric ionic aeolic parad form prose contr unasp preverb OR part sg pres act masc voc attic epic doric ionic aeolic parad form prose contr unasp preverb OR part sg pres act masc nom ionic contr unasp preverb OR part sg pres act masc nom attic epic doric contr OR part sg pres act masc nom attic epic doric contr OR part sg fut act neut voc epic poetic contr raw preverb OR part sg fut act neut nom epic poetic contr raw preverb OR part sg fut act neut acc epic poetic contr raw preverb OR part sg fut act masc nom attic epic ionic poetic contr raw preverb OR part sg fut act masc voc epic poetic contr raw preverb}
 \item \eintrag{ἀποσκήψαντος}{ ἀποσκήπτω }{hurl from above,}{part sg aor act neut gen OR part sg aor act masc gen}
 \item \eintrag{ἀποστεῖλαι}{ ἀποστέλλω }{send off}{verb 3rd sg aor opt act OR verb aor inf act OR verb 2nd sg aor imperat mid}
 \item \eintrag{ἀπόρρητον}{ ἀπόρρητος ἀπορραίνω ἀπορρέω }{forbidden,}{adj sg neut nom OR adj sg neut voc OR adj sg masc acc OR adj sg neut acc OR adj sg fem acc OR verb 3rd dual fut ind act epic doric contr OR verb 2nd dual fut ind act epic doric contr OR verb 3rd dual pres subj act contr OR verb 3rd dual pres ind act doric aeolic contr OR verb 2nd dual pres imperat act doric aeolic contr OR verb 2nd dual pres subj act contr OR verb 2nd dual pres ind act doric aeolic contr OR verb 2nd dual imperf ind act doric aeolic poetic contr unaugmented}
 \item \eintrag{ἀπὸ}{ ἀπό }{ápa,}{prep indeclform}
 \item \eintrag{ἀπῆγεν}{ ἀπάγω }{lead away, carry off,}{verb 3rd sg imperf ind act attic epic ionic nu movable}
 \item \eintrag{ἀργυρίου}{ ἀργύριον }{small coin, piece of money,}{noun sg neut gen}
 \item \eintrag{ἀρχήν}{ ἄρχω ἀρχή ἀρχήν }{to be first,}{verb pres inf act doric aeolic contr OR noun sg fem acc attic epic ionic OR adv indeclform}
 \item \eintrag{ἀρχομένης}{ ἄρχω }{to be first,}{part sg pres mp fem gen attic epic ionic}
 \item \eintrag{ἀρχομένους}{ ἄρχω }{to be first,}{part pl pres mp masc acc}
 \item \eintrag{ἀρχὴν}{ ἄρχω ἀρχή ἀρχήν }{to be first,}{verb pres inf act doric aeolic contr OR noun sg fem acc attic epic ionic OR adv indeclform}
 \item \eintrag{ἀρχῆς}{ ἀρχή }{beginning, origin,}{noun sg fem gen attic epic ionic}
 \item \eintrag{ἀσθένειαν}{ ἀσθένεια }{want of strength, weakness,}{noun pl fem gen doric ionic aeolic OR noun sg fem acc}
 \item \eintrag{ἀσθενοποιοῦντες}{ ἀσθενοποιέω }{make weak,}{part pl pres act masc voc attic epic doric contr OR part pl pres act masc nom attic epic doric contr}
 \item \eintrag{ἀσφαλῶς}{ ἀσφαλής ἀσφαλός }{not liable to fall, immovable, steadfast,}{adv attic epic doric contr OR noun pl masc acc doric}
 \item \eintrag{ἀτολμίαν}{ ἀτολμία }{want of daring, cowardice,}{noun pl fem gen doric aeolic OR noun sg fem acc attic doric aeolic}
 \item \eintrag{ἀτριβῆ}{ ἀτριβής }{not rubbed}{adj sg fem acc attic epic doric contr OR adj sg masc acc attic epic doric contr OR adj pl neut voc attic epic doric contr OR adj pl neut nom attic epic doric contr OR adj dual neut voc doric aeolic contr OR adj pl neut acc attic epic doric contr OR adj dual neut nom doric aeolic contr OR adj dual neut acc doric aeolic contr OR adj dual masc acc doric aeolic contr OR adj dual masc nom doric aeolic contr OR adj dual masc voc doric aeolic contr OR adj dual fem voc doric aeolic contr OR adj dual fem nom doric aeolic contr OR adj dual fem acc doric aeolic contr}
 \item \eintrag{ἀφʼ}{ }{keine Übersetzung gefunden}{Nichts gefunden}
 \item \eintrag{ἀφίκοντο}{ ἀφικνέομαι }{arrive at, come to, reach:}{verb 3rd pl aor ind mid}
 \item \eintrag{ἀφείλοντο}{ ἀφαιρέω }{take away from}{verb 3rd pl aor ind mid syll augment}
 \item \eintrag{ἀφικομένους}{ ἀφικνέομαι }{arrive at, come to, reach:}{part pl aor mid masc acc}
 \item \eintrag{ἀφιᾶσι}{ ἀφίημι }{send forth, discharge,}{verb 3rd pl pres ind act}
 \item \eintrag{ἀφορολόγητον}{ ἀφορολόγητος }{not subjected to tribute,}{adj sg neut nom OR adj sg neut voc OR adj sg masc acc OR adj sg neut acc OR adj sg fem acc}
 \item \eintrag{ἀφροσύνης}{ ἀφροσύνη }{folly, thoughtlessness,}{noun sg fem gen attic epic ionic}
 \item \eintrag{ἀφρούρητον}{ ἀφρούρητος ἀφρουρέω }{unguarded,}{adj sg neut nom OR adj sg neut voc OR adj sg neut acc OR adj sg fem acc OR adj sg masc acc OR verb 3rd dual pres subj act contr OR verb 2nd dual pres subj act contr OR verb 3rd dual pres ind act doric aeolic contr OR verb 2nd dual pres ind act doric aeolic contr OR verb 2nd dual pres imperat act doric aeolic contr OR verb 2nd dual imperf ind act doric aeolic poetic contr unaugmented}
 \item \eintrag{ἀφῄρηται}{ ἀφαιρέω }{take away from}{verb 3rd sg perf ind mp}
 \item \eintrag{ἀφῄρητο}{ ἀφαιρέω }{take away from}{verb 3rd sg plup ind mp OR verb 3rd sg imperf ind mp doric aeolic contr}
 \item \eintrag{ἀφῆκεν}{ ἀφήκω ἀφίημι }{arrive at or have arrived,}{verb pres inf act epic doric OR verb 3rd sg imperf ind act nu movable OR verb 3rd sg aor ind act nu movable}
 \item \eintrag{ἁλωνευόμενος}{ ἁλωνεύομαι }{work on a threshing-floor,}{part sg pres mp masc nom}
 \item \eintrag{ἁμαρτάνειν}{ ἁμαρτάνω }{Acut. (Sp.)}{verb pres inf act attic epic contr}
 \item \eintrag{ἁμαρτόντων}{ ἁμαρτάνω }{Acut. (Sp.)}{verb 3rd pl aor imperat act OR part pl aor act neut gen OR part pl aor act masc gen}
 \item \eintrag{ἁπάντων}{ ἅπας }{sṃ-,}{adj pl neut gen OR adj pl masc gen}
 \item \eintrag{ἂν}{ ἄν ἄν2 ἀνά ἐάν }{he came,}{partic indeclform OR conj attic indeclform OR prep poetic indeclform OR conj contr indeclform}
 \item \eintrag{ἃ}{ ὅς ὅς ὁ }{yas, yā, yad,}{pron sg fem nom doric aeolic indeclform OR pron pl neut nom indeclform OR pron pl neut acc indeclform OR pron dual fem nom indeclform OR pron dual fem acc indeclform OR pron sg fem nom doric aeolic indeclform OR pron pl neut nom indeclform OR pron pl neut acc indeclform OR pron dual fem nom indeclform OR pron dual fem acc indeclform OR article sg fem nom doric proclitic indeclform}
 \item \eintrag{ἃς}{ ἕως ὅς ὅς }{until, till}{conj doric aeolic indeclform OR pron sg fem gen doric aeolic indeclform OR pron pl fem acc indeclform OR pron sg fem gen doric aeolic indeclform OR pron pl fem acc indeclform}
 \item \eintrag{ἄδηλον}{ ἄδηλος }{unseen, invisible}{adj sg neut voc OR adj sg neut nom OR adj sg neut acc OR adj sg masc acc OR adj sg fem acc}
 \item \eintrag{ἄδικον}{ ἄδικος }{wrongdoing, unrighteous, unjust}{adj sg neut voc OR adj sg neut acc OR adj sg neut nom OR adj sg masc acc OR adj sg fem acc}
 \item \eintrag{ἄλλαις}{ ἄλλος ἄλλος }{y}{adj pl fem dat OR adj pl fem dat}
 \item \eintrag{ἄλλας}{ ἄλλος ἄλλος ἀλλᾶς }{y}{adj sg fem gen doric aeolic OR adj pl fem acc OR adj sg fem gen doric aeolic OR adj pl fem acc OR noun sg masc voc OR noun sg masc nom}
 \item \eintrag{ἄλλο}{ ἄλλος ἄλλος }{y}{adj sg neut voc OR adj sg neut nom OR adj sg neut acc OR adj sg neut voc OR adj sg neut nom OR adj sg neut acc}
 \item \eintrag{ἄλλοις}{ ἄλλος ἄλλος }{y}{adj pl neut dat OR adj pl masc dat OR adj pl neut dat OR adj pl masc dat}
 \item \eintrag{ἄλλον}{ ἄλλος ἄλλος }{y}{adj sg masc acc OR adj sg masc acc}
 \item \eintrag{ἄλλος}{ ἄλλος ἄλλος }{y}{adj sg masc nom OR adj sg masc nom}
 \item \eintrag{ἄλλους}{ ἄλλος ἄλλος }{y}{adj pl masc acc OR adj pl masc acc}
 \item \eintrag{ἄλλων}{ ἄλλος ἄλλος }{y}{adj pl neut gen OR adj pl masc gen OR adj pl fem gen OR adj pl neut gen OR adj pl masc gen OR adj pl fem gen}
 \item \eintrag{ἄμφω}{ ἄμφω }{both,}{adj dual neut voc indeclform OR adj dual neut nom indeclform OR adj dual neut acc indeclform OR adj dual masc voc indeclform OR adj dual masc nom indeclform OR adj dual masc acc indeclform OR adj dual fem voc indeclform OR adj dual fem nom indeclform OR adj dual fem acc indeclform}
 \item \eintrag{ἄνακτας}{ ἄναξ ἀνακτός }{lord, master,}{noun pl masc acc OR adj sg fem gen doric aeolic OR adj pl fem acc}
 \item \eintrag{ἄνδρας}{ ἀνήρ }{nar-}{noun pl masc acc indeclform}
 \item \eintrag{ἄνδρες}{ ἀνήρ }{nar-}{noun pl masc voc indeclform OR noun pl masc nom indeclform}
 \item \eintrag{ἄνευ}{ ἄνευ ἄνω ἀνέω ἀνίημι }{without,}{prep indeclform OR verb 2nd sg pres imperat mp epic doric ionic contr OR verb 2nd sg imperf ind mp epic doric ionic aeolic contr OR verb 2nd sg pres imperat mp doric ionic contr OR verb 2nd sg imperf ind mp doric ionic contr unaugmented OR verb aor imperat mid epic doric ionic contr}
 \item \eintrag{ἄξειν}{ ἄγνυμι ἄγω }{break, shiver}{verb fut inf act doric contr OR verb fut inf act doric contr}
 \item \eintrag{ἄξιον}{ Ἀξιός ἄξιος ἄγνυμι ἄγω }{keine Übersetzung gefunden}{noun sg masc acc OR adj sg neut voc OR adj sg neut nom OR adj sg masc acc OR adj sg neut acc OR part sg fut act neut voc doric OR part sg fut act neut nom doric OR part sg fut act neut acc doric OR part sg fut act masc voc doric OR part sg fut act neut voc doric OR part sg fut act neut acc doric OR part sg fut act neut nom doric OR part sg fut act masc voc doric}
 \item \eintrag{ἄρχει}{ ἄρχω }{to be first,}{verb 3rd sg pres ind act OR verb 2nd sg pres ind mp}
 \item \eintrag{ἄφνω}{ ἄφνω }{unawares, of a sudden,}{adv indeclform}
 \item \eintrag{ἅπαντα}{ ἅπας }{sṃ-,}{adj sg masc acc OR adj pl neut voc OR adj pl neut nom OR adj pl neut acc}
 \item \eintrag{ἅπαντας}{ ἅπας }{sṃ-,}{adj pl masc acc}
 \item \eintrag{ἅπτεσθαι}{ ἅπτω }{fasten}{verb pres inf mp}
 \item \eintrag{Ἀβρούπολιν}{ }{keine Übersetzung gefunden}{Nichts gefunden}
 \item \eintrag{Ἀθήνας}{ Ἀθῆναι }{the city of Athens}{noun sg fem gen epic doric aeolic geog name OR noun pl fem acc geog name}
 \item \eintrag{Ἀθαμάνων}{ }{keine Übersetzung gefunden}{Nichts gefunden}
 \item \eintrag{Ἀθαμᾶνος}{ }{keine Übersetzung gefunden}{Nichts gefunden}
 \item \eintrag{Ἀθηναίους}{ Ἀθηναῖος }{Athenian, of}{adj pl masc acc}
 \item \eintrag{Ἀθηναίων}{ Ἀθήναια Ἀθήναιον Ἀθηναῖος }{keine Übersetzung gefunden}{noun pl neut gen OR noun pl neut gen OR adj pl neut gen OR adj pl masc gen OR adj pl fem gen}
 \item \eintrag{Ἀκαρνανίαν}{ }{keine Übersetzung gefunden}{Nichts gefunden}
 \item \eintrag{Ἀλέξανδρον}{ Ἀλέξανδρος ἀλέξανδρος }{Alexander}{noun sg masc acc OR adj sg neut voc OR adj sg neut nom OR adj sg masc acc OR adj sg neut acc OR adj sg fem acc}
 \item \eintrag{Ἀλέξανδρος}{ Ἀλέξανδρος ἀλέξανδρος }{Alexander}{noun sg masc nom OR adj sg masc nom OR adj sg fem nom}
 \item \eintrag{Ἀλλʼ}{ }{keine Übersetzung gefunden}{Nichts gefunden}
 \item \eintrag{Ἀμβρακίαν}{ }{keine Übersetzung gefunden}{Nichts gefunden}
 \item \eintrag{Ἀμυνάνδρου}{ }{keine Übersetzung gefunden}{Nichts gefunden}
 \item \eintrag{Ἀνδρόνικον}{ }{keine Übersetzung gefunden}{Nichts gefunden}
 \item \eintrag{Ἀννίβαν}{ }{keine Übersetzung gefunden}{Nichts gefunden}
 \item \eintrag{Ἀννίβου}{ }{keine Übersetzung gefunden}{Nichts gefunden}
 \item \eintrag{Ἀντίοχον}{ }{keine Übersetzung gefunden}{Nichts gefunden}
 \item \eintrag{Ἀντίοχος}{ }{keine Übersetzung gefunden}{Nichts gefunden}
 \item \eintrag{Ἀντιόχου}{ ἀντοχέομαι }{drive}{verb 2nd sg pres imperat mp attic poetic contr raw preverb OR verb 2nd sg imperf ind mp attic poetic contr unaugmented raw preverb}
 \item \eintrag{Ἀντιόχῳ}{ }{keine Übersetzung gefunden}{Nichts gefunden}
 \item \eintrag{Ἀργεάδῃσιν}{ }{keine Übersetzung gefunden}{Nichts gefunden}
 \item \eintrag{Ἀρθέταυρον}{ }{keine Übersetzung gefunden}{Nichts gefunden}
 \item \eintrag{Ἀρθέταυρόν}{ }{keine Übersetzung gefunden}{Nichts gefunden}
 \item \eintrag{Ἀριαράθην}{ }{keine Übersetzung gefunden}{Nichts gefunden}
 \item \eintrag{Ἀσίαν}{ Ἄσιος Ἀσία ἄσιος }{Asian}{adj sg fem acc attic doric aeolic OR adj pl fem gen doric OR adj pl masc gen doric OR noun sg fem acc attic doric ionic aeolic OR noun pl fem gen doric ionic aeolic OR adj sg fem acc attic doric aeolic OR adj pl fem gen doric OR adj pl masc gen doric}
 \item \eintrag{Ἀσίας}{ Ἄσιος Ἀσία Ἀσιάς ἄσιος ἄσις }{Asian}{adj sg fem gen attic doric aeolic OR adj pl fem acc OR noun sg fem gen attic doric ionic aeolic OR noun pl fem acc ionic OR noun sg fem nom OR adj pl fem acc OR adj sg fem gen attic doric aeolic OR noun pl fem acc epic doric ionic aeolic}
 \item \eintrag{Ἀττάλου}{ }{keine Übersetzung gefunden}{Nichts gefunden}
 \item \eintrag{Ἀττικὴν}{ Ἀττικός }{Attic, Athenian,}{adj sg fem acc attic epic ionic}
 \item \eintrag{Ἀχαιῶν}{ Ἀχαιός }{Achaean,}{adj pl neut gen OR adj pl fem gen OR adj pl masc gen}
 \item \eintrag{Ἄτταλον}{ }{keine Übersetzung gefunden}{Nichts gefunden}
 \item \eintrag{Ἅρπαλόν}{ }{keine Übersetzung gefunden}{Nichts gefunden}
 \item \eintrag{ἐγένετο}{ γίγνομαι }{come into a new state of being}{verb 3rd sg aor ind mid}
 \item \eintrag{ἐγένοντο}{ γίγνομαι }{come into a new state of being}{verb 3rd pl aor ind mid}
 \item \eintrag{ἐγίγνετο}{ γίγνομαι }{come into a new state of being}{verb 3rd sg imperf ind mp pres redupl}
 \item \eintrag{ἐγίγνοντο}{ γίγνομαι }{come into a new state of being}{verb 3rd pl imperf ind mp pres redupl}
 \item \eintrag{ἐγγραφῆναι}{ ἐγγράφω }{make incisions into}{verb aor inf pass}
 \item \eintrag{ἐγκαλέσαι}{ ἐγκαλέω }{call in}{verb aor inf act OR verb 3rd sg aor opt act OR verb 2nd sg aor imperat mid}
 \item \eintrag{ἐγκειμένων}{ ἔγκειμαι }{lie in, be wrapped in}{part pl pres mp masc gen OR part pl pres mp neut gen OR part pl pres mp fem gen OR part pl perf mp neut gen OR part pl perf mp fem gen OR part pl perf mp masc gen}
 \item \eintrag{ἐγκλημάτων}{ ἔγκλημα }{accusation, charge}{noun pl neut gen}
 \item \eintrag{ἐγὼ}{ ἐγώ }{I at least, for my part, indeed, for myself}{pron 1st sg masc voc indeclform OR pron 1st sg masc nom indeclform OR pron 1st sg fem voc indeclform OR pron 1st sg fem nom indeclform}
 \item \eintrag{ἐδέξαντο}{ δέχομαι δείκνυμι }{take, accept, receive,}{verb 3rd pl aor ind mid OR verb 3rd pl aor ind mid ionic}
 \item \eintrag{ἐδέξατο}{ δέχομαι δείκνυμι }{take, accept, receive,}{verb 3rd sg aor ind mid OR verb 3rd sg aor ind mid ionic}
 \item \eintrag{ἐδήλωσε}{ δηλόω }{to make visible}{verb 3rd sg aor ind act}
 \item \eintrag{ἐδίδασκε}{ διδάσκω }{instruct}{verb 3rd sg imperf ind act pres redupl}
 \item \eintrag{ἐδεδοίκειν}{ δείδω }{to fear}{verb 1st sg plup ind act attic epic ionic}
 \item \eintrag{ἐδεδώκει}{ δίδωμι }{Aër.}{verb 3rd sg plup ind act attic epic contr}
 \item \eintrag{ἐδράσατε}{ δράω δράω2 }{do, accomplish,}{verb 2nd pl aor ind act attic epic doric aeolic OR verb 2nd pl aor ind act attic epic doric aeolic}
 \item \eintrag{ἐδόκει}{ δοκέω }{expect}{verb 3rd sg imperf ind act attic epic contr}
 \item \eintrag{ἐδόκουν}{ δοκέω δοκόω }{expect}{verb 3rd pl imperf ind act attic epic doric contr OR verb 1st sg imperf ind act attic epic doric contr OR verb 3rd pl imperf ind act contr OR verb 1st sg imperf ind act contr}
 \item \eintrag{ἐδῄου}{ δηιόω }{to cut down, slay}{verb 3rd sg imperf ind act attic epic contr OR verb 2nd sg imperf ind mp attic epic contr}
 \item \eintrag{ἐθέλοι}{ ἐθέλω }{to be willing}{verb 3rd sg pres opt act}
 \item \eintrag{ἐθίζοντες}{ ἐθίζω }{accustom}{part pl pres act masc nom OR part pl pres act masc voc}
 \item \eintrag{ἐκ}{ ἐκ }{from out of,}{prep proclitic indeclform}
 \item \eintrag{ἐκάλει}{ καλέω }{call, summon}{verb 3rd sg imperf ind act attic epic contr}
 \item \eintrag{ἐκάλουν}{ καλέω }{call, summon}{verb 3rd pl imperf ind act attic epic doric contr OR verb 1st sg imperf ind act attic epic doric contr}
 \item \eintrag{ἐκέλευε}{ κελεύω }{urge, drive on}{verb 3rd sg imperf ind act}
 \item \eintrag{ἐκέλευσαν}{ κελεύω }{urge, drive on}{verb 3rd pl aor ind act}
 \item \eintrag{ἐκέλευσε}{ κελεύω }{urge, drive on}{verb 3rd sg aor ind act}
 \item \eintrag{ἐκέλευσεν}{ κελεύω }{urge, drive on}{verb 3rd sg aor ind act nu movable}
 \item \eintrag{ἐκήρυττεν}{ κηρύσσω }{to be a herald, officiate as herald}{verb 3rd sg imperf ind act attic nu movable}
 \item \eintrag{ἐκβιασθέντες}{ ἐκβιάζω }{to force out, dislodge, expel,}{part pl aor pass masc voc OR part pl aor pass masc nom}
 \item \eintrag{ἐκείνοις}{ ἐκεῖνος }{the person there, that person}{adj pl neut dat OR adj pl masc dat}
 \item \eintrag{ἐκείνου}{ ἐκεῖνος κενόω }{the person there, that person}{adj sg neut gen OR adj sg masc gen OR verb 3rd sg imperf ind act epic contr OR verb 2nd sg imperf ind mp epic contr}
 \item \eintrag{ἐκεκράγεσαν}{ κράζω }{croak}{verb 3rd pl plup ind act}
 \item \eintrag{ἐκεῖθεν}{ ἐκεῖθεν }{from that place, thence,}{adv indeclform}
 \item \eintrag{ἐκκλησίαν}{ ἐκκλησία ἐκκλησιάζω }{assembly duly summoned,}{noun sg fem acc attic doric aeolic OR noun pl fem gen doric aeolic OR verb fut inf act OR part sg fut act neut voc doric aeolic contr OR part sg fut act neut nom doric aeolic contr OR part sg fut act neut acc doric aeolic contr OR part sg fut act masc voc doric aeolic contr OR part sg fut act masc nom doric aeolic contr}
 \item \eintrag{ἐκκλησίας}{ ἔκκλησις ἐκκλησία ἐκκλησιάζω }{appeal,}{noun pl fem acc epic doric ionic aeolic OR noun pl fem acc OR noun sg fem gen attic doric aeolic OR verb 2nd sg fut ind act doric contr}
 \item \eintrag{ἐκλήθονται}{ ἐκλανθάνω }{escape notice utterly}{verb 3rd pl pres ind mp}
 \item \eintrag{ἐκλύσειεν}{ ἐκλύω }{set free,}{verb 3rd sg aor opt act nu movable}
 \item \eintrag{ἐκλύων}{ ἐκλύω }{set free,}{part sg pres act masc nom}
 \item \eintrag{ἐκπολεμήσαντες}{ ἐκπολεμέω }{provoke to war,}{part pl aor act masc nom OR part pl aor act masc voc}
 \item \eintrag{ἐκστῆναι}{ ἐξίστημι }{displace}{verb aor inf act}
 \item \eintrag{ἐκταράσσουσαν}{ ἐκταράσσω }{throw into confusion,}{part pl pres act fem gen doric OR part sg pres act fem acc attic epic doric ionic}
 \item \eintrag{ἐκτετρῦσθαι}{ τρύω τρύζω }{Erster Bericht}{verb perf inf mp OR verb perf inf mp redupl}
 \item \eintrag{ἐκόμισεν}{ κομίζω }{take care of, provide for}{verb 3rd sg aor ind act nu movable}
 \item \eintrag{ἐκύρουν}{ κυρέω κυρόω }{hit, light upon}{verb 3rd pl imperf ind act attic epic doric contr OR verb 1st sg imperf ind act attic epic doric contr OR verb 1st sg imperf ind act contr OR verb 3rd pl imperf ind act contr}
 \item \eintrag{ἐκώλυσε}{ κωλύω }{hinder, prevent}{verb 3rd sg aor ind act}
 \item \eintrag{ἐλέγξαι}{ ἐλέγχω }{disgrace, put to shame}{verb aor inf act OR verb 3rd sg aor opt act OR verb 2nd sg aor imperat mid}
 \item \eintrag{ἐλέλυντο}{ λύω }{luo}{verb 3rd pl plup ind mp}
 \item \eintrag{ἐλαυνομένων}{ ἐλαύνω }{drive, set in motion}{part pl pres mp neut gen OR part pl pres mp masc gen OR part pl pres mp fem gen}
 \item \eintrag{ἐλαύνοντος}{ ἐλαύνω }{drive, set in motion}{part sg pres act masc gen OR part sg pres act neut gen}
 \item \eintrag{ἐλευθέρας}{ ἐλεύθερος }{free}{adj sg fem gen attic doric aeolic OR adj pl fem acc}
 \item \eintrag{ἐλθούσης}{ ἔρχομαι }{ibo}{part sg aor act fem gen attic epic ionic}
 \item \eintrag{ἐλθόντας}{ ἔρχομαι }{ibo}{part pl aor act masc acc}
 \item \eintrag{ἐλπίζων}{ ἐλπίζω }{hope for}{part sg pres act masc nom}
 \item \eintrag{ἐλπίσας}{ ἐλπίζω }{hope for}{part sg aor act masc nom attic epic ionic OR part sg aor act masc voc attic epic ionic OR verb 2nd sg aor ind act homeric ionic unaugmented}
 \item \eintrag{ἐλυμαίνετο}{ λυμαίνομαι λυμαίνομαι2 }{cleanse from dirt}{verb 3rd sg imperf ind mp OR verb 3rd sg imperf ind mp}
 \item \eintrag{ἐμέμφεσθε}{ μέμφομαι }{blame, censure}{verb 2nd pl imperf ind mp}
 \item \eintrag{ἐμέμψαντο}{ μέμφομαι }{blame, censure}{verb 3rd pl aor ind mid}
 \item \eintrag{ἐμήνυον}{ μηνύω }{disclose what is secret, reveal}{verb 3rd pl imperf ind act OR verb 1st sg imperf ind act}
 \item \eintrag{ἐμήνυσαν}{ μηνύω }{disclose what is secret, reveal}{verb 3rd pl aor ind act}
 \item \eintrag{ἐμαρτύρησεν}{ μαρτυρέω }{bear witness, give evidence}{verb 3rd sg aor ind act nu movable}
 \item \eintrag{ἐμαυτῷ}{ ἐμαυτοῦ }{of me, of myself}{pron sg neut dat indeclform OR pron sg masc dat indeclform}
 \item \eintrag{ἐμμένωσιν}{ ἐμμένω }{abide in}{verb 3rd pl pres subj act nu movable}
 \item \eintrag{ἐμπρησμὸν}{ ἐμπρησμός }{keine Übersetzung gefunden}{noun sg masc acc}
 \item \eintrag{ἐμὲ}{ ἐγώ ἐμός }{I at least, for my part, indeed, for myself}{pron 1st sg masc acc indeclform OR pron 1st sg fem acc indeclform OR adj sg masc voc}
 \item \eintrag{ἐν}{ ἐν εἰμί εἰς }{in, into.}{prep proclitic indeclform OR verb 3rd pl imperf ind act epic OR prep doric aeolic proclitic indeclform}
 \item \eintrag{ἐνίους}{ ἔνιοι }{some;}{adj pl masc acc}
 \item \eintrag{ἐναντίον}{ ἐναντίον ἐναντίος }{keine Übersetzung gefunden}{adv indeclform OR adj sg neut nom OR adj sg neut voc OR adj sg neut acc OR adj sg masc acc}
 \item \eintrag{ἐνεκάλουν}{ ἐγκαλέω }{call in}{verb 3rd pl imperf ind act attic epic doric contr OR verb 1st sg imperf ind act attic epic doric contr}
 \item \eintrag{ἐνθάδʼ}{ }{keine Übersetzung gefunden}{Nichts gefunden}
 \item \eintrag{ἐνθύμιος}{ ἐνθύμιος }{taken to heart, weighing upon the mind,}{adj sg fem nom OR adj sg masc nom}
 \item \eintrag{ἐνοχλούμενοι}{ ἐνοχλέω }{trouble, annoy,}{part pl pres mp masc voc attic epic doric contr OR part pl pres mp masc nom attic epic doric contr}
 \item \eintrag{ἐνῆγε}{ ἐνάγω }{lead in}{verb 3rd sg imperf ind act attic epic ionic}
 \item \eintrag{ἐξ}{ ἐξ ἐκ }{keine Übersetzung gefunden}{prep indeclform OR prep proclitic indeclform}
 \item \eintrag{ἐξάγειν}{ ἐξάγω }{lead out, lead away}{verb pres inf act attic epic contr}
 \item \eintrag{ἐξέβαλε}{ ἐκβάλλω }{throw}{verb 3rd sg aor ind act}
 \item \eintrag{ἐξέλοι}{ ἐξαιρέω }{take out,}{verb 3rd sg aor opt act OR verb 3rd sg fut opt act attic epic doric contr}
 \item \eintrag{ἐξέτασιν}{ ἐξέτασις ἐξετάζω }{close examination, scrutiny, test,}{noun sg fem acc OR part pl fut act neut dat doric nu movable OR part pl fut act masc dat doric nu movable}
 \item \eintrag{ἐξήλατο}{ ἐξάλλομαι ἐξελαύνω }{leap out of}{verb 3rd sg aor ind mid OR verb 3rd sg imperf ind mp epic poetic contr rare}
 \item \eintrag{ἐξαγαγεῖν}{ ἐξάγω }{lead out, lead away}{verb aor inf act attic epic doric contr redupl}
 \item \eintrag{ἐξαγγελεῖν}{ ἐξαγγέλλω }{tell out, proclaim, make known,}{verb fut inf act attic epic doric contr}
 \item \eintrag{ἐξαιρεθῆναι}{ ἐξαιρέω }{take out,}{verb aor inf pass}
 \item \eintrag{ἐξεκήρυξεν}{ ἐκκηρύσσω }{proclaim by voice of herald}{verb 3rd sg aor ind act nu movable}
 \item \eintrag{ἐξηγρίωσαν}{ ἐξαγριόω }{make wild}{verb 3rd pl aor ind act attic epic ionic}
 \item \eintrag{ἐξιέναι}{ ἔξειμι2 ἐξίημι }{sum}{verb pres inf act OR verb pres inf act}
 \item \eintrag{ἐξῆρχε}{ ἐξάρχω }{begin, take the lead in, initiate,}{verb 3rd sg perf ind act OR verb 3rd sg imperf ind act attic epic ionic OR verb 2nd sg perf imperat act}
 \item \eintrag{ἐπʼ}{ }{keine Übersetzung gefunden}{Nichts gefunden}
 \item \eintrag{ἐπέβαλλον}{ ἐπιβάλλω }{throw}{verb 3rd pl imperf ind act OR verb 1st sg imperf ind act}
 \item \eintrag{ἐπέκρυπτεν}{ ἐπικρύπτω }{throw a}{verb 3rd sg imperf ind act nu movable}
 \item \eintrag{ἐπέτρεχεν}{ ἐπιτρέχω }{—run upon}{verb 3rd sg imperf ind act nu movable}
 \item \eintrag{ἐπήγετο}{ ἐπάγω πήγνυμι }{bring on,}{verb 3rd sg imperf ind mp attic epic ionic OR verb 3rd sg imperf ind mp}
 \item \eintrag{ἐπί}{ ἐπί ἐπί2 }{on, upon with gen., dat., and acc.}{prep indeclform OR prep indeclform}
 \item \eintrag{ἐπίφθονος}{ ἐπίφθονος }{liable to envy}{adj sg masc nom OR adj sg fem nom}
 \item \eintrag{ἐπαγγείλας}{ ἐπαγγέλλω }{tell, proclaim, announce,}{part sg aor act masc voc attic epic ionic OR verb 2nd sg aor ind act doric aeolic OR part sg aor act masc nom attic epic ionic}
 \item \eintrag{ἐπαινοῖτο}{ ἐπαινέω }{approve, applaud, commend,}{verb 3rd sg pres opt mp attic epic doric contr}
 \item \eintrag{ἐπαιρόμενον}{ ἐπαίρω }{lift up and set on,}{part sg pres mp neut voc OR part sg pres mp neut nom OR part sg pres mp neut acc OR part sg pres mp masc acc}
 \item \eintrag{ἐπανιόντας}{ ἐπάνειμι }{ibo}{part pl pres act masc acc unasp preverb}
 \item \eintrag{ἐπανιόντι}{ ἐπάνειμι }{ibo}{part sg pres act neut dat unasp preverb OR part sg pres act masc dat unasp preverb}
 \item \eintrag{ἐπανιὼν}{ ἐπάνειμι }{ibo}{part sg pres act masc nom unasp preverb}
 \item \eintrag{ἐπανῄεσαν}{ ἐπάνειμι }{ibo}{verb 3rd pl imperf ind act unasp preverb}
 \item \eintrag{ἐπεβούλευον}{ ἐπιβουλεύω }{plot, contrive against}{verb 1st sg imperf ind act OR verb 3rd pl imperf ind act}
 \item \eintrag{ἐπεδείχθη}{ ἐπιδείκνυμι }{exhibit as a specimen}{verb 3rd sg aor ind pass}
 \item \eintrag{ἐπεκηρυκεύετο}{ ἐπικηρυκεύομαι }{send}{verb 3rd sg imperf ind mp}
 \item \eintrag{ἐπεκύρωσε}{ ἐπικυρόω }{confirm, sanction, ratify}{verb 3rd sg aor ind act}
 \item \eintrag{ἐπελθεῖν}{ ἐπέρχομαι }{come upon}{verb aor inf act attic epic doric contr}
 \item \eintrag{ἐπελθών}{ ἐπέρχομαι }{come upon}{part sg aor act masc nom}
 \item \eintrag{ἐπεποιήκεσαν}{ ποιέω }{make}{verb 3rd pl plup ind act redupl}
 \item \eintrag{ἐπεπόμφει}{ πέμπω }{send}{verb 3rd sg plup ind act attic epic ionic contr}
 \item \eintrag{ἐπεστράτευσεν}{ ἐπιστρατεύω }{march}{verb futperf inf act epic doric redupl OR verb 3rd sg aor ind act nu movable}
 \item \eintrag{ἐπιβουλεύοι}{ ἐπιβουλεύω }{plot, contrive against}{verb 3rd sg pres opt act}
 \item \eintrag{ἐπιβουλεύσας}{ ἐπιβουλεύω }{plot, contrive against}{verb 2nd sg aor ind act homeric ionic unaugmented OR part sg aor act masc voc attic epic ionic OR part sg aor act masc nom attic epic ionic}
 \item \eintrag{ἐπιβουλὴν}{ ἐπιβουλή }{plan formed against}{noun sg fem acc attic epic ionic}
 \item \eintrag{ἐπιγαμίας}{ ἐπιγαμία }{additional marriage}{noun sg fem gen attic doric aeolic OR noun pl fem acc}
 \item \eintrag{ἐπιγελῶν}{ ἐπιγελάω }{—laugh approvingly}{verb 3rd pl imperf ind act homeric ionic contr unaugmented OR verb 1st sg imperf ind act homeric ionic contr unaugmented OR part sg pres act neut nom contr OR part sg pres act neut voc contr OR part sg pres act neut acc contr OR part sg pres act masc voc contr OR part sg fut act neut voc contr OR part sg pres act masc nom attic epic ionic contr OR part sg fut act neut nom contr OR part sg fut act neut acc contr OR part sg fut act masc voc contr OR part sg fut act masc nom attic epic ionic contr}
 \item \eintrag{ἐπιγιγνομένης}{ ἐπιγίγνομαι }{to be born after, come}{part sg pres mp fem gen attic epic ionic pres redupl}
 \item \eintrag{ἐπιδημοῦσι}{ ἐπιδημέω }{to be at home, live at home}{verb 3rd pl pres ind act attic epic doric contr OR part pl pres act neut dat attic epic doric contr OR part pl pres act masc dat attic epic doric contr}
 \item \eintrag{ἐπιδραμεῖται}{ ἐπιτρέχω }{—run upon}{verb 3rd sg fut ind mid attic epic contr}
 \item \eintrag{ἐπιδραμόντα}{ ἐπιτρέχω }{—run upon}{part pl aor act neut voc OR part sg aor act masc acc OR part pl aor act neut nom OR part pl aor act neut acc}
 \item \eintrag{ἐπιθυμίᾳ}{ ἐπιθυμία ἐπιθυμιάω }{desire, yearning}{noun sg fem dat attic doric aeolic OR noun pl fem voc OR noun pl fem nom OR verb 3rd sg pres subj act contr OR verb 3rd sg pres ind act epic contr OR verb 2nd sg pres subj mp contr OR verb 2nd sg pres ind mp epic contr}
 \item \eintrag{ἐπικειμένην}{ ἐπίκειμαι }{to be laid upon}{part sg pres mp fem acc attic epic ionic OR part sg perf mp fem acc attic epic ionic}
 \item \eintrag{ἐπικειμένους}{ ἐπίκειμαι }{to be laid upon}{part pl perf mp masc acc OR part pl pres mp masc acc}
 \item \eintrag{ἐπικυρώσει}{ ἐπικύρωσις ἐπικυρόω }{ratification, confirmation}{noun dual fem voc attic epic contr OR noun sg fem dat attic ionic contr OR noun dual fem nom attic epic contr OR noun dual fem acc attic epic contr OR verb 2nd sg fut ind mid OR verb 3rd sg aor subj act epic short subj OR verb 3rd sg fut ind act doric contr}
 \item \eintrag{ἐπιλαθέσθαι}{ ἐπιλανθάνομαι }{to forget}{verb pres inf mp doric OR verb aor inf mid}
 \item \eintrag{ἐπιλείπειν}{ ἐπιλείπω }{leave behind}{verb pres inf act attic epic contr}
 \item \eintrag{ἐπιμέλειαν}{ ἐπιμέλεια }{care bestowed upon}{noun sg fem acc OR noun pl fem gen doric aeolic}
 \item \eintrag{ἐπιμελῶν}{ ἐπιμελής }{careful}{adj pl masc gen attic epic doric contr OR adj pl neut gen attic epic doric contr OR adj pl fem gen attic epic doric contr}
 \item \eintrag{ἐπιπλέοντας}{ ἐπιπλέω }{sail upon}{part pl pres act masc acc epic doric ionic aeolic}
 \item \eintrag{ἐπιπολάσαι}{ ἐπιπολάζω }{to come to the surface, float}{verb aor inf act OR verb 2nd sg aor imperat mid OR verb 3rd sg aor opt act comp only OR part pl fut act fem voc doric comp only OR part sg fut act fem dat doric OR part pl fut act fem nom doric comp only}
 \item \eintrag{ἐπισκήπτειν}{ ἐπισκήπτω }{make to lean upon}{verb pres inf act attic epic contr}
 \item \eintrag{ἐπισκεψομένας}{ ἐπισκέπτομαι }{pass in review}{part sg fut mp fem gen doric aeolic OR part pl fut mp fem acc}
 \item \eintrag{ἐπιστέλλοντος}{ ἐπιστέλλω }{send to}{part sg pres act neut gen OR part sg pres act masc gen OR part sg aor act neut gen OR part sg aor act masc gen}
 \item \eintrag{ἐπιτρέψειε}{ ἐπιτρέπω }{to turn to}{verb 3rd sg aor opt act}
 \item \eintrag{ἐπιτρίβουσιν}{ ἐπιτρίβω }{—rub on the surface, crush,}{verb 3rd pl pres ind act attic epic doric ionic nu movable OR verb 3rd pl aor subj pass epic contr nu movable short subj OR part pl pres act neut dat attic epic doric ionic nu movable OR part pl pres act masc dat attic epic doric ionic nu movable}
 \item \eintrag{ἐπιτυχὼν}{ ἐπιτυγχάνω ἐπιτυχής }{hit the mark,}{part sg aor act masc nom OR adj pl neut gen attic epic doric contr OR adj pl masc gen attic epic doric contr OR adj pl fem gen attic epic doric contr}
 \item \eintrag{ἐπιχειρεῖν}{ ἐπιχειρέω }{put one's hand to,}{verb pres inf act attic epic doric comp only contr}
 \item \eintrag{ἐπιόντων}{ ἔπειμι ἔπειμι2 }{sum}{verb 3rd pl pres imperat act attic poetic raw preverb OR part pl pres act neut gen doric unasp preverb OR part pl pres act masc gen doric unasp preverb OR verb 3rd pl pres imperat act unasp preverb OR part pl pres act neut gen unasp preverb OR part pl pres act masc gen unasp preverb}
 \item \eintrag{ἐπλάττετο}{ πλάσσω πλήσσω }{form, mould}{verb 3rd sg imperf ind mp attic OR verb 3rd sg imperf ind mp attic}
 \item \eintrag{ἐποίει}{ ἐποίζω ποιέω }{lament over,}{verb 3rd sg fut ind act attic epic doric ionic contr OR verb 2nd sg fut ind mid attic epic doric ionic contr OR verb 3rd sg imperf ind act attic epic contr}
 \item \eintrag{ἐπολιόρκει}{ πολιορκέω }{besiege,}{verb 3rd sg imperf ind act attic epic contr}
 \item \eintrag{ἐπολυπραγμόνουν}{ πολυπραγμονέω }{to be busy about many things,}{verb 3rd pl imperf ind act attic epic doric contr OR verb 1st sg imperf ind act attic epic doric contr}
 \item \eintrag{ἐπρέσβευεν}{ πρεσβεύω }{to be the elder}{verb 3rd sg imperf ind act nu movable}
 \item \eintrag{ἐπρέσβευσε}{ πρεσβεύω }{to be the elder}{verb 3rd sg aor ind act}
 \item \eintrag{ἐπόρθησε}{ πορθέω }{destroy, ravage, plunder,}{verb 3rd sg aor ind act}
 \item \eintrag{ἐπώνυμον}{ ἐπώνυμος }{given as a significant name,}{adj sg neut nom OR adj sg neut voc OR adj sg neut acc OR adj sg masc acc OR adj sg fem acc}
 \item \eintrag{ἐπὶ}{ ἐπί ἐπί2 }{on, upon with gen., dat., and acc.}{prep indeclform OR prep indeclform}
 \item \eintrag{ἐργασαμένους}{ ἐργάζομαι }{work, labour,}{part pl aor mp masc acc attic}
 \item \eintrag{ἐρρίπτουν}{ ῥίπτω ῥιπτέω }{throw, cast, hurl}{verb 1st sg imperf ind act attic epic doric contr OR verb 3rd pl imperf ind act attic epic doric contr OR verb 1st sg imperf ind act attic epic doric contr OR verb 3rd pl imperf ind act attic epic doric contr}
 \item \eintrag{ἐς}{ εἰς }{into}{prep proclitic indeclform}
 \item \eintrag{ἐσέβαλεν}{ εἰσβάλλω }{throw into,}{verb 3rd sg aor ind act nu movable short eis}
 \item \eintrag{ἐσήμηνεν}{ σημαίνω }{show by a sign, indicate, point out}{verb 3rd sg aor ind act nu movable}
 \item \eintrag{ἐσενεγκεῖν}{ εἰσφέρω }{carry in,}{verb aor inf act attic epic doric contr short eis}
 \item \eintrag{ἐστίν}{ εἰμί }{sum}{verb 3rd sg pres ind act enclitic nu movable}
 \item \eintrag{ἐστι}{ εἰμί }{sum}{verb 3rd sg pres ind act enclitic}
 \item \eintrag{ἐστὶ}{ εἰμί }{sum}{verb 3rd sg pres ind act enclitic}
 \item \eintrag{ἐτάραττεν}{ ταράσσω }{stir, trouble}{verb 3rd sg imperf ind act attic nu movable}
 \item \eintrag{ἐτελεύτα}{ τελευτάω }{bring to pass, accomplish}{verb 3rd sg imperf ind act contr}
 \item \eintrag{ἐτράπετο}{ τρέπω }{Studien zum griech. Perf.}{verb 3rd sg aor ind mid}
 \item \eintrag{ἐφʼ}{ }{keine Übersetzung gefunden}{Nichts gefunden}
 \item \eintrag{ἐφάνη}{ φαίνω φανάω }{A ren.}{verb 3rd sg aor ind pass OR verb 3rd sg imperf ind act doric contr}
 \item \eintrag{ἐφαίνετο}{ φαίνω }{A ren.}{verb 3rd sg imperf ind mp}
 \item \eintrag{ἐφείσαντο}{ ἐφίζω φείδομαι }{set upon}{verb 3rd pl aor ind mid OR verb 3rd pl aor ind mid}
 \item \eintrag{ἐφεδρεύουσαν}{ ἐφεδρεύω }{sit upon, rest upon}{part sg pres act fem acc attic epic doric ionic OR part pl pres act fem gen doric}
 \item \eintrag{ἐφθόνησε}{ φθονέω }{bear ill-will}{verb 3rd sg aor ind act}
 \item \eintrag{ἐχθρόν}{ ἐχθρός }{hated, hateful}{adj sg neut nom OR adj sg neut voc OR adj sg masc acc OR adj sg neut acc}
 \item \eintrag{ἐχθρός}{ ἐχθρός }{hated, hateful}{adj sg masc nom}
 \item \eintrag{ἐχθρὸν}{ ἐχθρός }{hated, hateful}{adj sg neut nom OR adj sg neut voc OR adj sg masc acc OR adj sg neut acc}
 \item \eintrag{ἐχθρῶν}{ ἔχθρα ἔχθρη ἐχθρός }{hatred, enmity}{noun pl fem gen OR noun pl fem gen OR adj pl neut gen OR adj pl masc gen OR adj pl fem gen}
 \item \eintrag{ἐχορήγει}{ χορηγέω }{lead a chorus,}{verb 3rd sg imperf ind act attic epic contr}
 \item \eintrag{ἐψεύσατο}{ ψεύδω ψεύδω2 }{cheat by lies, beguile,}{verb 3rd sg aor ind mid OR verb 3rd sg aor ind mid}
 \item \eintrag{ἐψηφίζοντο}{ ψηφίζω }{count, reckon,}{verb 3rd pl imperf ind mp}
 \item \eintrag{ἐᾶν}{ ἐάν ἐάω }{if haply, if}{conj indeclform OR verb pres inf act epic doric contr OR part sg pres act neut voc doric aeolic contr OR verb 1st sg imperf ind act doric aeolic poetic syll augment contr unaugmented OR verb 3rd pl imperf ind act doric aeolic poetic syll augment contr unaugmented OR part sg pres act neut nom doric aeolic contr OR part sg pres act masc voc doric aeolic contr OR part sg pres act neut acc doric aeolic contr OR part sg pres act masc nom doric aeolic contr}
 \item \eintrag{ἑαλώκοι}{ ἁλίσκομαι }{to be taken, conquered, fall into an enemy's hand}{verb 3rd sg perf opt act attic attic redupl}
 \item \eintrag{ἑαυτοὺς}{ ἑαυτοῦ }{Stadtrecht von Gortyn}{adj pl masc acc}
 \item \eintrag{ἑαυτοῦ}{ ἑαυτοῦ }{Stadtrecht von Gortyn}{adj sg masc gen OR adj sg neut gen}
 \item \eintrag{ἑαυτὸν}{ ἑαυτοῦ }{Stadtrecht von Gortyn}{adj sg masc acc}
 \item \eintrag{ἑαυτῶν}{ ἑαυτοῦ }{Stadtrecht von Gortyn}{adj pl fem gen OR adj pl masc gen OR adj pl neut gen}
 \item \eintrag{ἑκάστου}{ ἕκαστος }{each,}{adj sg neut gen OR adj sg masc gen}
 \item \eintrag{ἑκάστῳ}{ ἕκαστος }{each,}{adj sg masc dat OR adj sg neut dat}
 \item \eintrag{ἑκατέρωθεν}{ ἑκατέρωθεν }{on each side, on either hand,}{adv indeclform}
 \item \eintrag{ἑκατέρων}{ ἑκάτερος ἑκατερέω }{each of two, each singly,}{adj pl masc gen OR adj pl neut gen OR adj pl fem gen OR part sg pres act masc nom attic epic doric contr}
 \item \eintrag{ἑλεῖν}{ αἱρέω }{take with the hand, grasp, seize}{verb fut inf act attic epic doric contr OR verb aor inf act attic epic doric contr}
 \item \eintrag{ἑνὸς}{ ἕνος ἕνος εἷς }{belonging to the former of two periods}{adj sg masc nom OR adj sg masc nom OR noun sg neut gen indeclform OR noun sg masc gen indeclform}
 \item \eintrag{ἑξήρους}{ ἑξήρης }{with six banks of oars,}{adj sg fem gen attic epic doric contr OR adj sg masc gen attic epic doric contr OR adj sg neut gen attic epic doric contr}
 \item \eintrag{ἑξηκοντούτης}{ ἑξηκοντούτης }{keine Übersetzung gefunden}{adj sg masc nom OR adj pl masc voc doric aeolic contr OR adj sg fem nom OR adj pl masc nom doric aeolic contr OR adj pl masc acc attic epic doric contr OR adj pl fem voc doric aeolic contr OR adj pl fem nom doric aeolic contr OR adj pl fem acc attic epic doric contr}
 \item \eintrag{ἑορτὴν}{ ἑορτή }{feast, festival, holiday,}{noun sg fem acc attic epic ionic}
 \item \eintrag{ἑσπερίοισιν}{ ἑσπέριος }{at even, at eventide}{adj pl neut dat epic ionic aeolic nu movable OR adj pl masc dat epic ionic aeolic nu movable OR adj pl fem dat epic ionic aeolic nu movable}
 \item \eintrag{ἑτέροις}{ ἕτερος }{D Mort.}{adj pl neut dat OR adj pl masc dat}
 \item \eintrag{ἑτέρους}{ ἕτερος }{D Mort.}{adj pl masc acc}
 \item \eintrag{ἑτέρωθεν}{ ἑτέρωθεν }{from the other side}{adv indeclform}
 \item \eintrag{ἑτέρων}{ ἕτερος }{D Mort.}{adj pl masc gen OR adj pl neut gen OR adj pl fem gen}
 \item \eintrag{ἑτέρῳ}{ ἕτερος }{D Mort.}{adj sg masc dat OR adj sg neut dat}
 \item \eintrag{ἔγεμε}{ γέμω }{to be full}{verb 3rd sg imperf ind act}
 \item \eintrag{ἔγκλημα}{ ἔγκλημα }{accusation, charge}{noun sg neut voc OR noun sg neut nom OR noun sg neut acc}
 \item \eintrag{ἔδει}{ δέω δέω2 δεῖ ἔδω ἐσθίω }{bind, tie, fetter,}{verb 3rd sg imperf ind act attic epic contr OR verb 3rd sg imperf ind act attic epic contr OR verb 3rd sg imperf ind act attic epic ionic contr impersonal OR verb 2nd sg pres ind mp epic OR verb 3rd sg pres ind act epic OR verb 2nd sg fut ind mid epic doric contr OR verb 2nd sg fut ind mid}
 \item \eintrag{ἔδησεν}{ δέω δέω2 }{bind, tie, fetter,}{verb 3rd sg aor ind act nu movable OR verb 3rd sg aor ind act nu movable}
 \item \eintrag{ἔθει}{ ἔθος ἔθω θέω θέω2 }{custom, habit}{noun sg neut dat epic ionic OR noun dual neut voc attic epic contr OR noun dual neut nom attic epic contr OR noun dual neut acc attic epic contr OR verb 3rd sg pres ind act OR verb 2nd sg pres ind mp OR verb 3rd sg imperf ind act attic epic contr OR verb 3rd sg imperf ind act attic epic contr}
 \item \eintrag{ἔθνεσι}{ ἔθνος }{number of people living together, company, body of men}{noun pl neut dat}
 \item \eintrag{ἔθνη}{ ἔθνος }{number of people living together, company, body of men}{noun pl neut voc attic epic doric contr OR noun pl neut nom attic epic doric contr OR noun pl neut acc attic epic doric contr OR noun dual neut voc doric aeolic contr OR noun dual neut nom doric aeolic contr OR noun dual neut acc doric aeolic contr}
 \item \eintrag{ἔθνους}{ ἔθνος }{number of people living together, company, body of men}{noun sg neut gen attic epic doric contr}
 \item \eintrag{ἔθους}{ ἔθος }{custom, habit}{noun sg neut gen attic epic doric contr}
 \item \eintrag{ἔκπαλαι}{ ἔκπαλαι ἐκπάλλω }{for a long time,}{adv indeclform OR verb aor inf act doric OR verb 3rd sg aor opt act doric OR verb 2nd sg aor imperat mid doric}
 \item \eintrag{ἔκρινε}{ κρίνω }{separate, put asunder, distinguish}{verb 3rd sg aor ind act OR verb 3rd sg imperf ind act}
 \item \eintrag{ἔκτειναν}{ ἐκτείνω κτείνω }{stretch out,}{verb 3rd pl aor ind act homeric ionic unaugmented OR part sg aor act neut voc OR part sg aor act neut nom OR part sg aor act neut acc OR verb 3rd pl aor ind act}
 \item \eintrag{ἔκτεινεν}{ ἐκτείνω κτείνω }{stretch out,}{verb 3rd sg imperf ind act homeric ionic unaugmented nu movable OR verb pres inf act epic doric OR verb 3rd sg aor ind act homeric ionic unaugmented nu movable OR verb 3rd sg imperf ind act nu movable OR verb 3rd sg aor ind act nu movable}
 \item \eintrag{ἔλαβε}{ λαμβάνω }{a}{verb 3rd sg aor ind act}
 \item \eintrag{ἔλεγον}{ ἔλεγος λέγω λέγω2 λέγω3 }{song, melody}{noun sg masc acc OR verb 3rd pl imperf ind act OR verb 1st sg imperf ind act OR verb 3rd pl imperf ind act OR verb 1st sg imperf ind act OR verb 3rd pl imperf ind act OR verb 1st sg imperf ind act}
 \item \eintrag{ἔληξεν}{ λήγω }{stay, abate}{verb 3rd sg aor ind act nu movable}
 \item \eintrag{ἔμαθεν}{ μανθάνω }{learn}{verb 3rd sg aor ind act nu movable}
 \item \eintrag{ἔναγχος}{ ἔναγχος }{just now, lately}{adv indeclform}
 \item \eintrag{ἔνθα}{ ἔνθα }{there,}{adv indeclform}
 \item \eintrag{ἔξωθεν}{ ἔξωθεν }{from without}{adv indeclform}
 \item \eintrag{ἔπαθεν}{ πάσχω }{have}{verb 3rd sg aor ind act nu movable}
 \item \eintrag{ἔπειθον}{ πείθω }{persuade}{verb 3rd pl imperf ind act OR verb 1st sg imperf ind act}
 \item \eintrag{ἔπεμπε}{ πέμπω }{send}{verb 3rd sg imperf ind act}
 \item \eintrag{ἔπεμπεν}{ πέμπω }{send}{verb 3rd sg imperf ind act nu movable}
 \item \eintrag{ἔπεμπον}{ πέμπω }{send}{verb 3rd pl imperf ind act OR verb 1st sg imperf ind act}
 \item \eintrag{ἔπεμψαν}{ πέμπω }{send}{verb 3rd pl aor ind act}
 \item \eintrag{ἔπεμψεν}{ πέμπω }{send}{verb 3rd sg aor ind act nu movable}
 \item \eintrag{ἔπραττεν}{ πράσσω }{pass through, pass over,}{verb 3rd sg imperf ind act attic nu movable}
 \item \eintrag{ἔργῳ}{ ἔργον }{weorc}{noun sg neut dat}
 \item \eintrag{ἔστι}{ εἰμί }{sum}{verb 3rd sg pres ind act enclitic}
 \item \eintrag{ἔτεσι}{ ἔτος }{year}{noun pl neut dat}
 \item \eintrag{ἔτι}{ ἔτι }{yet, still}{adv indeclform}
 \item \eintrag{ἔτους}{ ἔτος }{year}{noun sg neut gen attic epic doric contr}
 \item \eintrag{ἔφασκον}{ φάσκω }{say, affirm, assert,}{verb 3rd pl imperf ind act OR verb 1st sg imperf ind act}
 \item \eintrag{ἔφη}{ φημί }{Spir. Prooem., Eratosth.Prooem.}{verb 3rd sg imperf ind act}
 \item \eintrag{ἔφθανον}{ φθάνω }{come}{verb 1st sg imperf ind act OR verb 3rd pl imperf ind act}
 \item \eintrag{ἔχει}{ ἔχις ἔχω ἔχω2 χάω χέω }{viper}{noun sg masc dat epic OR noun dual masc voc attic epic contr OR noun dual masc nom attic epic contr OR noun dual masc acc attic epic contr OR verb 3rd sg pres ind act OR verb 2nd sg pres ind mp OR verb 3rd sg pres ind act OR verb 2nd sg pres ind mp OR verb 3rd sg imperf ind act attic epic ionic contr OR verb 3rd sg imperf ind act attic epic contr}
 \item \eintrag{ἔχειν}{ ἔχω ἔχω2 }{check}{verb pres inf act attic epic contr OR verb pres inf act attic epic contr}
 \item \eintrag{ἔχθος}{ ἔχθος ἐχθός }{hate}{noun sg neut acc OR noun sg neut nom OR noun sg neut voc OR adv indeclform}
 \item \eintrag{ἔχθρας}{ ἔχθρα ἔχθρη ἐχθρός }{hatred, enmity}{noun sg fem gen attic doric aeolic OR noun pl fem acc OR noun sg fem gen attic doric ionic aeolic OR noun pl fem acc OR adj sg fem gen attic doric aeolic OR adj pl fem acc}
 \item \eintrag{ἔχοι}{ ἔχω ἔχω2 }{check}{verb 3rd sg pres opt act OR verb 3rd sg pres opt act}
 \item \eintrag{ἔχοντα}{ ἔχω ἔχω2 }{check}{part sg pres act masc acc OR part pl pres act neut voc OR part pl pres act neut nom OR part pl pres act neut acc OR part sg pres act masc acc OR part pl pres act neut voc OR part pl pres act neut nom OR part pl pres act neut acc}
 \item \eintrag{ἔχοντας}{ ἔχω ἔχω2 }{check}{part pl pres act masc acc OR part pl pres act masc acc}
 \item \eintrag{ἔχοντος}{ ἔχω ἔχω2 }{check}{part sg pres act neut gen OR part sg pres act masc gen OR part sg pres act neut gen OR part sg pres act masc gen}
 \item \eintrag{ἔχων}{ ἔχω ἔχω2 χάω χόω }{check}{part sg pres act masc nom OR part sg pres act masc nom OR verb 3rd pl imperf ind act contr OR verb 1st sg imperf ind act contr OR verb 3rd pl imperf ind act doric aeolic contr OR verb 1st sg imperf ind act doric aeolic contr}
 \item \eintrag{ἕξειν}{ ἔχω ἔχω2 }{check}{verb fut inf act doric contr OR verb fut inf act doric contr}
 \item \eintrag{ἕξων}{ ἔχω ἔχω2 }{check}{part sg fut act masc nom doric contr OR part sg fut act masc nom doric contr}
 \item \eintrag{ἕτερα}{ ἕτερος }{D Mort.}{adj sg fem voc attic doric aeolic OR adj pl neut voc OR adj sg fem nom attic doric aeolic OR adj pl neut acc OR adj pl neut nom OR adj dual fem acc OR adj dual fem nom OR adj dual fem voc}
 \item \eintrag{ἕτεροι}{ ἕτερος }{D Mort.}{adj pl masc nom OR adj pl masc voc}
 \item \eintrag{ἕως}{ ἕως ἑός ἠώς }{until, till}{conj indeclform OR adv OR adj pl masc acc doric OR noun sg fem nom attic OR noun sg fem gen attic doric aeolic contr OR noun pl fem voc attic doric aeolic contr OR noun pl fem acc attic contr OR noun pl fem nom attic doric aeolic contr}
 \item \eintrag{Ἐκτελεσθέντος}{ ἐκτελέω }{bring to an end, accomplish, achieve,}{part sg aor pass neut gen OR part sg aor pass masc gen}
 \item \eintrag{Ἐρέννιον}{ }{keine Übersetzung gefunden}{Nichts gefunden}
 \item \eintrag{Ἐρέννιος}{ }{keine Übersetzung gefunden}{Nichts gefunden}
 \item \eintrag{Ἑλλάδα}{ Ἑλλάς }{part of Phthiotis}{noun sg fem acc}
 \item \eintrag{Ἑλλάδι}{ Ἑλλάς }{part of Phthiotis}{noun sg fem dat}
 \item \eintrag{Ἑλλάδος}{ Ἑλλάς }{part of Phthiotis}{noun sg fem gen}
 \item \eintrag{Ἑλλήνων}{ Ἕλλην }{the Thessalian tribe of which Hellen was the reputed chief}{noun pl fem gen OR noun pl masc gen}
 \item \eintrag{Ἑλλήσποντον}{ Ἑλλήσποντος }{sea of Helle}{noun sg masc acc}
 \item \eintrag{Ἑλληνίδες}{ Ἑλληνίς }{Grecian woman}{noun pl fem voc OR noun pl fem nom}
 \item \eintrag{Ἕλληνας}{ Ἕλλην }{the Thessalian tribe of which Hellen was the reputed chief}{noun pl masc acc OR noun pl fem acc}
 \item \eintrag{Ἕλληνες}{ Ἕλλην }{the Thessalian tribe of which Hellen was the reputed chief}{noun pl masc voc OR noun pl masc nom OR noun pl fem voc OR noun pl fem nom}
 \item \eintrag{Ἕλλησι}{ Ἕλλην }{the Thessalian tribe of which Hellen was the reputed chief}{noun pl masc dat OR noun pl fem dat}
 \item \eintrag{Ἕλλησιν}{ Ἕλλην }{the Thessalian tribe of which Hellen was the reputed chief}{noun pl masc dat nu movable OR noun pl fem dat nu movable}
 \item \eintrag{ἠγανάκτει}{ ἀγανακτέω }{feel a violent irritation,}{verb 3rd sg imperf ind act attic epic ionic contr}
 \item \eintrag{ἠδικηκότας}{ ἀδικέω }{to be}{part pl perf act masc acc attic epic doric ionic aeolic redupl}
 \item \eintrag{ἠκολούθουν}{ ἀκολουθέω }{follow}{verb 3rd pl imperf ind act attic epic doric ionic contr OR verb 1st sg imperf ind act attic epic doric ionic contr}
 \item \eintrag{ἠξίουν}{ ἀξιόω ἐξιόω }{think, deem worthy,}{verb 3rd pl imperf ind act attic epic ionic contr OR verb 1st sg imperf ind act attic epic ionic contr OR verb 3rd pl imperf ind act contr OR verb 1st sg imperf ind act contr}
 \item \eintrag{ἠπείγετο}{ ἐπείγω }{press by weight,}{verb 3rd sg imperf ind mp}
 \item \eintrag{ἠπόρουν}{ ἀπορέω ἀπορέω2 }{keine Übersetzung gefunden}{verb 3rd pl imperf ind act attic epic doric ionic contr OR verb 1st sg imperf ind act attic epic doric ionic contr OR verb 3rd pl imperf ind act attic epic doric ionic contr OR verb 1st sg imperf ind act attic epic doric ionic contr}
 \item \eintrag{ἠρέθιζεν}{ ἐρεθίζω }{rouse to anger, rouse to fight,}{verb 3rd sg imperf ind act nu movable}
 \item \eintrag{ἠρεμήσοντος}{ ἠρεμέω }{to be still, keep quiet, be at rest}{verb sg futperf ind act neut gen redupl OR verb sg futperf ind act masc gen redupl OR part sg fut act neut gen OR part sg fut act masc gen}
 \item \eintrag{ἡ}{ ἧ ὅς ὅς ὁ }{where,}{adv indeclform OR pron sg fem nom attic homeric ionic indeclform OR pron sg fem nom attic homeric ionic indeclform OR article sg fem voc proclitic indeclform OR article sg fem nom proclitic indeclform}
 \item \eintrag{ἡγεμόνι}{ ἡγεμονίς ἡγεμών }{imperial}{noun sg fem voc OR noun sg masc dat OR noun sg fem dat}
 \item \eintrag{ἡγεῖτο}{ ἁγέομαι ἡγέομαι }{custom, prescription}{verb 3rd sg imperf ind mp attic epic ionic contr OR verb 3rd sg pres opt mid epic ionic OR verb 3rd sg imperf ind mid attic epic contr}
 \item \eintrag{ἡγουμένοις}{ ἡγέομαι }{go before, lead the way}{part pl pres mid neut dat attic epic doric contr OR part pl pres mid masc dat attic epic doric contr}
 \item \eintrag{ἡγουμένων}{ ἡγέομαι }{go before, lead the way}{part pl pres mid neut gen attic epic doric contr OR part pl pres mid fem gen attic epic doric contr OR part pl pres mid masc gen attic epic doric contr}
 \item \eintrag{ἡγούμενον}{ ἡγέομαι }{go before, lead the way}{part sg pres mid neut nom attic epic doric contr OR part sg pres mid neut voc attic epic doric contr OR part sg pres mid neut acc attic epic doric contr OR part sg pres mid masc acc attic epic doric contr}
 \item \eintrag{ἡγούμενος}{ ἡγέομαι }{go before, lead the way}{part sg pres mid masc nom attic epic doric contr}
 \item \eintrag{ἡγῆται}{ ἁγέομαι ἡγέομαι ἡγητήρ ἡγητής }{custom, prescription}{verb 3rd sg perf ind mp attic epic doric ionic aeolic redupl OR verb 3rd sg pres subj mid contr OR verb 3rd sg perf ind mid redupl OR verb 3rd sg pres ind mid doric aeolic contr OR noun pl masc voc OR noun sg masc dat doric aeolic OR noun pl masc nom OR noun pl masc voc OR noun sg masc dat doric aeolic OR noun pl masc nom}
 \item \eintrag{ἡδομένους}{ ἥδομαι }{swād-}{part pl pres mp masc acc}
 \item \eintrag{ἡμέραις}{ ἥμερος ἡμέρα }{tame}{adj pl fem dat OR noun pl fem dat ionic}
 \item \eintrag{ἡμέρας}{ ἥμερος ἡμέρα }{tame}{adj pl fem acc OR adj sg fem gen attic doric aeolic OR noun sg fem gen attic doric ionic aeolic OR noun pl fem acc ionic}
 \item \eintrag{ἡμέτεροι}{ ἡμέτερος }{our}{adj pl masc nom OR adj pl masc voc}
 \item \eintrag{ἡμίσεα}{ ἥμισυς ἡμίσεια }{half}{adj sg fem voc epic ionic OR adj sg fem nom epic ionic OR adj pl neut nom epic ionic OR adj pl neut voc epic ionic OR adj dual fem voc epic ionic OR adj pl neut acc epic ionic OR adj dual fem acc epic ionic OR adj dual fem nom epic ionic OR noun pl neut voc OR noun pl neut acc OR noun pl neut nom}
 \item \eintrag{ἡμερῶν}{ ἥμερος ἡμέρα ἡμερόω }{tame}{adj pl neut gen OR adj pl masc gen OR adj pl fem gen OR noun pl fem gen ionic OR verb pres inf act doric OR verb 3rd pl imperf ind act doric aeolic poetic contr unaugmented OR verb 1st sg imperf ind act doric aeolic poetic contr unaugmented OR part sg pres act neut voc doric aeolic contr OR part sg pres act neut nom doric aeolic contr OR part sg pres act masc voc doric aeolic contr OR part sg pres act neut acc doric aeolic contr OR part sg pres act masc nom contr}
 \item \eintrag{ἡμετέροις}{ ἡμέτερος }{our}{adj pl neut dat OR adj pl masc dat}
 \item \eintrag{ἡμετέρους}{ ἡμέτερος }{our}{adj pl masc acc}
 \item \eintrag{ἡσυχίας}{ ἡσυχία }{rest, quiet}{noun sg fem gen attic doric aeolic OR noun pl fem acc}
 \item \eintrag{ἡτοιμάζετο}{ ἑτοιμάζω }{get ready, prepare}{verb 3rd sg imperf ind mp}
 \item \eintrag{ἡττηθεὶς}{ ἡσσάομαι }{to be less}{part sg aor mp masc voc attic OR part sg aor mp masc nom attic}
 \item \eintrag{ἢ}{ εἰμί ἤ ἤ2 ἦ ἦ2 ἦ3 ἦ4 ἠμί }{sum}{verb 1st sg imperf ind act attic OR conj indeclform OR exclam indeclform OR adv indeclform OR adv indeclform OR adv indeclform OR adv indeclform OR verb 3rd sg imperf ind act attic}
 \item \eintrag{ἣ}{ ἧ ὅς ὅς ὁ }{where,}{adv indeclform OR pron sg fem nom attic homeric ionic indeclform OR pron sg fem nom attic homeric ionic indeclform OR article sg fem voc proclitic indeclform OR article sg fem nom proclitic indeclform}
 \item \eintrag{ἣν}{ ὅς ὅς }{yas, yā, yad,}{pron sg fem acc attic homeric ionic indeclform OR pron sg fem acc attic homeric ionic indeclform}
 \item \eintrag{ἤγαγεν}{ ἄγω }{lead, carry, fetch, bring}{verb 3rd sg aor ind act attic epic ionic nu movable redupl}
 \item \eintrag{ἤδη}{ ἤδη ἤδη2 ἦδος }{already, by this time}{adv indeclform OR adv indeclform OR noun pl neut voc attic epic doric contr OR noun pl neut nom attic epic doric contr OR noun pl neut acc attic epic doric contr OR noun dual neut voc doric aeolic contr OR noun dual neut nom doric aeolic contr OR noun dual neut acc doric aeolic contr}
 \item \eintrag{ἤθεσι}{ ἦθος }{an accustomed place}{noun pl neut dat}
 \item \eintrag{ἤκουσεν}{ ἀκούω }{hear}{verb 3rd sg aor ind act attic epic ionic nu movable}
 \item \eintrag{ἤλπισεν}{ ἐλπίζω }{hope for}{verb 3rd sg aor ind act nu movable}
 \item \eintrag{ἤρετο}{ αἴρω ἔρομαι }{to take up, raise, lift up}{verb 3rd sg aor ind mid attic epic ionic OR verb 3rd sg aor ind mid}
 \item \eintrag{ἤτοι}{ ἤτοι }{now surely, truly, verily}{partic indeclform}
 \item \eintrag{ἤχθετο}{ ἄχθομαι ἔχθω ἔχθω2 }{to be loaded,}{verb 3rd sg imperf ind mp attic epic ionic OR verb 3rd sg imperf ind mp OR verb 3rd sg imperf ind mp}
 \item \eintrag{ἥδιστος}{ ἡδύς }{pleasant}{adj sg masc nom irreg superl}
 \item \eintrag{ἥμισυ}{ ἥμισυς }{half}{adj sg neut voc OR adj sg neut nom OR adj sg neut acc OR adj sg masc voc}
 \item \eintrag{ἥττης}{ ἧσσα }{defeat, discomfiture}{noun sg fem gen attic epic ionic}
 \item \eintrag{ἦν}{ ἐάν εἰμί ἤν ἤν2 ἠμί }{if haply, if}{conj contr indeclform OR verb 3rd pl imperf ind act epic doric aeolic OR verb 3rd sg imperf ind act OR verb 1st sg imperf ind act OR exclam indeclform OR exclam indeclform OR verb 1st sg imperf ind act attic}
 \item \eintrag{ἦρχεν}{ ἄρχω }{to be first,}{verb 3rd sg perf ind act nu movable OR verb perf inf act epic poetic OR verb 3rd sg imperf ind act attic epic ionic nu movable OR verb 3rd pl plup ind act epic doric aeolic}
 \item \eintrag{ἦσαν}{ ἀσάω εἰμί }{glut oneself, take a surfeit,}{verb 3rd pl imperf ind act attic epic doric ionic aeolic contr OR verb 1st sg imperf ind act attic epic doric ionic aeolic contr OR verb 3rd pl imperf ind act}
 \item \eintrag{ἧκεν}{ ἥκω ἵημι }{to have come, be present}{verb pres inf act epic doric OR verb 3rd sg imperf ind act nu movable OR verb 3rd sg aor ind act nu movable}
 \item \eintrag{ἧς}{ εἷς ὅς ὅς }{sem}{noun sg masc nom doric indeclform OR pron sg fem gen attic homeric ionic indeclform OR pron sg fem gen attic homeric ionic indeclform}
 \item \eintrag{ἧττον}{ ἥσσων }{inferior}{adj sg neut voc comp attic OR adj sg neut nom comp attic OR adj sg neut acc comp attic OR adj sg masc voc comp attic OR adj sg fem voc comp attic}
 \item \eintrag{Ἠπειρωτῶν}{ ἠπειρόω ἠπειρώτης }{to make into mainland}{verb 3rd pl pres imperat act doric aeolic poetic contr OR verb 3rd dual pres imperat act doric aeolic contr OR noun pl masc gen}
 \item \eintrag{Ἡ}{ ἧ ὅς ὅς ὁ }{where,}{adv indeclform OR pron sg fem nom attic homeric ionic indeclform OR pron sg fem nom attic homeric ionic indeclform OR article sg fem voc proclitic indeclform OR article sg fem nom proclitic indeclform}
 \item \eintrag{Ἡλίου}{ ἥλιος ἡλιόομαι ἡλιόω }{sun}{noun sg masc gen OR verb 2nd sg pres imperat mp contr OR verb 2nd sg imperf ind mp homeric ionic contr unaugmented OR verb 3rd sg imperf ind act homeric ionic contr unaugmented OR verb 2nd sg pres imperat mp contr OR verb 2nd sg pres imperat act contr OR verb 2nd sg imperf ind mp homeric ionic contr unaugmented}
 \item \eintrag{Ἤπειρον}{ ἤπειρος }{terra firma, land}{noun sg fem acc}
 \item \eintrag{ἰδίας}{ ἴδιος ἰδιάζω }{one's own, pertaining to oneself}{adj sg fem gen attic doric aeolic OR adj pl fem acc OR verb 2nd sg fut ind act doric contr}
 \item \eintrag{ἰδίοις}{ ἴδιος ἰδέω ἰδίω }{one's own, pertaining to oneself}{adj pl neut dat OR adj pl fem dat OR adj pl masc dat OR verb 2nd sg pres opt act doric OR verb 2nd sg pres opt act}
 \item \eintrag{ἰδεῖν}{ εἶδον ἰδέω }{to see}{verb aor inf act attic epic doric contr OR verb pres inf act attic epic doric contr}
 \item \eintrag{ἰδὼν}{ Ἴδη εἶδον ἴδη ἴδη2 ἶδος ἰδέω }{Ida}{noun pl fem gen OR part sg aor act masc nom OR noun pl fem gen OR noun pl fem gen OR noun pl neut gen attic epic doric contr OR part sg pres act masc nom attic epic doric contr}
 \item \eintrag{ἰσχυρῶς}{ ἰσχυρός ἰσχυρόω }{strong}{adj pl masc acc doric OR adv OR verb 2nd sg pres ind act doric contr OR verb 2nd sg imperf ind act doric aeolic contr}
 \item \eintrag{ἱερῶν}{ ἱερά ἱεράζω ἱερή ἱερόν ἱερός ἱερόω }{serpent}{noun pl fem gen OR noun pl neut gen OR part sg fut act neut voc contr OR part sg fut act neut nom contr OR part sg fut act neut acc contr OR part sg fut act masc nom attic epic ionic contr OR part sg fut act masc voc contr OR noun pl fem gen OR noun pl neut gen OR adj pl fem gen OR adj pl masc gen OR adj pl neut gen OR verb 1st sg imperf ind act doric aeolic poetic contr unaugmented OR verb 3rd pl imperf ind act doric aeolic contr OR verb pres inf act doric OR part sg pres act neut voc doric aeolic contr OR part sg pres act neut nom doric aeolic contr OR part sg pres act masc nom contr OR part sg pres act masc voc doric aeolic contr OR part sg pres act neut acc doric aeolic contr}
 \item \eintrag{ἱκανὰ}{ ἱκανός }{sufficing, becoming, befitting}{adj sg fem voc doric aeolic OR adj sg fem nom doric aeolic OR adj pl neut voc OR adj pl neut nom OR adj pl neut acc OR adj dual fem voc OR adj dual fem nom OR adj dual fem acc}
 \item \eintrag{ἱκανὴν}{ ἱκάνω ἱκανός }{come}{verb pres inf act doric aeolic contr OR adj sg fem acc attic epic ionic}
 \item \eintrag{ἱππέας}{ ἱππεύς }{one who fights from a chariot}{noun pl masc acc}
 \item \eintrag{ἱππεῖ}{ ἱππεύς }{one who fights from a chariot}{noun sg masc dat epic poetic}
 \item \eintrag{ἵνα}{ ἵνα }{in that place, there}{conj indeclform OR adv indeclform}
 \item \eintrag{ἵππους}{ ἵππος }{horse}{noun pl masc acc OR noun pl fem acc}
 \item \eintrag{Ἰβηρίαν}{ }{keine Übersetzung gefunden}{Nichts gefunden}
 \item \eintrag{Ἰλλυρίδα}{ }{keine Übersetzung gefunden}{Nichts gefunden}
 \item \eintrag{Ἰλλυριοῖς}{ Ἰλλυριοί }{region}{noun pl masc dat}
 \item \eintrag{Ἰλλυριῶν}{ Ἰλλυριοί }{region}{noun pl masc gen}
 \item \eintrag{Ἰσθμίων}{ Ἴσθμια Ἴσθμιος Ἰσθμιάζω ἴσθμιον ἴσθμιος }{keine Übersetzung gefunden}{noun pl neut gen OR noun pl masc gen OR adj pl neut gen OR adj pl masc gen OR adj pl fem gen OR part sg fut act neut nom contr OR part sg fut act neut voc contr OR part sg fut act neut acc contr OR part sg fut act masc nom attic epic ionic contr OR part sg fut act masc voc contr OR noun pl neut gen geog name OR adj pl masc gen geog name OR adj pl neut gen geog name OR adj pl fem gen geog name}
 \item \eintrag{Ἰταλίαν}{ Ἰταλία Ἰταλιάζω }{Italy,}{noun sg fem acc attic doric aeolic OR noun pl fem gen doric aeolic OR part sg fut act neut voc doric aeolic contr OR verb fut inf act OR part sg fut act neut acc doric aeolic contr OR part sg fut act neut nom doric aeolic contr OR part sg fut act masc nom doric aeolic contr OR part sg fut act masc voc doric aeolic contr}
 \item \eintrag{Ἰταλίας}{ Ἰταλία Ἰταλιάζω }{Italy,}{noun sg fem gen attic doric aeolic OR noun pl fem acc OR verb 2nd sg fut ind act doric contr}
 \item \eintrag{Ἰωνίαν}{ Ἰώνιος }{keine Übersetzung gefunden}{adj sg fem acc attic doric aeolic OR adj pl masc gen doric OR adj pl fem gen doric}
 \item \eintrag{Ἴστρον}{ Ἴστρος }{Ister, Danube}{noun sg masc acc OR noun sg fem acc}
 \item \eintrag{ὀκνεῖ}{ ὀκνέω }{shrink from}{verb 3rd sg imperf ind act attic epic contr unaugmented OR verb 3rd sg pres ind act attic epic doric ionic contr OR verb 2nd sg pres ind mp attic epic doric ionic contr OR verb 2nd sg pres imperat act attic epic contr}
 \item \eintrag{ὀλέσσει}{ ὄλλυμι }{destroy, make an end of,}{verb 3rd sg fut ind act epic doric contr OR verb 3rd sg aor subj act epic short subj OR verb 2nd sg fut ind mid epic doric contr}
 \item \eintrag{ὀλίγαις}{ ὀλίγος }{little, small,}{adj pl fem dat}
 \item \eintrag{ὀλίγον}{ ὀλίγος }{little, small,}{adj sg neut voc OR adj sg neut nom OR adj sg neut acc OR adj sg masc acc}
 \item \eintrag{ὀλίγῳ}{ ὀλίγος }{little, small,}{adj sg neut dat OR adj sg masc dat}
 \item \eintrag{ὀλεῖται}{ ὄλλυμι }{destroy, make an end of,}{verb 3rd sg fut ind mid attic epic contr}
 \item \eintrag{ὀλιγότητα}{ ὀλιγότης }{fewness,}{noun sg fem acc}
 \item \eintrag{ὀμόσαντος}{ ὄμνυμι }{swear,}{part sg aor act neut gen OR part sg aor act masc gen}
 \item \eintrag{ὀξέως}{ ὀξέως ὀξύς }{keine Übersetzung gefunden}{adv indeclform OR adv}
 \item \eintrag{ὀπίσω}{ ὀπίσω ὀπίζω }{backwards,}{adv indeclform OR verb 2nd sg aor ind mid homeric ionic unaugmented OR verb 1st sg fut ind act doric contr OR verb 1st sg aor subj act}
 \item \eintrag{ὀφλήσει}{ ὄφλησις ὀφλισκάνω }{penalty}{noun sg fem dat epic OR noun dual fem nom attic epic contr OR noun dual fem voc attic epic contr OR noun dual fem acc attic epic contr OR verb 2nd sg fut ind mid OR verb 3rd sg fut ind act doric contr}
 \item \eintrag{ὁ}{ ὅς ὅς ὁ }{yas, yā, yad,}{pron sg neut voc indeclform OR pron sg neut nom indeclform OR pron sg neut acc indeclform OR pron sg neut voc indeclform OR pron sg neut nom indeclform OR pron sg neut acc indeclform OR article sg masc nom proclitic indeclform}
 \item \eintrag{ὁδεῦσαι}{ ὁδάω ὁδεύω ὁδόω }{export and sell}{part pl pres act fem voc epic doric ionic contr OR part sg pres act fem dat epic doric ionic contr OR part pl pres act fem nom epic doric ionic contr OR verb aor inf act OR verb 3rd sg aor opt act OR verb 2nd sg aor imperat mid OR part pl pres act fem voc epic ionic contr OR part pl pres act fem nom epic ionic contr}
 \item \eintrag{ὁδοποιῶν}{ ὁδοποιέω ὁδοποιός }{make}{part sg pres act masc nom attic epic doric contr OR noun pl masc gen}
 \item \eintrag{ὁδοῖς}{ ὁδάω ὁδός ὁδός ὁδόω }{export and sell}{verb 2nd sg pres opt act attic epic doric ionic contr OR noun pl masc dat OR noun pl fem dat OR verb 2nd sg pres subj act contr OR verb 2nd sg pres opt act contr OR verb 2nd sg pres ind act contr}
 \item \eintrag{ὁδὸν}{ ὁδός ὁδός }{way, road,}{noun sg masc acc OR noun sg fem acc}
 \item \eintrag{ὁμηρείας}{ ὁμηρεία }{giving of hostages}{noun pl fem acc OR noun sg fem gen attic doric aeolic}
 \item \eintrag{ὁμηρεύοντα}{ ὁμηρεύω ὁμηρεύω2 ὁμηρεύω3 }{to be}{part sg pres act masc acc OR part pl pres act neut voc OR part pl pres act neut nom OR part pl pres act neut acc OR part sg pres act masc acc OR part pl pres act neut voc OR part pl pres act neut nom OR part pl pres act neut acc OR part sg pres act masc acc OR part pl pres act neut voc OR part pl pres act neut nom OR part pl pres act neut acc}
 \item \eintrag{ὁμολογουμένως}{ ὁμολογέω ὁμολογουμένως }{to be}{part pl pres mp masc acc doric contr OR adv indeclform}
 \item \eintrag{ὁμοῦ}{ ὁμός ὁμόω ὁμοῦ }{one and the same, common, joint,}{adj sg neut gen OR adj sg masc gen OR verb 3rd sg imperf ind act homeric ionic contr unaugmented OR verb 2nd sg pres imperat mp contr OR verb 2nd sg pres imperat act contr OR verb 2nd sg imperf ind mp homeric ionic contr unaugmented OR adv indeclform}
 \item \eintrag{ὁπλότερος}{ ὁπλότερος }{the younger,}{adj sg masc nom}
 \item \eintrag{ὁρμητήριον}{ ὁρμητήριον }{stimulant, incentive,}{noun sg neut voc OR noun sg neut nom OR noun sg neut acc}
 \item \eintrag{ὁρῶν}{ ὅρος ὁράω }{boundary, landmark,}{noun pl masc gen OR verb 3rd pl imperf ind act homeric ionic contr unaugmented OR verb 1st sg imperf ind act homeric ionic contr unaugmented OR part sg pres act neut voc epic contr OR part sg pres act neut nom epic contr OR part sg pres act masc voc epic contr OR part sg pres act neut acc epic contr OR part sg pres act masc nom attic epic ionic contr}
 \item \eintrag{ὃ}{ ὅς ὅς ὁ }{yas, yā, yad,}{pron sg neut voc indeclform OR pron sg neut nom indeclform OR pron sg neut acc indeclform OR pron sg neut voc indeclform OR pron sg neut nom indeclform OR pron sg neut acc indeclform OR article sg masc nom proclitic indeclform}
 \item \eintrag{ὄντα}{ εἰμί ὄντα }{sum}{part pl pres act neut voc OR part sg pres act masc acc OR part pl pres act neut acc OR part pl pres act neut nom OR noun pl neut nom indeclform OR noun pl neut voc indeclform OR noun pl neut acc indeclform}
 \item \eintrag{ὄντι}{ εἰμί }{sum}{part sg pres act neut dat OR part sg pres act masc dat}
 \item \eintrag{ὄντος}{ εἰμί }{sum}{part sg pres act neut gen OR part sg pres act masc gen}
 \item \eintrag{ὄντων}{ εἰμί ὄντα }{sum}{verb 3rd pl pres imperat act attic OR part pl pres act neut gen OR part pl pres act masc gen OR noun pl neut gen indeclform}
 \item \eintrag{ὄψιν}{ ὄψις }{aspect, appearance}{noun sg fem acc}
 \item \eintrag{ὅ}{ ὅς ὅς ὁ }{yas, yā, yad,}{pron sg neut voc indeclform OR pron sg neut nom indeclform OR pron sg neut acc indeclform OR pron sg neut voc indeclform OR pron sg neut nom indeclform OR pron sg neut acc indeclform OR article sg masc nom proclitic indeclform}
 \item \eintrag{ὅδε}{ ὅδε ὁδός ὁδός }{this,}{pron sg masc nom indeclform OR noun sg masc voc OR noun sg fem voc}
 \item \eintrag{ὅθεν}{ ὅθεν }{whence,}{adv indeclform}
 \item \eintrag{ὅλης}{ ὅλοξ ὅλος ὁλάω }{whole, entire, complete}{adj sg fem gen attic epic ionic OR adj sg fem gen attic epic ionic OR verb 2nd sg pres ind act doric contr OR verb 2nd sg imperf ind act doric poetic contr unaugmented}
 \item \eintrag{ὅλως}{ ὅλοξ ὅλος }{whole, entire, complete}{adv OR adj pl masc acc doric OR adv OR adj pl masc acc doric}
 \item \eintrag{ὅλῳ}{ ὅλοξ ὅλος ὁλάω }{whole, entire, complete}{adj sg masc dat OR adj sg neut dat OR adj sg masc dat OR adj sg neut dat OR verb 3rd sg pres opt act contr}
 \item \eintrag{ὅμηρά}{ ὅμηρος }{pledge, surety, hostage,}{noun pl neut nom OR noun pl neut voc OR noun pl neut acc}
 \item \eintrag{ὅμως}{ ὅμως ὅμως ὁμός ὁμόω ὁμῶς ὁμῶς }{all the same, nevertheless, notwithstanding, still}{conj indeclform OR conj indeclform OR adj pl masc acc doric OR verb 2nd sg pres ind act doric contr OR verb 2nd sg imperf ind act doric aeolic poetic contr unaugmented OR adv indeclform OR adv indeclform}
 \item \eintrag{ὅπερ}{ ὅσπερ }{the very man who, the very thing which}{pron sg neut voc indeclform OR pron sg neut acc indeclform OR pron sg neut nom indeclform}
 \item \eintrag{ὅπως}{ ὅπως ὅπως ὅπωϲ }{as, in such manner as,}{conj indeclform OR conj indeclform OR conj indeclform OR conj indeclform OR conj indeclform}
 \item \eintrag{ὅρκους}{ ὅρκος ὁρκόω }{the object by which one swears,}{noun pl masc acc OR verb 2nd sg pres ind act doric contr OR verb 2nd sg imperf ind act homeric ionic contr unaugmented}
 \item \eintrag{ὅσα}{ ὅσος }{as great as, how great}{adj sg fem voc doric aeolic OR adj sg fem nom doric aeolic OR adj pl neut voc OR adj pl neut nom OR adj pl neut acc OR adj dual fem nom OR adj dual fem voc OR adj dual fem acc}
 \item \eintrag{ὅσαι}{ ὅσος }{as great as, how great}{adj sg fem dat doric aeolic OR adj pl fem voc OR adj pl fem nom}
 \item \eintrag{ὅσας}{ ὅσος }{as great as, how great}{adj sg fem gen doric aeolic OR adj pl fem acc}
 \item \eintrag{ὅσον}{ ὅσος }{as great as, how great}{adj sg neut voc OR adj sg neut nom OR adj sg neut acc OR adj sg masc acc}
 \item \eintrag{ὅτε}{ ὅστε ὅτε ὅτε2 }{who, which,}{pron sg neut voc attic poetic indeclform OR pron sg neut nom attic poetic indeclform OR pron sg neut acc attic poetic indeclform OR conj indeclform OR conj indeclform}
 \item \eintrag{ὅτι}{ ὅστις ὅτι ὅτι2 }{that}{pron sg neut nom indeclform OR pron sg neut acc indeclform OR adv indeclform OR conj indeclform}
 \item \eintrag{Ὁ}{ ὅς ὅς ὁ }{yas, yā, yad,}{pron sg neut voc indeclform OR pron sg neut nom indeclform OR pron sg neut acc indeclform OR pron sg neut voc indeclform OR pron sg neut nom indeclform OR pron sg neut acc indeclform OR article sg masc nom proclitic indeclform}
 \item \eintrag{Ὃς}{ ὅς ὅς }{yas, yā, yad,}{pron sg masc nom indeclform OR pron sg masc nom indeclform}
 \item \eintrag{Ὅτι}{ ὅστις ὅτι ὅτι2 }{that}{pron sg neut nom indeclform OR pron sg neut acc indeclform OR adv indeclform OR conj indeclform}
 \item \eintrag{ὑγιὲς}{ ὑγιής }{healthy, sound}{adj sg fem voc OR adj sg masc voc OR adj sg neut acc OR adj sg neut nom OR adj sg neut voc}
 \item \eintrag{ὑμέτερον}{ ὑμέτερος }{your, yours,}{adj sg masc acc OR adj sg neut acc OR adj sg neut nom OR adj sg neut voc}
 \item \eintrag{ὑμετέρων}{ ὑμέτερος }{your, yours,}{adj pl fem gen OR adj pl masc gen OR adj pl neut gen}
 \item \eintrag{ὑμεῖς}{ σύ }{thou}{pron 2nd pl nom indeclform}
 \item \eintrag{ὑμᾶς}{ σύ ὑμός }{thou}{pron 2nd pl acc indeclform OR adj sg fem gen epic doric aeolic OR adj pl fem acc epic doric}
 \item \eintrag{ὑμῖν}{ σύ }{thou}{pron 2nd pl dat epic poetic indeclform}
 \item \eintrag{ὑμῶν}{ σύ ὑμός }{thou}{pron 2nd pl gen indeclform OR adj pl neut gen epic doric OR adj pl masc gen epic doric OR adj pl fem gen epic doric}
 \item \eintrag{ὑπʼ}{ }{keine Übersetzung gefunden}{Nichts gefunden}
 \item \eintrag{ὑπέστητε}{ ὑφίστημι }{place}{verb 2nd pl aor ind act OR verb 2nd pl perf subj act ionic contr unasp preverb}
 \item \eintrag{ὑπέσχετο}{ ὑπέχω ὑπισχνέομαι }{hold under,}{verb 3rd sg aor ind mid OR verb 3rd sg aor ind mid comp only}
 \item \eintrag{ὑπεκρίνετο}{ ὑποκρίνομαι }{to reply, make answer, answer}{verb 3rd sg imperf ind mp}
 \item \eintrag{ὑπηκόους}{ ὑπήκοος ὑπήκους }{hearkening,}{adj pl fem acc OR adj pl masc acc OR adj pl fem acc OR adj pl masc acc}
 \item \eintrag{ὑπηκόων}{ ὑπήκοον ὑπήκοος ὑπήκους }{horned cummin, Hypecoum procumbens,}{noun pl neut gen OR adj pl fem gen OR adj pl masc gen OR adj pl neut gen OR adj pl fem gen OR adj pl masc gen OR adj pl neut gen}
 \item \eintrag{ὑπισχνεῖτο}{ ὑπισχνέομαι }{take upon oneself,}{verb 3rd sg imperf ind mp attic epic contr OR verb 3rd sg pres opt mp epic ionic}
 \item \eintrag{ὑπισχνούμενος}{ ὑπισχνέομαι }{take upon oneself,}{part sg pres mp masc nom attic epic doric contr}
 \item \eintrag{ὑποβάλλουσιν}{ ὑποβάλλω }{throw, put,}{part pl pres act masc dat attic epic doric ionic nu movable OR part pl pres act neut dat attic epic doric ionic nu movable OR verb 3rd pl pres ind act attic epic doric ionic nu movable}
 \item \eintrag{ὑποδέδεκται}{ ὑποδέχομαι }{receive into one's house, welcome,}{verb 3rd sg perf ind mp}
 \item \eintrag{ὑποδέξασθαι}{ ὑποδέχομαι ὑποδείκνυμι }{receive into one's house, welcome,}{verb aor inf mid OR verb aor inf mid ionic}
 \item \eintrag{ὑποδέχεσθε}{ ὑποδέχομαι }{receive into one's house, welcome,}{verb 2nd pl imperf ind mp homeric ionic unaugmented OR verb 2nd pl pres imperat mp OR verb 2nd pl pres ind mp}
 \item \eintrag{ὑποζύγια}{ ὑποζύγιον }{beast for the yoke, beast of draught}{noun pl neut acc OR noun pl neut nom OR noun pl neut voc}
 \item \eintrag{ὑπολοίπους}{ ὑπόλοιπος }{left over,}{adj pl fem acc OR adj pl masc acc}
 \item \eintrag{ὑπομνήματα}{ ὑπόμνημα }{reminder, memorial,}{noun pl neut voc OR noun pl neut nom OR noun pl neut acc}
 \item \eintrag{ὑποστάντες}{ ὑφίστημι ὑποστάζω }{place}{part pl aor act masc nom OR part pl aor act masc voc OR part pl fut act masc nom doric aeolic contr OR part pl fut act masc voc doric aeolic contr}
 \item \eintrag{ὑπόγυον}{ ὑπόγυιος ὑπόγυος }{nigh at hand, imminent,}{adj sg fem acc OR adj sg masc acc OR adj sg neut acc OR adj sg neut nom OR adj sg neut voc OR adj sg fem acc OR adj sg masc acc OR adj sg neut acc OR adj sg neut nom OR adj sg neut voc}
 \item \eintrag{ὑπόνοιαν}{ ὑπόνοια }{suspicion, conjecture, guess,}{noun pl fem gen doric aeolic OR noun sg fem acc}
 \item \eintrag{ὑπόπτου}{ ὕποπτος ὑπόπτης ὑποπτάω }{viewed with suspicion}{adj sg fem gen OR adj sg masc gen OR adj sg neut gen OR noun sg masc gen OR verb 2nd sg imperf ind mp attic epic ionic contr OR verb 2nd sg pres imperat mp attic epic ionic contr}
 \item \eintrag{ὑπόσχοιντο}{ ὑπέχω ὑπισχνέομαι }{hold under,}{verb 3rd pl aor opt mid OR verb 3rd pl aor opt mid comp only}
 \item \eintrag{ὑπὲρ}{ ὑπέρ }{upaári}{prep indeclform}
 \item \eintrag{ὑπὸ}{ ὑπό }{úpa}{prep indeclform}
 \item \eintrag{ὑφεωρῶντο}{ ὑφοράω }{look at from below, eye stealthily, view with suspicion}{verb 3rd pl imperf ind mid syll augment contr OR verb 3rd pl imperf ind mp syll augment contr}
 \item \eintrag{ὑφορωμένων}{ ὑφοράω }{look at from below, eye stealthily, view with suspicion}{part pl pres mid fem gen contr OR part pl pres mid masc gen contr OR part pl pres mid neut gen contr OR part pl pres mp fem gen contr OR part pl pres mp masc gen contr OR part pl pres mp neut gen contr}
 \item \eintrag{ὑφορώμενος}{ ὑφοράω }{look at from below, eye stealthily, view with suspicion}{part sg pres mp masc nom contr OR part sg pres mid masc nom contr}
 \item \eintrag{ὑφορᾶσθαι}{ ὑφοράω }{look at from below, eye stealthily, view with suspicion}{verb pres inf mid contr OR verb pres inf mp contr}
 \item \eintrag{ὕδατος}{ ὕδωρ }{water,}{noun sg neut gen indeclform}
 \item \eintrag{ὕπατος}{ ὕπατος }{highest, uppermost,}{adj sg masc nom}
 \item \eintrag{ὕποπτος}{ ὕποπτος }{viewed with suspicion}{adj sg fem nom OR adj sg masc nom}
 \item \eintrag{ὕστερον}{ ὕστερον ὕστερος }{the afterbirth,}{noun sg neut voc OR noun sg neut nom OR noun sg neut acc OR adj sg neut voc irreg comp OR adj sg neut nom irreg comp OR adj sg neut acc irreg comp OR adj sg masc acc irreg comp}
 \item \eintrag{ὠμόσατε}{ ὄμνυμι }{swear,}{verb 2nd pl aor ind act}
 \item \eintrag{ὠμὸς}{ ὦμος ὠμός ὠμός }{the shoulder with the upper arm}{noun sg masc nom OR adj sg masc nom OR adj sg masc nom}
 \item \eintrag{ὠχυρωμένην}{ ὀχυρόω }{fortify}{part sg perf mp fem acc attic epic ionic redupl}
 \item \eintrag{ὡμολόγουν}{ ὁμολογέω }{to be}{verb 1st sg imperf ind act attic epic doric contr OR verb 3rd pl imperf ind act attic epic doric contr}
 \item \eintrag{ὡρισμένα}{ ὁρίζω }{divide}{part dual perf mp fem acc redupl OR part dual perf mp fem nom redupl OR part dual perf mp fem voc redupl OR part pl perf mp neut acc redupl OR part pl perf mp neut nom redupl OR part pl perf mp neut voc redupl OR part sg perf mp fem nom doric aeolic redupl OR part sg perf mp fem voc doric aeolic redupl}
 \item \eintrag{ὡς}{ ὅς ὅς ὡς ὡς }{yas, yā, yad,}{pron pl masc acc doric indeclform OR pron pl masc acc doric indeclform OR conj proclitic indeclform OR adv indeclform OR conj proclitic indeclform OR adv indeclform}
 \item \eintrag{ὢν}{ εἰμί οὖν }{sum}{part sg pres act masc nom OR partic doric ionic indeclform}
 \item \eintrag{ὣς}{ ὅς ὅς ὡς ὡς }{yas, yā, yad,}{pron pl masc acc doric indeclform OR pron pl masc acc doric indeclform OR conj proclitic indeclform OR adv indeclform OR conj proclitic indeclform OR adv indeclform}
 \item \eintrag{ὤκνησεν}{ ὀκνέω }{shrink from}{verb 3rd sg aor ind act nu movable OR verb futperf inf act epic doric redupl}
 \item \eintrag{ὤν}{ εἰμί οὖν }{sum}{part sg pres act masc nom OR partic doric ionic indeclform}
 \item \eintrag{ὤνησεν}{ ὀνίνημι }{D Mort.}{verb 3rd sg aor ind act nu movable}
 \item \eintrag{ὤφειλεν}{ ὀφέλλω ὀφέλλω2 ὀφέλλω3 ὀφείλω }{keine Übersetzung gefunden}{verb 3rd sg aor ind act nu movable OR verb 3rd sg aor ind act nu movable OR verb 3rd sg aor ind act nu movable OR verb 3rd sg imperf ind act nu movable}
 \item \eintrag{ὥραις}{ ὥρα ὥρα2 }{sūra.}{noun pl fem dat OR noun pl fem dat}
 \item \eintrag{ὥρισε}{ ὁρίζω }{divide}{verb 3rd sg aor ind act}
 \item \eintrag{ὦ}{ εἰμί ὦ ὦ2 }{sum}{verb 1st sg pres subj act attic epic doric contr OR exclam indeclform OR exclam indeclform}
 \item \eintrag{ὧδε}{ ὧδε }{in this wise, thus,}{adv indeclform}
 \item \eintrag{ὧν}{ ὅς ὅς }{yas, yā, yad,}{pron pl neut gen indeclform OR pron pl masc gen indeclform OR pron pl fem gen indeclform OR pron pl neut gen indeclform OR pron pl masc gen indeclform OR pron pl fem gen indeclform}
 \item \eintrag{Ὧν}{ ὅς ὅς }{yas, yā, yad,}{pron pl neut gen indeclform OR pron pl masc gen indeclform OR pron pl fem gen indeclform OR pron pl neut gen indeclform OR pron pl masc gen indeclform OR pron pl fem gen indeclform}
 \item \eintrag{ᾐσθάνεσθε}{ αἰσθάνομαι }{perceive, apprehend by the senses}{verb 2nd pl imperf ind mid}
 \item \eintrag{ᾐτιῶντο}{ αἰτιάομαι }{accuse, censure}{verb 3rd pl imperf ind mp contr}
 \item \eintrag{ᾑροῦντο}{ αἱρέω }{take with the hand, grasp, seize}{verb 3rd pl imperf ind mp attic epic doric contr}
 \item \eintrag{ᾔτει}{ αἰτέω }{ask, beg}{verb 3rd sg imperf ind act attic epic contr}
 \item \eintrag{ᾗ}{ ᾗ ἵημι ὅς ὅς }{which way, where, whither}{adv indeclform OR verb 3rd sg aor subj act OR verb 2nd sg aor subj mid OR pron sg fem dat attic homeric ionic indeclform OR pron sg fem dat attic homeric ionic indeclform}
 \item \eintrag{ᾧ}{ ὅς ὅς }{yas, yā, yad,}{pron sg masc dat indeclform OR pron sg neut dat indeclform OR pron sg masc dat indeclform OR pron sg neut dat indeclform}
 \item \eintrag{ῥωμαϊζόντων}{ }{keine Übersetzung gefunden}{Nichts gefunden}
 \item \eintrag{ῥᾳδίως}{ ῥᾴδιος }{easy, ready}{adv OR adj pl masc acc doric OR adj pl fem acc attic doric}
 \item \eintrag{Ῥοδίοις}{ Ῥόδιος Ῥοδιακός ῥοδίζω }{Rhodian, of}{adj pl neut dat OR adj pl masc dat OR adj pl neut dat OR adj pl masc dat OR verb 2nd sg fut opt act attic epic doric contr}
 \item \eintrag{Ῥοδίους}{ Ῥόδιος Ῥοδιακός }{Rhodian, of}{adj pl masc acc OR adj pl masc acc}
 \item \eintrag{Ῥοδίων}{ Ῥόδιος Ῥοδιακός ῥοδίζω }{Rhodian, of}{adj pl neut gen OR adj pl masc gen OR adj pl fem gen OR adj pl neut gen OR adj pl masc gen OR adj pl fem gen OR part sg fut act masc nom attic epic doric contr}
 \item \eintrag{Ῥωμαίοις}{ Ῥωμαῖος }{a Roman}{adj pl neut dat OR adj pl masc dat}
 \item \eintrag{Ῥωμαίους}{ Ῥωμαῖος }{a Roman}{adj pl masc acc}
 \item \eintrag{Ῥωμαίων}{ Ῥωμαῖος }{a Roman}{adj pl neut gen OR adj pl masc gen OR adj pl fem gen}
 \item \eintrag{Ῥωμαῖοι}{ Ῥωμαῖος }{a Roman}{adj pl masc nom OR adj pl masc voc}
 \item \eintrag{Ῥωμαῖον}{ Ῥωμαῖος }{a Roman}{adj sg neut voc OR adj sg neut acc OR adj sg neut nom OR adj sg masc acc}
 \item \eintrag{Ῥόδιοι}{ Ῥόδιος Ῥοδιακός ῥοδίζω }{Rhodian, of}{adj pl masc voc OR adj pl masc nom OR adj pl masc voc OR adj pl masc nom OR verb 3rd sg fut opt act attic epic doric contr}
 \item \eintrag{Ῥώμην}{ Ῥώμη ῥώμη }{Roma, Rome}{noun sg fem acc attic epic ionic OR noun sg fem acc attic epic ionic}
 \item \eintrag{Ῥώμης}{ Ῥώμη ῥώμη }{Roma, Rome}{noun sg fem gen attic epic ionic OR noun sg fem gen attic epic ionic}
 \end{itemize}


\end{document}
